%
% Copyright � 2012 Peeter Joot.  All Rights Reserved.
% Licenced as described in the file LICENSE under the root directory of this GIT repository.
%
%
%\chapter{PHY456H1F: Quantum Mechanics II.  Lecture 10 (Taught by Prof J.E. Sipe).  Fermi's golden rule (cont.)}
%\chapter{Fermi's golden rule (cont.)}
\index{Fermi's golden rule}
\label{chap:qmTwoL10}
%\blogpage{http://sites.google.com/site/peeterjoot/math2011/qmTwoL10.pdf}
%\date{Oct 10, 2011}
%
\section{Recap. Where we got to on Fermi's golden rule.}
%
We are continuing on the topic of Fermi golden rule, as also covered in \S 17.2 of the text \citep{desai2009quantum}.  Utilizing a wave train with peaks separation \(\Delta t = 2\pi/\omega_0\), zero before some initial time \cref{fig:qmTwoL10:unitStepSine}.
\imageFigure{../figures/phy456-qmII/unitStepSine}{Sine only after an initial time.}{fig:qmTwoL10:unitStepSine}{0.2}
Perturbing a state in the \(i\)th energy level, and looking at the states for the \(m\)th energy level as illustrated in \cref{fig:qmTwoL10:2}.
\pdfTexFigure{../figures/phy456-qmII/qmTwoL10fig2.pdf_tex}{Perturbation from \(i\) to \(m\)th energy levels.}{fig:qmTwoL10:2}{0.2}
Our matrix element was
%
\begin{equation}\label{eqn:qmTwoL10:10}
\begin{aligned}
H_{mi}'(t)
&= 2 A_mi \sin(\omega_0 t) \theta(t) \\
&= i A_{mi} ( e^{-i \omega_0 t} - e^{i \omega_0 t} ) \theta(t),
\end{aligned}
\end{equation}
%
and we found
%
\begin{equation}\label{eqn:qmTwoL10:30}
c_m^{(1)}(t) = \frac{A_{mi}}{\Hbar} \int_0^t dt' \left(
e^{i (\omega_{mi} - \omega_0) t'}
-
e^{i (\omega_{mi} + \omega_0) t'}
\right),
\end{equation}
%
and argued that
%
\begin{equation}\label{eqn:qmTwoL10:50}
\Abs{ c_m^{(1)}(t) }^2  \sim \left( \frac{A_{mi}}{\Hbar} \right)^2 t^2 + \cdots
\end{equation}
%
where \(\omega_0 t \gg 1\) for \(\omega_{mi} \sim \pm \omega_0\).

We can also just integrate \eqnref{eqn:qmTwoL10:30} directly
%
\begin{equation}\label{eqn:qmTwoL10:30b}
\begin{aligned}
c_m^{(1)}(t)
&=
\frac{A_{mi}}{\Hbar} \left(
\frac{e^{i (\omega_{mi} - \omega_0) t} - 1}
{ i (\omega_{mi} - \omega_0) }
-
\frac{e^{i (\omega_{mi} + \omega_0) t} - 1}{ i (\omega_{mi} + \omega_0) }
\right) \\
&\equiv
A_{mi}(\omega_0, t) - A_{mi}(-\omega_0, t),
\end{aligned}
\end{equation}
%
where
%
\begin{equation}\label{eqn:qmTwoL10:70}
A_{mi}(\omega_0, t) =
\frac{A_{mi}}{\Hbar}
\frac{e^{i (\omega_{mi} - \omega_0) t} - 1}
{ i (\omega_{mi} - \omega_0) }
\end{equation}
%
Factoring out the phase term, we have
%
\begin{equation}\label{eqn:qmTwoL10:90}
A_{mi}(\omega_0, t) =
\frac{2 A_{mi}}{\Hbar}
e^{i (\omega_{mi} - \omega_0) t/2}
\frac{
\sin(
(\omega_{mi} - \omega_0) t/2
)
}
{ (\omega_{mi} - \omega_0) }
\end{equation}
%
We we will have two lobes, centered on \(\pm \omega_0\), as illustrated in \cref{fig:qmTwoL10:qmTwoL10fig3}.
\imageFigure{../figures/phy456-qmII/qmTwoL10fig3}{Two sinc lobes.}{fig:qmTwoL10:qmTwoL10fig3}{0.2}
\section{Fermi's Golden rule.}
Fermi's Golden rule applies to a continuum of states (there are other forms of Fermi's golden rule, but this is the one we will talk about, and is the one in the book).  One example is the ionized states of an atom, where the energy level separation becomes so small that we can consider it continuous.
\imageFigure{../figures/phy456-qmII/continuumEnergyLevels}{Continuum of energy levels for ionized states of an atom.}{fig:qmTwoL10:continuumEnergyLevels}{0.2}
%\cref{fig:qmTwoL10:continuumEnergyLevels}.
Another example are the unbound states in a semiconductor well as illustrated in \cref{fig:qmTwoL10:semiConductorWell}.
\imageFigure{../figures/phy456-qmII/semiConductorWell}{Semi-conductor well.}{fig:qmTwoL10:semiConductorWell}{0.2}
Note that we can have reflection from the well even in the continuum states where we would have no such reflection classically.  However, with enough energy, states are approximately plane waves.  In one dimension
%
\begin{equation}\label{eqn:qmTwoL10:110}
\begin{aligned}
\braket{x}{\psi_p} &\approx \frac{e^{i p x/\Hbar}}{\sqrt{2 \pi \Hbar}} \\
\braket{\psi_p}{\psi_p'} &= \delta(p - p')
\end{aligned}
\end{equation}
%
or in 3d
%
\begin{equation}\label{eqn:qmTwoL10:130}
\begin{aligned}
\braket{\Br}{\psi_{\Bp}} &\approx \frac{e^{i \Bp \cdot \Br/\Hbar}}{(2 \pi \Hbar)^{3/2} } \\
\braket{\psi_{\Bp}}{\psi_{\Bp'}} &= \delta^3(\Bp - \Bp')
\end{aligned}
\end{equation}
%
Let us consider the 1d model for the quantum well in more detail.  Including both discrete and continuous states we have
%
\begin{equation}\label{eqn:qmTwoL10:150}
\ket{\psi(t)} =
\sum_n c_n(t) e^{-i \omega_n t} \ket{\psi_n} +
\int dp c_p(t) e^{-i \omega_p t} \ket{\psi_p}
\end{equation}
%
Imagine at \(t=0\) that the wave function started in some discrete state, and look at the probability that we ``kick the electron out of the well''.  Calculate
%
\begin{equation}\label{eqn:qmTwoL10:170}
\calP = \int dp \Abs{c_p^{(1)}(t)}^2
\end{equation}
%
Now, we assume that our matrix element has the following form
%
\begin{equation}\label{eqn:qmTwoL10:190}
H_{pi}'(t) = \left(
\overbar{A}_{pi} e^{-i \omega_0 t}
+\overbar{B}_{pi} e^{i \omega_0 t} \right) \theta(t)
\end{equation}
%
generalizing the wave train matrix element that we had previously
%
\begin{equation}\label{eqn:qmTwoL10:210}
H_{mi}'(t) = i A_{mi} \left(
e^{-i \omega_0 t}
- e^{i \omega_0 t} \right) \theta(t)
\end{equation}
%
Doing the perturbation we have
%
\begin{equation}\label{eqn:qmTwoL10:230}
\calP = \int dp \Abs{
A_{pi}(\omega_0, t)
+ B_{pi}(-\omega_0, t)
}^2
\end{equation}
%
where
%
\begin{equation}\label{eqn:qmTwoL10:250}
A_{pi}(\omega_0, t) =
\frac{2 \overbar{A}_{pi}}{i \Hbar }
e^{i (\omega_{pi} - \omega_0) t/2}
\frac{\sin((\omega_{pi} - \omega_0) t/2)}{
\omega_{pi} - \omega_0
}
\end{equation}
%
which is peaked at \(\omega_{pi} = \omega_0\), and
%
\begin{equation}\label{eqn:qmTwoL10:270}
B_{pi}(\omega_0, t) =
\frac{2 \overbar{B}_{pi}}{i \Hbar }
e^{i (\omega_{pi} + \omega_0) t/2}
\frac{\sin((\omega_{pi} + \omega_0) t/2)}{
\omega_{pi} + \omega_0
}
\end{equation}
%
which is peaked at \(\omega_{pi} = -\omega_0\).

FIXME: show that this is the perturbation result.

In \eqnref{eqn:qmTwoL10:230} at \(t \gg 0\) the only significant contribution is from the \(A\) portion as illustrated in \cref{fig:qmTwoL10:qmTwoL10fig6} where we are down in the wiggles of \(A_{pi}\).
\imageFigure{../figures/phy456-qmII/qmTwoL10fig6}{.}{fig:qmTwoL10:qmTwoL10fig6}{0.2}
Our probability to find the particle in the continuum range is now approximately
%
\begin{equation}\label{eqn:qmTwoL10:290}
\calP = \int dp \Abs{
A_{pi}(\omega_0, t)
}^2
\end{equation}
%
With
%
\begin{equation}\label{eqn:qmTwoL10:310}
\omega_{pi} - \omega_0 = \inv{\Hbar}\left( \frac{p^2}{2m} - E_i \right) - \omega_0,
\end{equation}
%
define \(\overbar{p}\) so that
%
\begin{equation}\label{eqn:qmTwoL10:330}
0 = \inv{\Hbar}\left( \frac{\overbar{p}^2}{2m} - E_i \right) - \omega_0.
\end{equation}
%
In momentum space, we know have the sinc functions peaked at \(\pm \overbar{p}\) as in \cref{fig:qmTwoL10:qmTwoL10fig7}.
\imageFigure{../figures/phy456-qmII/qmTwoL10fig7}{Momentum space view.}{fig:qmTwoL10:qmTwoL10fig7}{0.2}
The probability that the electron goes to the right is then
%
\begin{equation}\label{eqn:qmTwoL10:350}
\begin{aligned}
\calP_{+}
&=
\int_0^\infty dp
\Abs{
c_p^{(1)}(t)
}^2 \\
&=
\int_0^\infty dp
\Abs{
\overbar{A}_{pi}
}^2
\frac{
\sin^2\left( (\omega_{pi} - \omega_0) t/2 \right)
}{
\left( \omega_{pi} - \omega_0 \right)^2
},
\end{aligned}
\end{equation}
%
with
%
\begin{equation}\label{eqn:qmTwoL10:370}
\omega_{pi} = \inv{\Hbar}\left( \frac{p^2}{2m} - E_i
\right)
\end{equation}
%
we have with a change of variables
%
\begin{equation}\label{eqn:qmTwoL10:390}
\calP_{+}
=
\frac{4}{\Hbar^2}
\int_{-E_i/\Hbar}^\infty d\omega_{pi}
\Abs{
\overbar{A}_{pi}
}^2
\frac{dp}{d\omega_{pi}}
\frac{
\sin^2\left( (\omega_{pi} - \omega_0) t/2 \right)
}{
\left( \omega_{pi} - \omega_0 \right)^2
}.
\end{equation}
%
Now suppose we have \(t\) small enough so that \(\calP_{+} \ll 1\) and \(t\) large enough so
%
\begin{equation}\label{eqn:qmTwoL10:410}
\Abs{
\overbar{A}_{pi}
}^2
\frac{dp}{d\omega_{pi}}
\end{equation}
%
is roughly constant over \(\Delta \omega\).  This is a sort of ``Goldilocks condition'', a time that can not be too small, and can not be too large, but instead has to be ``just right''.  Given such a condition
%
\begin{equation}\label{eqn:qmTwoL10:430}
\calP_{+}
=
\frac{4}{\Hbar^2}
\Abs{
\overbar{A}_{pi}
}^2
\frac{dp}{d\omega_{pi}}
\int_{-E_i/\Hbar}^\infty d\omega_{pi}
\frac{
\sin^2\left( (\omega_{pi} - \omega_0) t/2 \right)
}{
\left( \omega_{pi} - \omega_0 \right)^2
},
\end{equation}
%
where we can pull stuff out of the integral since the main contribution is at the peak.  Provided \(\overbar{p}\) is large enough, using \eqnref{eqn:sincIntegral:50}, then
%
\begin{equation}\label{eqn:qmTwoL10:450}
\begin{aligned}
\int_{-E_i/\Hbar}^\infty d\omega_{pi}
\frac{
\sin^2\left( (\omega_{pi} - \omega_0) t/2 \right)
}{
\left( \omega_{pi} - \omega_0 \right)^2
}
&\approx
\int_{-\infty}^\infty d\omega_{pi}
\frac{
\sin^2\left( (\omega_{pi} - \omega_0) t/2 \right)
}{
\left( \omega_{pi} - \omega_0 \right)^2
} \\
&=
\frac{t}{2} \pi,
\end{aligned}
\end{equation}
%
leaving the probability of the electron with a going right continuum state as
%
\begin{equation}\label{eqn:qmTwoL10:470}
\calP_{+}
=
\frac{4}{\Hbar^2}
\mathLabelBox{
\Abs{
\overbar{A}_{pi}
}^2
}{matrix element}
\mathLabelBox
[
   labelstyle={below of=m\themathLableNode, below of=m\themathLableNode}
]
{\evalbar{\frac{dp}{d\omega_{pi}}}{\overbar{p}}}{density of states}
\frac{t}{2} \pi.
\end{equation}
%
The \(dp/d\omega_{pi}\) is something like ``how many continuous states are associated with a transition from a discrete frequency interval.''

We can also get this formally from \eqnref{eqn:qmTwoL10:430} with
%
\begin{equation}\label{eqn:qmTwoL10:490}
\frac{
\sin^2\left( (\omega_{pi} - \omega_0) t/2 \right)
}{
\left( \omega_{pi} - \omega_0 \right)^2
}
\rightarrow
\frac{t}{2} \pi \delta(\omega_{pi} - \omega_0),
\end{equation}
%
so
%
\begin{equation}\label{eqn:qmTwoL10:510}
\begin{aligned}
c_p^{(1)}(t)
&\rightarrow \frac{2 \pi t}{\Hbar^2}
\Abs{
\overbar{A}_{pi}
}^2
\delta(\omega_{pi} - \omega_0) \\
&=
\frac{2 \pi t}{\Hbar}
\Abs{
\overbar{A}_{pi}
}^2
\delta(E_{pi} - \Hbar \omega_0)
\end{aligned}
\end{equation}
%
where \(\delta(ax) = \delta(x)/\Abs{a}\) has been used to pull in a factor of \(\Hbar\) into the delta.

The ratio of the coefficient to time is then
%
\begin{equation}\label{eqn:qmTwoL10:530}
\frac{c_p^{(1)}(t) }{t}
=
\frac{2 \pi}{\Hbar}
\Abs{
\overbar{A}_{pi}
}^2
\delta(E_{pi} - \Hbar \omega_0).
\end{equation}
%
or ``between friends''
%
\begin{equation}\label{eqn:qmTwoL10:550}
''\frac{dc_p^{(1)}(t) }{dt}''
=
\frac{2 \pi}{\Hbar}
\Abs{
\overbar{A}_{pi}
}^2
\delta(E_{pi} - \Hbar \omega_0),
\end{equation}
%
roughly speaking we have a ``rate'' or transitions from the discrete into the continuous.  Here ``rate'' is in quotes since it does not hold for small t.

This has been worked out for \(\calP_{+}\).  This can also be done for \(\calP_{-}\), the probability that the electron will end up in a left trending continuum state.

While the above is not a formal derivation, but illustrates the form of what is called Fermi's golden rule.  Namely that such a rate has the structure
%
\begin{equation}\label{eqn:qmTwoL10:570}
\frac{2 \pi}{\Hbar} \times (\text{matrix element})^2 \times \text{energy conservation}
\end{equation}

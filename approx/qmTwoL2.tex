%
% Copyright � 2012 Peeter Joot.  All Rights Reserved.
% Licenced as described in the file LICENSE under the root directory of this GIT repository.
%
%
%
%
%\input{../peeter_prologue_print.tex}
%\input{../peeter_prologue_widescreen.tex}
%
%\chapter{PHY456H1F: Quantum Mechanics II.  Lecture 2 (Taught by Prof J.E. Sipe).  Approximate methods}
\index{approximate methods}
\label{chap:qmTwoL2}
%\blogpage{http://sites.google.com/site/peeterjoot/math2011/qmTwoL2.pdf}
%\date{Sept 14, 2011}
%
\section{Approximate methods for finding energy eigenvalues and eigenkets.}
\index{energy eigenvalue!approximate}
\index{energy eigenket!approximate}
In many situations one has a Hamiltonian \(H\)
%
\begin{equation}\label{eqn:qmTwoL2:10}
H \ket{\Psi_{n \alpha}} = E_n \ket{\Psi_{n \alpha}}.
\end{equation}
%
Here \(\alpha\) is a ``degeneracy index'' (example: as in Hydrogen atom).
%
\paragraph{Why?}
%
\begin{itemize}
\item Because this simplifies dynamics. Given an expansion of an initial state in some basis
%
\begin{equation}\label{eqn:qmTwoL2:730}
\begin{aligned}
\ket{\Psi(0)}
= \sum_{n\alpha}
\ket{\Psi_{n \alpha}}
\braket{\Psi_{n \alpha}}{\Psi(0)}
&
= \sum_{n\alpha} c_{n \alpha} \ket{\Psi_{n \alpha}},
\end{aligned}
\end{equation}
%
the time evolution of the state is
\begin{equation}\label{eqn:qmTwoL2:750}
\begin{aligned}
\ket{\Psi(t)}
&=
e^{-i H t/\Hbar}
\ket{\Psi(0)} \\
&=
\sum_{n\alpha} c_{n \alpha}
e^{-i H t/\Hbar}
\ket{\Psi_{n \alpha}}  \\
&=
\sum_{n\alpha} c_{n \alpha}
e^{-i E_n t/\Hbar}
\ket{\Psi_{n \alpha}}.
\end{aligned}
\end{equation}
%
\item An ``applied  field"', as depicted crudely in \cref{fig:qmTwoL2:1},
can often be thought of a driving the system from one eigenstate to another.
\imageFigure{../figures/phy456-qmII/qmTwoL2fig1}{Stimulated emission.}{fig:qmTwoL2:1}{0.2}
\item In Statistical mechanics, at thermal equilibrium, the expectation value of an observable \( \calO \) is
%
\begin{equation}\label{eqn:qmTwoL2:30}
\expectation{\calO} =
\frac{\sum_{n \alpha} \bra{\Psi_{n\alpha}} \calO \ket{\Psi_{n \alpha}}  e^{-\beta E_n}}{
Z
},
\end{equation}
%
where
%
\begin{equation}\label{eqn:qmTwoL2:50}
\beta = \inv{\kB T},
\end{equation}
%
and
%
\begin{equation}\label{eqn:qmTwoL2:70}
Z = \sum_{n \alpha} e^{-\beta E_n}.
\end{equation}
\end{itemize}
%
\section{Variational principle.}
\index{variational principle}

Consider any ket
%
\begin{equation}\label{eqn:qmTwoL2:90}
\ket{\Psi} = \sum_{n \alpha} c_{n \alpha} \ket{\Psi_{n \alpha}},
\end{equation}
%
perhaps not even normalized, where the coefficients
%
\begin{equation}\label{eqn:qmTwoL2:110}
c_{n \alpha} = \braket{\Psi_{n \alpha}}{\Psi},
\end{equation}
%
are unknown.
%
We do know that
\begin{equation}\label{eqn:qmTwoL2:130}
\braket{\Psi}{\Psi} = \sum_{n \alpha} \Abs{c_{n \alpha}}^2,
\end{equation}
%
and can relate the average energy to the ground state energy as follows
\begin{equation}\label{eqn:qmTwoL2:770}
\begin{aligned}
\frac{
\bra{\Psi} H \ket{\Psi}
}{
\braket{\Psi}{\Psi}
}
&=
\frac{
\sum_{n \alpha} \Abs{c_{n \alpha}}^2 E_n
}{
\sum_{m \beta} \Abs{c_{m \beta}}^2
} \\
&\ge
\frac{
\sum_{n \alpha} \Abs{c_{n \alpha}}^2 E_0
}{
\sum_{m \beta} \Abs{c_{m \beta}}^2
}  \\
&=
E_0.
\end{aligned}
\end{equation}
%
So for any ket we can form the upper bound for the ground state energy
%
\begin{equation}\label{eqn:qmTwoL2:150}
\frac{
\bra{\Psi} H \ket{\Psi}
}{
\braket{\Psi}{\Psi}
}
\ge E_0.
\end{equation}
%
There is a whole set of strategies based on estimating the ground state energy.  This is called the Variational principle for ground state.  See \S 24.2 in the text \citep{desai2009quantum}.

We define the functional
%
\begin{equation}\label{eqn:qmTwoL2:170}
E[\Psi] =
\frac{
\bra{\Psi} H \ket{\Psi}
}{
\braket{\Psi}{\Psi}
}
\ge E_0.
\end{equation}
%
In particular, if \(\ket{\Psi} = c \ket{\Psi_0}\) where \(\ket{\Psi_0}\) is the normalized ground state, then
%
\begin{equation}\label{eqn:qmTwoL2:190}
E[ c \Psi_0 ] = E_0.
\end{equation}
%
\makeexample{Hydrogen atom}{approx:ex1}{
%
The solution of the Hydrogen problem is given by
\begin{equation}\label{eqn:qmTwoL2:210}
\bra{\Br} H \ket{\Br'} = \calH \delta^3(\Br - \Br'),
\end{equation}
%
where
%
\begin{equation}\label{eqn:qmTwoL2:230}
\calH = -\frac{\Hbar^2}{2 \mu} \spacegrad^2 - \frac{e^2}{r},
\end{equation}
%
and \(\mu\) is the reduced mass.

We know the exact ground state energy
%
%\begin{equation}\label{eqn:qmTwoL2:250}
%H \ket{\Psi_0}
%\end{equation}
%
\begin{equation}\label{eqn:qmTwoL2:270}
E_0 = -R_y,
\end{equation}
%
where
\begin{equation}\label{eqn:qmTwoL2:290}
R_y = \frac{\mu e^4}{2 \Hbar^2} \approx 13.6 \text{eV}.
\end{equation}
%
The ground state wavefunction, plotted in \cref{fig:qmTwoL2:2}, is
\begin{equation}\label{eqn:qmTwoL2:310}
\braket{\Br}{\Psi_0} = \Phi_{100}(\Br) = \left( \inv{\pi a_0^3}\right)^{1/2} e^{-r/a_0},
\end{equation}
%
where
\begin{equation}\label{eqn:qmTwoL2:330}
a_0 = \frac{\Hbar^2}{\mu e^2} \approx 0.53 \angstrom.
\end{equation}
%
\imageFigure{../figures/phy456-qmII/hydrogenGroundStateFig2}{Hydrogen ground state magnitude.}{fig:qmTwoL2:2}{0.2}
%\imageFigure{../figures/phy456-qmII/qmTwoL2fig2}{qmTwoL2fig2.}{fig:qmTwoL2:2}{0.2}
%\imageFigure{../figures/phy456-qmII/qmTwoL2fig3}{qmTwoL2fig3.}{fig:qmTwoL2:3}{0.2}
We can estimate by evaluating
%
\begin{equation}\label{eqn:qmTwoL2:350}
\begin{aligned}
\bra{\Psi} H \ket{\Psi} &= \int d^3 \Br \Psi^\conj(\Br) \left( -\frac{\Hbar^2}{2 \mu} \spacegrad^2 - \frac{e^2}{r} \right) \Psi(\Br) \\
\braket{\Psi}{\Psi} &= \int d^3 \Br \Abs{\Psi(\Br)}^2,
\end{aligned}
\end{equation}
%
Using the trial wave function \(\Psi = e^{-\alpha r^2}\), 
plotted in \cref{fig:qmTwoL2:4}
%\imageFigure{../figures/phy456-qmII/qmTwoL2fig4}{qmTwoL2fig4.}{fig:qmTwoL2:4}{0.2}
\imageFigure{../figures/phy456-qmII/hydrogenGroundStateFig4}{Gaussian test function.}{fig:qmTwoL2:4}{0.2}
%
we want to calculate the ground state energy estimate associated with this \( \alpha \) parameter
\begin{equation}\label{eqn:qmTwoL2:370}
E[\Psi] \rightarrow E(\alpha).
\end{equation}
Explicitly, this is
\begin{equation}\label{eqn:qmTwoL2:390}
E(\alpha) =
\frac{\int d^3 \Br e^{-\alpha r^2} \left( -\frac{\Hbar^2}{2 \mu} \spacegrad^2 - \frac{e^2}{r} \right) e^{-\alpha r^2}}{
\int d^3\Br e^{-2 \alpha r^2}
}.
\end{equation}
%
It can apparently be shown that this results in
\begin{equation}\label{eqn:qmTwoL2:410}
E(\alpha) = A \alpha - B \alpha^{1/2},
\end{equation}
%
where
\begin{equation}\label{eqn:qmTwoL2:430}
\begin{aligned}
A &= \frac{3 \Hbar^2}{2\mu} \\
B &= 2 e^2 \left( \frac{2}{\pi} \right)^{1/2}.
\end{aligned}
\end{equation}
%
(I tried to calculate this, and get a different answer, so I've either taken bad notes, or calculated wrong, or what was on the board
was not right.)
\Cref{eqn:qmTwoL2:410} is of the form \cref{fig:qmTwoL2:5}, with the minimum energy occuring at
%\imageFigure{../figures/phy456-qmII/qmTwoL2fig5}{qmTwoL2fig5.}{fig:qmTwoL2:5}{0.2}
\imageFigure{../figures/phy456-qmII/hydrogenGroundStateFig5}{Ground state energy for the Gaussian test function.}{fig:qmTwoL2:5}{0.2}
%
\begin{equation}\label{eqn:qmTwoL2:450}
\alpha_0 =
\left( \frac{\mu e^2}{ \Hbar^2 } \right) \frac{8 }{9 \pi}.
\end{equation}
%
So
%
\begin{equation}\label{eqn:qmTwoL2:470}
E(\alpha_0) =
- \frac{\mu e^4 }{2 \Hbar^2} \frac{8 }{3 \pi} = -0.85 R_y
\end{equation}
%
maybe not too bad...
}
%
\makeexample{Helium atom}{approx:ex2}{
Assume an infinite nuclear mass with nucleus charge \(2 e\)
\imageFigure{../figures/phy456-qmII/qmTwoL2fig6}{qmTwoL2fig6.}{fig:qmTwoL2:6}{0.2}
%
The problem that we want to solve is
%
\begin{equation}\label{eqn:qmTwoL2:510}
\left(
-\frac{\Hbar^2}{2 m} \spacegrad_1^2
-\frac{\Hbar^2}{2 m} \spacegrad_2^2
- \frac{2 e}{r}
+
\frac{e^2}{\Abs{\Br_1 - \Br_2}}
\right)
\Psi_0(\Br_1, \Br_2) = E_0 \Psi_0(\Br_1, \Br_2),
\end{equation}
%
where \( \Psi_0(\Br_1, \Br_2) \) is the ground state wavefunction.
%\begin{equation}\label{eqn:qmTwoL2:490}
%\end{equation}
%
Nobody can solve this problem.  It is one of the simplest real problems in QM that cannot be solved exactly.
Suppose that we neglected the electron, electron repulsion.  Then
%
\begin{equation}\label{eqn:qmTwoL2:530}
\Psi_0(\Br_1, \Br_2)
=
\overbar{\Phi}_{100}(\Br_1)
\overbar{\Phi}_{100}(\Br_2),
\end{equation}
%
where
%
\begin{equation}\label{eqn:qmTwoL2:550}
\left( -\frac{\Hbar^2}{2 m} \spacegrad^2
- \frac{2 e}{r} \right)
\overbar{\Phi}_{100}(\Br) = \epsilon \overbar{\Phi}_{100}(\Br),
\end{equation}
%
and where
%
\begin{equation}\label{eqn:qmTwoL2:570}
\epsilon = - 4 R_y,
\end{equation}
%
and
\begin{equation}\label{eqn:qmTwoL2:590}
R_y = \frac{m e^4}{2 \Hbar^2}.
\end{equation}
%
This is the solution to
%
%\begin{equation}\label{eqn:qmTwoL2:610}
\begin{dmath}\label{eqn:qmTwoL2:610}
\left(
-\frac{\Hbar^2}{2 m} \spacegrad_1^2
-\frac{\Hbar^2}{2 m} \spacegrad_2^2
- \frac{2 e}{r}
\right)
\Psi_0^{(0)}(\Br_1, \Br_2) = E_0 \Psi_0(\Br_1, \Br_2)
=
E_0^{(0)} \Psi_0^{(0)}(\Br_1, \Br_2),
\end{dmath}
%\end{equation}
so
%
\begin{equation}\label{eqn:qmTwoL2:630}
E_0^{(0)} = - 8 R_y.
\end{equation}
%
Now we want to put back in the electron electron repulsion, and make an estimate.

With a trial wavefunction
%
\begin{equation}\label{eqn:qmTwoL2:650}
\Psi(\Br_1, \Br_2, Z) =
\left(
\left(\frac{Z^3}{ \pi a_0^3 }\right)^{1/2} e^{-Z r_1/a_0}
\right)
\left(
\left(\frac{Z^3}{ \pi a_0^3 }\right)^{1/2} e^{-Z r_2/a_0}
\right)
\end{equation}
%
expect that the best estimate is for \(Z \in [1,2]\).
This can be calculated numerically, and we find
%
\begin{equation}\label{eqn:qmTwoL2:670}
E(Z) = 2 R_Y \left( Z^2 - 4 Z + \frac{5}{8} Z \right).
\end{equation}
%
The \(Z^2\) comes from the kinetic energy.  The \(-4 Z\) is the electron nuclear attraction, and the final term is from the electron-electron repulsion.

The actual minimum is
%
\begin{equation}\label{eqn:qmTwoL2:690}
Z = 2 - \frac{5}{16},
\end{equation}
%
\begin{equation}\label{eqn:qmTwoL2:710}
E(2 - 5/16) = -77.5 \text{eV}.
\end{equation}
%
As expected, this is greater than the actual interaction free value of \(-78.6 \text{eV}\).
}
%\shipoutAnswer

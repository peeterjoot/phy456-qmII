%
% Copyright � 2012 Peeter Joot.  All Rights Reserved.
% Licenced as described in the file LICENSE under the root directory of this GIT repository.
%

%
%
%\input{../peeter_prologue_print.tex}
%\input{../peeter_prologue_widescreen.tex}

%\chapter{PHY456H1F: Quantum Mechanics II.  Lecture 4 (Taught by Prof J.E. Sipe).  Time independent perturbation theory (continued)}
\index{time independent perturbation}
\label{chap:qmTwoL4}
\blogpage{http://sites.google.com/site/peeterjoot/math2011/qmTwoL4.pdf}
%\date{Sept 21, 2011}


% subst: \Esn => {E_s}^{(n)}
% subst: \kpsi_s^n => \ket{{\psi_s}^{(n)}}
% subst: \kpsi_s => \ket{\psi_s}
% subst: \cns\d => {c_{ns}}^{(d)}
% subst: \cns => c_{ns}
%:,$ s/\\E\(.\)\(.\)/E_\1^{(\2)}/cg
%:,$ s/\\kpsi_\(.\)^\(.\)/\\ket{\\psi_\1^{(\2)}}/cg
%:,$ s/\\kpsi_\(.\)/\\ket{\\psi_\1}/cg
%:,$ s/\\cns\([0-9]\)/{c_{ns}}^{(\1)}/cg
%:,$ s/\\cns\>/c_{ns}/cg



\section{Time independent perturbation}
\paragraph{The setup}
%
To recap, we were covering the time independent perturbation methods from \S 16.1 of the text \citep{desai2009quantum}.  We start with a known Hamiltonian \(H_0\), and alter it with the addition of a ``small'' perturbation
%
\begin{equation}\label{eqn:qmTwoL4:10}
H = H_0 + \lambda H', \qquad \lambda \in [0,1]
\end{equation}
%
For the original operator, we assume that a complete set of eigenvectors and eigenkets is known
%
\begin{equation}\label{eqn:qmTwoL4:30}
H_0 \ket{{\psi_s}^{(0)}} = {E_s}^{(0)} \ket{{\psi_s}^{(0)}}
\end{equation}
%
We seek the perturbed eigensolution
%
\begin{equation}\label{eqn:qmTwoL4:50}
H \ket{\psi_s} = E_s \ket{\psi_s}
\end{equation}
%
and assumed a perturbative series representation for the energy eigenvalues in the new system
%
\begin{equation}\label{eqn:qmTwoL4:70}
E_s = {E_s}^{(0)} + \lambda {E_s}^{(1)} + \lambda^2 {E_s}^{(2)} + \cdots
\end{equation}
%
Given an assumed representation for the new eigenkets in terms of the known basis
%
\begin{equation}\label{eqn:qmTwoL4:90}
\ket{\psi_s} = \sum_n c_{ns} \ket{{\psi_n}^{(0)}}
\end{equation}
%
and a pertubative series representation for the probability coefficients
%
\begin{equation}\label{eqn:qmTwoL4:110}
c_{ns} = {c_{ns}}^{(0)} + \lambda {c_{ns}}^{(1)} + \lambda^2 {c_{ns}}^{(2)},
\end{equation}
%
so that
%
\begin{equation}\label{eqn:qmTwoL4:130}
\ket{\psi_s} =
\sum_n {c_{ns}}^{(0)} \ket{{\psi_n}^{(0)}}
+
\lambda
\sum_n {c_{ns}}^{(1)} \ket{{\psi_n}^{(0)}}
+
\lambda^2
\sum_n {c_{ns}}^{(2)} \ket{{\psi_n}^{(0)}}
+ \cdots
\end{equation}
%
Setting \(\lambda = 0\) requires
%
\begin{equation}\label{eqn:qmTwoL4:150}
{c_{ns}}^{(0)} = \delta_{ns},
\end{equation}
%
for
%
\begin{equation}\label{eqn:qmTwoL4:170}
\begin{aligned}
\ket{\psi_s}
&=
\ket{{\psi_s}^{(0)}}
+
\lambda
\sum_n {c_{ns}}^{(1)} \ket{{\psi_n}^{(0)}}
+
\lambda^2
\sum_n {c_{ns}}^{(2)} \ket{{\psi_n}^{(0)}}
+ \cdots \\
&=
\left(
1
+ \lambda {c_{ss}}^{(1)}
+ \lambda^2 {c_{ss}}^{(2)}
+ \cdots
\right)
\ket{{\psi_s}^{(0)}}
+
\lambda
\sum_{n \ne s} {c_{ns}}^{(1)} \ket{{\psi_n}^{(0)}}
+
\lambda^2
\sum_{n \ne s} {c_{ns}}^{(2)} \ket{{\psi_n}^{(0)}}
+ \cdots
\end{aligned}
\end{equation}
%
We rescaled our kets
%
\begin{equation}\label{eqn:qmTwoL4:190}
\ket{\overbar{\psi}_s}
=
\ket{{\psi_s}^{(0)}}
+
\lambda
\sum_{n \ne s} {\overbar{c}_{ns}}^{(1)} \ket{{\psi_n}^{(0)}}
+
\lambda^2
\sum_{n \ne s} {\overbar{c}_{ns}}^{(2)} \ket{{\psi_n}^{(0)}}
+ \cdots
\end{equation}
%
where
\begin{equation}\label{eqn:qmTwoL4:210}
{\overbar{c}_{ns}}^{(j)} =
\frac{{c_{ns}}^{(j)}}
{
1
+ \lambda {c_{ss}}^{(1)}
+ \lambda^2 {c_{ss}}^{(2)}
+ \cdots
}
\end{equation}
%
The normalization of the rescaled kets is then
%
\begin{equation}\label{eqn:qmTwoL4:230}
\braket{\overbar{\psi}_s}{\overbar{\psi}_s}
=
1
+
\lambda^2
\sum_{n \ne s} \Abs{{\overbar{c}_{ss}}^{(1)}}^2
+
\cdots
\equiv \inv{Z_s},
\end{equation}
%
One can then construct a renormalized ket if desired
%
\begin{equation}\label{eqn:qmTwoL4:250}
\ket{\overbar{\psi}_s}_R = Z_s^{1/2} \ket{\overbar{\psi}_s},
\end{equation}
%
so that
\begin{equation}\label{eqn:qmTwoL4:270}
(\ket{\overbar{\psi}_s}_R)^\dagger \ket{\overbar{\psi}_s}_R = Z_s \braket{\overbar{\psi}_s}{\overbar{\psi}_s} = 1.
\end{equation}
%
\paragraph{The meat}
%
That is as far as we got last time.  We continue by renaming terms in \eqnref{eqn:qmTwoL4:190}
%
\begin{equation}\label{eqn:qmTwoL4:300}
\ket{\overbar{\psi}_s}
=
\ket{{\psi_s}^{(0)}}
+
\lambda \ket{{\psi_s}^{(1)}}
+
\lambda^2 \ket{{\psi_s}^{(2)}}
+ \cdots
\end{equation}
%
where
%
\begin{equation}\label{eqn:qmTwoL4:320}
\ket{{\psi_s}^{(j)}} = \sum_{n \ne s} {\overbar{c}_{ns}}^{(j)} \ket{{\psi_n}^{(0)}}.
\end{equation}
%
Now we act on this with the Hamiltonian
%
\begin{equation}\label{eqn:qmTwoL4:340}
H \ket{\overbar{\psi}_s} = E_s \ket{\overbar{\psi}_s},
\end{equation}
%
or
%
\begin{equation}\label{eqn:qmTwoL4:360}
H \ket{\overbar{\psi}_s} - E_s \ket{\overbar{\psi}_s} = 0.
\end{equation}
%
Expanding this, we have
\begin{equation}\label{eqn:qmTwoL4:380}
\begin{aligned}
&(H_0 + \lambda H')
\left(
\ket{{\psi_s}^{(0)}}
+
\lambda \ket{{\psi_s}^{(1)}}
+
\lambda^2 \ket{{\psi_s}^{(2)}}
+ \cdots
\right) \\
&\quad -
\left( {E_s}^{(0)} + \lambda {E_s}^{(1)} + \lambda^2 {E_s}^{(2)} + \cdots \right)
\left(
\ket{{\psi_s}^{(0)}}
+
\lambda \ket{{\psi_s}^{(1)}}
+
\lambda^2 \ket{{\psi_s}^{(2)}}
+ \cdots
\right)
= 0.
\end{aligned}
\end{equation}
%
We want to write this as
%
\begin{equation}\label{eqn:qmTwoL4:400}
\ket{A} + \lambda \ket{B} + \lambda^2 \ket{C} + \cdots = 0.
\end{equation}
%
This is
%
\begin{equation}\label{eqn:qmTwoL4:420}
\begin{aligned}
0 &=
\lambda^0
(H_0 - E_s^{(0)}) \ket{{\psi_s}^{(0)}}  \\
&+ \lambda
\left(
(H_0 - E_s^{(0)}) \ket{{\psi_s}^{(1)}}
+(H' - E_s^{(1)}) \ket{{\psi_s}^{(0)}}
\right) \\
&+ \lambda^2
\left(
(H_0 - E_s^{(0)}) \ket{{\psi_s}^{(2)}}
+(H' - E_s^{(1)}) \ket{{\psi_s}^{(1)}}
-E_s^{(2)} \ket{{\psi_s}^{(0)}}
\right) \\
&\cdots
\end{aligned}
\end{equation}
%
So we form
%
\begin{equation}\label{eqn:qmTwoL4:440}
\begin{aligned}
\ket{A} &=
(H_0 - E_s^{(0)}) \ket{{\psi_s}^{(0)}} \\
\ket{B} &=
(H_0 - E_s^{(0)}) \ket{{\psi_s}^{(1)}}
+(H' - E_s^{(1)}) \ket{{\psi_s}^{(0)}} \\
\ket{C} &=
(H_0 - E_s^{(0)}) \ket{{\psi_s}^{(2)}}
+(H' - E_s^{(1)}) \ket{{\psi_s}^{(1)}}
-E_s^{(2)} \ket{{\psi_s}^{(0)}},
\end{aligned}
\end{equation}
%
and so forth.
%
\paragraph{Zeroth order in \(\lambda\)}
%
Since \(H_0 \ket{{\psi_s}^{(0)}} = E_s^{(0)} \ket{{\psi_s}^{(0)}}\), this first condition on \(\ket{A}\) is not much more than a statement that \(0 - 0 = 0\).
%
\paragraph{First order in \(\lambda\)}
%
How about \(\ket{B} = 0\)?  For this to be zero we require that both of the following are simultaneously zero
%
\begin{equation}\label{eqn:qmTwoL4:460}
\begin{aligned}
\braket{{\psi_s}^{(0)}}{B} &= 0 \\
\braket{{\psi_m}^{(0)}}{B} &= 0, \qquad m \ne s
\end{aligned}
\end{equation}
%
This first condition is
\begin{equation}\label{eqn:qmTwoL4:480}
\bra{{\psi_s}^{(0)}} (H' - E_s^{(1)}) \ket{{\psi_s}^{(0)}} = 0.
\end{equation}
%
With
\begin{equation}\label{eqn:qmTwoL4:500}
\bra{{\psi_m}^{(0)}} H' \ket{{\psi_s}^{(0)}} \equiv {H'}_{ms},
\end{equation}
%
or
\begin{equation}\label{eqn:qmTwoL4:520}
{H'}_{ss} = E_s^{(1)}.
\end{equation}
%
From the second condition we have
\begin{equation}\label{eqn:qmTwoL4:540}
0 = \bra{{\psi_m}^{(0)}}
(H_0 - E_s^{(0)}) \ket{{\psi_s}^{(1)}}
+\bra{{\psi_m}^{(0)}}
(H' - E_s^{(1)}) \ket{{\psi_s}^{(0)}}
\end{equation}
%
Utilizing the Hermitian nature of \(H_0\) we can act backwards on \(\bra{{\psi_m}^{(0)}}\)
%
\begin{equation}\label{eqn:qmTwoL4:560}
\bra{{\psi_m}^{(0)}} H_0
=
E_m^{(0)} \bra{{\psi_m}^{(0)}}.
\end{equation}
%
We note that \(\braket{{\psi_m}^{(0)}}{{\psi_s}^{(0)}} = 0, m \ne s\).  We can also expand the \(\braket{{\psi_m}^{(0)}}{{\psi_s}^{(1)}}\), which is
%
\begin{equation}\label{eqn:qmTwoL4:780}
\begin{aligned}
\braket{{\psi_m}^{(0)}}{{\psi_s}^{(1)}}
&=
\bra{{\psi_m}^{(0)}}
\left(
\sum_{n \ne s} {\overbar{c}_{ns}}^{(1)} \ket{{\psi_n}^{(0)}}
\right) \\
\end{aligned}
\end{equation}
%
I found that reducing this sum was not obvious until some actual integers were plugged in.  Suppose that \(s = 3\), and \(m = 5\), then this is
%
\begin{equation}\label{eqn:qmTwoL4:800}
\begin{aligned}
\braket{{\psi_5}^{(0)}}{{\psi_3}^{(1)}}
&=
\bra{{\psi_5}^{(0)}}
\left(
\sum_{n = 0, 1, 2, 4, 5, \cdots} {\overbar{c}_{n3}}^{(1)} \ket{{\psi_n}^{(0)}}
\right) \\
&=
{\overbar{c}_{53}}^{(1)} \braket{{\psi_5}^{(0)}}{{\psi_5}^{(0)}} \\
&=
{\overbar{c}_{53}}^{(1)}.
\end{aligned}
\end{equation}
%
Observe that we can also replace the superscript \((1)\) with \((j)\) in the above manipulation without impacting anything else.  That and putting back in the abstract indices, we have the general result
%
\begin{equation}\label{eqn:qmTwoL4:580}
\braket{{\psi_m}^{(0)}}{{\psi_s}^{(j)}}
=
{\overbar{c}_{ms}}^{(j)}.
\end{equation}
%
Utilizing this gives us
\begin{equation}\label{eqn:qmTwoL4:600}
0 =
( E_m^{(0)} - E_s^{(0)})
{\overbar{c}_{ms}}^{(1)}
+
{H'}_{ms}
\end{equation}
%
And summarizing what we learn from our \(\ket{B} = 0\) conditions we have
%
\begin{equation}\label{eqn:qmTwoL4:620}
\begin{aligned}
E_s^{(1)} &= {H'}_{ss} \\
{\overbar{c}_{ms}}^{(1)}
&=
\frac{{H'}_{ms} }
{ E_s^{(0)} - E_m^{(0)} }
\end{aligned}
\end{equation}
%
\paragraph{Second order in \(\lambda\)}
%
Doing the same thing for \(\ket{C} = 0\) we form (or assume)
%
\begin{equation}\label{eqn:qmTwoL4:640}
\braket{{\psi_s}^{(0)}}{C} = 0
\end{equation}
% not used:
%\braket{{\psi_m}^{(0)}}{C} &= 0, \qquad m \ne s
% not used:
%
\begin{equation}\label{eqn:qmTwoL4:820}
\begin{aligned}
0
&= \braket{{\psi_s}^{(0)}}{C}  \\
&=
\bra{{\psi_s}^{(0)}}
\left(
(H_0 - E_s^{(0)}) \ket{{\psi_s}^{(2)}}
+(H' - E_s^{(1)}) \ket{{\psi_s}^{(1)}}
-E_s^{(2)} \ket{{\psi_s}^{(0)}}
\right) \\
&=
(E_s^{(0)} - E_s^{(0)})
\braket{{\psi_s}^{(0)}}{{\psi_s}^{(2)}}
+
\bra{{\psi_s}^{(0)}}
(H' - E_s^{(1)}) \ket{{\psi_s}^{(1)}}
-E_s^{(2)} \braket{{\psi_s}^{(0)}}{{\psi_s}^{(0)}}
\end{aligned}
\end{equation}
%
We need to know what the \(\braket{{\psi_s}^{(0)}}{{\psi_s}^{(1)}}\) is, and find that it is zero
%
\begin{equation}\label{eqn:qmTwoL4:660}
\braket{{\psi_s}^{(0)}}{{\psi_s}^{(1)}}
=
\bra{{\psi_s}^{(0)}}
\sum_{n \ne s} {\overbar{c}_{ns}}^{(1)} \ket{{\psi_n}^{(0)}}
\end{equation}
%
Again, suppose that \(s = 3\).  Our sum ranges over all \(n \ne 3\), so all the brakets are zero.  Utilizing that we have
%
\begin{equation}\label{eqn:qmTwoL4:840}
\begin{aligned}
E_s^{(2)}
&=
\bra{{\psi_s}^{(0)}} H' \ket{{\psi_s}^{(1)}}  \\
&=
\bra{{\psi_s}^{(0)}} H' \sum_{m \ne s} {\overbar{c}_{ms}}^{(1)} \ket{{\psi_m}^{(0)}} \\
&=
\sum_{m \ne s} {\overbar{c}_{ms}}^{(1)} {H'}_{sm}
\end{aligned}
\end{equation}
%
From \eqnref{eqn:qmTwoL4:620} we have
%
\begin{equation}\label{eqn:qmTwoL4:700}
E_s^{(2)}
=
\sum_{m \ne s}
\frac{{H'}_{ms} }
{ E_s^{(0)} - E_m^{(0)} }
{H'}_{sm}
=
\sum_{m \ne s}
\frac{\Abs{{H'}_{ms}}^2 }
{ E_s^{(0)} - E_m^{(0)} }
\end{equation}
%
We can now summarize by forming the first order terms of the perturbed energy and the corresponding kets
%
\begin{equation}\label{eqn:qmTwoL4:720}
\begin{aligned}
E_s &= E_s^{(0)} + \lambda {H'}_{ss} + \lambda^2
\sum_{m \ne s}
\frac{\Abs{{H'}_{ms}}^2 }
{ E_s^{(0)} - E_m^{(0)} }
+ \cdots
\\
\ket{\overbar{\psi}_s} &= \ket{{\psi_s}^{(0)}} + \lambda
\sum_{m \ne s}
\frac{{H'}_{ms}}
{ E_s^{(0)} - E_m^{(0)} } \ket{{\psi_m}^{(0)}}
+ \cdots
\end{aligned}
\end{equation}
%
We can continue calculating, but are hopeful that we can stop the calculation without doing more work, even if \(\lambda = 1\).  If one supposes that the
%
\begin{equation}\label{eqn:qmTwoL4:740}
\sum_{m \ne s}
\frac{{H'}_{ms}}
{ E_s^{(0)} - E_m^{(0)} }
\end{equation}
%
term is ``small'', then we can hope that truncating the sum will be reasonable for \(\lambda = 1\).  This would be the case if
%
\begin{equation}\label{eqn:qmTwoL4:760}
{H'}_{ms} \ll \Abs{ E_s^{(0)} - E_m^{(0)} },
\end{equation}
%
however, to put some mathematical rigor into making a statement of such smallness takes a lot of work.  We are referred to \citep{messiah1999quantum}.  Incidentally, these are loosely referred to as the first and second testaments, because of the author's name, and the fact that they came as two volumes historically.



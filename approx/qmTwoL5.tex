%
% Copyright � 2012 Peeter Joot.  All Rights Reserved.
% Licenced as described in the file LICENSE under the root directory of this GIT repository.
%
%
%
%
%\input{../peeter_prologue_print.tex}
%\input{../peeter_prologue_widescreen.tex}
%
%\chapter{PHY456H1F: Quantum Mechanics II.  Lecture 5 (Taught by Prof J.E. Sipe).  Pertubation theory and degeneracy.  Review of dynamics}
\index{perturbation theory}
\index{degeneracy}
\index{dynamics}
\label{chap:qmTwoL5}
%\blogpage{http://sites.google.com/site/peeterjoot/math2011/qmTwoL5.pdf}
%\date{Sept 23, 2011}
%
\section{Issues concerning degeneracy.}
%
\paragraph{When the perturbed state is non-degenerate}
%
Suppose the state of interest is non-degenerate but others are

FIXME: diagram.  states designated by dashes labelled \(n1\), \(n2\), \(n3\) degeneracy \(\alpha = 3\) for energy \(E_n^{(0)}\).

This is no problem except for notation, and if the analysis is repeated we find
%
\begin{align}\label{eqn:qmTwoL5:10}
E_s &= E_s^{(0)} + \lambda {H'}_{ss} + \lambda^2
\sum_{m \ne s, \alpha}
\frac{\Abs{{H'}_{m \alpha ; s}}^2 }
{ E_s^{(0)} - E_m^{(0)} }
+ \cdots
\\
\ket{\overbar{\psi}_s} &= \ket{{\psi_s}^{(0)}} + \lambda
\sum_{m \ne s, \alpha}
\frac{{H'}_{m \alpha ; s}}
{ E_s^{(0)} - E_m^{(0)} } \ket{{\psi_{m \alpha}}^{(0)}}
+ \cdots,
\end{align}
%
where
%
\begin{equation}\label{eqn:qmTwoL5:30}
{H'}_{m \alpha ; s} =
\bra{{\psi_{m \alpha}}^{(0)}} H' \ket{{\psi_{s \alpha}}^{(0)}}.
\end{equation}
%
\paragraph{When the perturbed state is also degenerate}
%
FIXME: diagram.  states designated by dashes labelled \(n1\), \(n2\), \(n3\) degeneracy \(\alpha = 3\) for energy \(E_n^{(0)}\), and states designated by dashes labelled \(s1\), \(s2\), \(s3\) degeneracy \(\alpha = 3\) for energy \(E_s^{(0)}\).

If we just blindly repeat the derivation for the non-degenerate case we would obtain
%
\begin{align}\label{eqn:qmTwoL5:50}
E_s &= E_s^{(0)} + \lambda {H'}_{s1 ; s1}
+ \lambda^2
\sum_{m \ne s, \alpha}
\frac{\Abs{{H'}_{m \alpha ; s1}}^2 }
{ E_s^{(0)} - E_m^{(0)} } \\
&\quad + \lambda^2
\sum_{\alpha \ne 1}
\frac{\Abs{{H'}_{s \alpha ; s1}}^2 }
{ E_s^{(0)} - \color{red}{E_s^{(0)}} }
+ \cdots
\\
\ket{\overbar{\psi}_s} &= \ket{{\psi_s}^{(0)}}
+ \lambda
\sum_{m \ne s, \alpha}
\frac{{H'}_{m \alpha ; s}}
{ E_s^{(0)} - E_m^{(0)} } \ket{{\psi_{m \alpha}}^{(0)}} \\
&\quad + \lambda
\sum_{\alpha \ne s1}
\frac{{H'}_{s \alpha ; s1}}
{ E_s^{(0)} - \color{red}{E_s^{(0)}} } \ket{{\psi_{s \alpha}}^{(0)}}
+ \cdots,
\end{align}
%
where
%
\begin{equation}\label{eqn:qmTwoL5:70}
{H'}_{m \alpha ; s1} =
\bra{{\psi_{m \alpha}}^{(0)}} H' \ket{{\psi_{s1}}^{(0)}}.
\end{equation}
%
Note that the \(E_s^{(0)} -\color{red}{E_s^{(0)}}\) is NOT a typo, and why we run into trouble.  There is one case where a perturbation approach is still possible.  That case is if we happen to have
%
\begin{equation}\label{eqn:qmTwoL5:90}
\bra{{\psi_{m \alpha}}^{(0)}} H' \ket{{\psi_{s1}}^{(0)}} = 0.
\end{equation}
%
That may not be obvious, but if one returns to the original derivation, the right terms cancel so that one will not end up with the \(0/0\) problem.

FIXME: performing this derivation outside of class (below), it was found that we do not need the matrix elements of \(H'\) to be diagonal, but just need
%
\begin{equation}\label{eqn:qmTwoL5:90b}
\bra{{\psi_{s \alpha}}^{(0)}} H' \ket{{\psi_{s \beta}}^{(0)}} = 0, \quad \mbox{for \(\beta \ne \alpha\)}.
\end{equation}
%
That is consistent with problem set III where we did not diagonalize \(H'\), but just the subset of it associated with the degenerate states.  I am unsure now if \eqnref{eqn:qmTwoL5:90} was copied in error or provided in error in class, but it definitely appears to be a more severe requirement than actually needed to deal with perturbation of a state found in a degenerate energy level.

%
% Copyright � 2012 Peeter Joot.  All Rights Reserved.
% Licenced as described in the file LICENSE under the root directory of this GIT repository.
%
%
%
%
%\input{../peeter_prologue_print.tex}
%%\input{../peeter_prologue_widescreen.tex}
%
%%\chapter{PHY456H1F: Quantum Mechanics II.  Lecture 4 (Taught by Prof J.E. Sipe).  Time independent perturbation theory (continued)}
\index{time independent perturbation}
%\chapter{Time independent perturbation theory with degeneracy}
%\label{chap:qmTwoL5a}
%\blogpage{http://sites.google.com/site/peeterjoot/math2011/qmTwoL5.pdf}
%\date{Sept 30, 2011}
%
%
%
%
\subsubsection{Time independent perturbation with degeneracy.}
%
Now we repeat the derivation of the first order perturbation with degenerate states from lecture 4.  We see explicitly how we would get into (divide by zero) trouble if the state we were perturbing had degeneracy.  Here I alter the previous derivation to show this explicitly.
%\subsubsection{The setup}
Like the non-degenerate case, we are covering the time independent perturbation methods from \S 16.1 of the text \citep{desai2009quantum}.

We start with a known Hamiltonian \(H_0\), and alter it with the addition of a ``small'' perturbation
%
\begin{equation}\label{eqn:qmTwoL5a:10}
H = H_0 + \lambda H', \qquad \lambda \in [0,1]
\end{equation}
%
For the original operator, we assume that a complete set of eigenvectors and eigenkets is known
%
\begin{equation}\label{eqn:qmTwoL5a:30}
H_0 \ket{{\psi_{s \alpha}}^{(0)}} = {E_s}^{(0)} \ket{{\psi_{s \alpha}}^{(0)}}
\end{equation}
%
We seek the perturbed eigensolution
%
\begin{equation}\label{eqn:qmTwoL5a:50}
H \ket{\psi_{s \alpha}} = E_{s \alpha} \ket{\psi_{s \alpha}}
\end{equation}
%
and assumed a perturbative series representation for the energy eigenvalues in the new system
%
\begin{equation}\label{eqn:qmTwoL5a:70}
E_{s \alpha} = {E_s}^{(0)} + \lambda {E_{s \alpha}}^{(1)} + \lambda^2 {E_{s \alpha}}^{(2)} + \cdots
\end{equation}
%
Note that we do not assume that the perturbed energy states, if degenerate in the original system, are still degenerate after perturbation.

Given an assumed representation for the new eigenkets in terms of the known basis
%
\begin{equation}\label{eqn:qmTwoL5a:90}
\ket{\psi_{s \alpha}} = \sum_{n, \beta} c_{ns;\beta \alpha} \ket{{\psi_{n \beta}}^{(0)}}
\end{equation}
%
and a pertubative series representation for the probability coefficients
%
\begin{equation}\label{eqn:qmTwoL5a:110}
c_{ns;\beta \alpha} = {c_{ns;\beta \alpha}}^{(0)} + \lambda {c_{ns;\beta \alpha}}^{(1)} + \lambda^2 {c_{ns;\beta \alpha}}^{(2)},
\end{equation}
%
so that
%
\begin{dmath}\label{eqn:qmTwoL5a:130}
\ket{\psi_{s \alpha}} =
\sum_{n, \beta} {c_{ns;\beta \alpha}}^{(0)} \ket{{\psi_{n \beta}}^{(0)}}
+
\lambda
\sum_{n, \beta} {c_{ns;\beta \alpha}}^{(1)} \ket{{\psi_{n \beta}}^{(0)}}
+
\lambda^2
\sum_{n, \beta} {c_{ns;\beta \alpha}}^{(2)} \ket{{\psi_{n \beta}}^{(0)}}
+ \cdots
\end{dmath}
%
Setting \(\lambda = 0\) requires
%
\begin{equation}\label{eqn:qmTwoL5a:150}
{c_{ns;\beta \alpha}}^{(0)} = \delta_{ns;\beta \alpha},
\end{equation}
%
for
%
\begin{equation}\label{eqn:qmTwoL5a:170}
\begin{aligned}
\ket{\psi_{s \alpha}}
&=
\ket{{\psi_{s \alpha}}^{(0)}}
+
\lambda
\sum_{n, \beta} {c_{ns;\beta \alpha}}^{(1)} \ket{{\psi_{n \beta}}^{(0)}}
+
\lambda^2
\sum_{n, \beta} {c_{ns;\beta \alpha}}^{(2)} \ket{{\psi_{n \beta}}^{(0)}}
+ \cdots \\
&=
\left(
1
+ \lambda {c_{ss ; \alpha \alpha}}^{(1)}
+ \lambda^2 {c_{ss ; \alpha \alpha}}^{(2)}
+ \cdots
\right)
\ket{{\psi_{s \alpha}}^{(0)}}  \\
&\quad +
\lambda
\sum_{n\beta \ne s\alpha} {c_{ns;\beta \alpha}}^{(1)} \ket{{\psi_{n \beta}}^{(0)}}  \\
&\quad +
\lambda^2
\sum_{n\beta \ne s\alpha} {c_{ns;\beta \alpha}}^{(2)} \ket{{\psi_{n \beta}}^{(0)}}
+ \cdots
\end{aligned}
\end{equation}
%
We rescale our kets
%
\begin{equation}\label{eqn:qmTwoL5a:190}
\ket{\overbar{\psi}_{s \alpha}}
=
\ket{{\psi_{s \alpha}}^{(0)}}
+
\lambda
\sum_{n\beta \ne s\alpha} {\overbar{c}_{ns;\beta \alpha}}^{(1)} \ket{{\psi_{n \beta}}^{(0)}}
+
\lambda^2
\sum_{n\beta \ne s\alpha} {\overbar{c}_{ns;\beta \alpha}}^{(2)} \ket{{\psi_{n \beta}}^{(0)}}
+ \cdots
\end{equation}
%
where
\begin{equation}\label{eqn:qmTwoL5a:210}
{\overbar{c}_{ns;\beta \alpha}}^{(j)} =
\frac{{c_{ns;\beta \alpha}}^{(j)}}
{
1
+ \lambda {c_{ss ; \alpha \alpha}}^{(1)}
+ \lambda^2 {c_{ss ; \alpha \alpha}}^{(2)}
+ \cdots
}
\end{equation}
%
The normalization of the rescaled kets is then
%
\begin{equation}\label{eqn:qmTwoL5a:230}
\braket{\overbar{\psi}_{s \alpha}}{\overbar{\psi}_{s \alpha}}
=
1
+
\lambda^2
\sum_{n\beta \ne s\alpha} \Abs{{\overbar{c}_{ss}}^{(1)}}^2
+
\cdots
\equiv \inv{Z_{s \alpha}},
\end{equation}
%
One can then construct a renormalized ket if desired
%
\begin{equation}\label{eqn:qmTwoL5a:250}
\ket{\overbar{\psi}_{s \alpha}}_R = Z_{s \alpha}^{1/2} \ket{\overbar{\psi}_{s \alpha}},
\end{equation}
%
so that
\begin{equation}\label{eqn:qmTwoL5a:270}
(\ket{\overbar{\psi}_{s \alpha}}_R)^\dagger \ket{\overbar{\psi}_{s \alpha}}_R = Z_{s \alpha} \braket{\overbar{\psi}_{s \alpha}}{\overbar{\psi}_{s \alpha}} = 1.
\end{equation}
%
%\subsubsection{The meat}
%
We continue by renaming terms in \eqnref{eqn:qmTwoL5a:190}
%
\begin{equation}\label{eqn:qmTwoL5a:300}
\ket{\overbar{\psi}_{s \alpha}}
=
\ket{{\psi_{s \alpha}}^{(0)}}
+
\lambda \ket{{\psi_{s \alpha}}^{(1)}}
+
\lambda^2 \ket{{\psi_{s \alpha}}^{(2)}}
+ \cdots
\end{equation}
%
where
%
\begin{equation}\label{eqn:qmTwoL5a:320}
\ket{{\psi_{s \alpha}}^{(j)}} = \sum_{n\beta \ne s\alpha} {\overbar{c}_{ns;\beta \alpha}}^{(j)} \ket{{\psi_{n \beta}}^{(0)}}.
\end{equation}
%
Now we act on this with the Hamiltonian
%
\begin{equation}\label{eqn:qmTwoL5a:340}
H \ket{\overbar{\psi}_{s \alpha}} = E_{s \alpha} \ket{\overbar{\psi}_{s \alpha}},
\end{equation}
%
or
%
\begin{equation}\label{eqn:qmTwoL5a:360}
H \ket{\overbar{\psi}_{s \alpha}} - E_{s \alpha} \ket{\overbar{\psi}_{s \alpha}} = 0.
\end{equation}
%
Expanding this, we have
\begin{equation}\label{eqn:qmTwoL5a:380}
\begin{aligned}
0
&= (H_0 + \lambda H')
\lr{
   \ket{{\psi_{s \alpha}}^{(0)}}
   +
   \lambda \ket{{\psi_{s \alpha}}^{(1)}}
   +
   \lambda^2 \ket{{\psi_{s \alpha}}^{(2)}}
   + \cdots
}
\\
&\quad -
\lr{
   {E_s}^{(0)} + \lambda {E_{s \alpha}}^{(1)} + \lambda^2 {E_{s \alpha}}^{(2)} + \cdots
}
\lr{
   \ket{{\psi_{s \alpha}}^{(0)}}
   +
   \lambda \ket{{\psi_{s \alpha}}^{(1)}}
   +
   \lambda^2 \ket{{\psi_{s \alpha}}^{(2)}}
   + \cdots
}
.
\end{aligned}
\end{equation}
%
We want to write this as
%
\begin{equation}\label{eqn:qmTwoL5a:400}
\ket{A} + \lambda \ket{B} + \lambda^2 \ket{C} + \cdots = 0.
\end{equation}
%
This is
%
\begin{equation}\label{eqn:qmTwoL5a:420}
\begin{aligned}
0 &=
\lambda^0
(H_0 - E_s^{(0)}) \ket{{\psi_{s \alpha}}^{(0)}}  \\
&+ \lambda
\left(
(H_0 - E_s^{(0)}) \ket{{\psi_{s \alpha}}^{(1)}}
+(H' - E_{s \alpha}^{(1)}) \ket{{\psi_{s \alpha}}^{(0)}}
\right) \\
&+ \lambda^2
\left(
(H_0 - E_s^{(0)}) \ket{{\psi_{s \alpha}}^{(2)}}
+(H' - E_{s \alpha}^{(1)}) \ket{{\psi_{s \alpha}}^{(1)}}
-E_{s \alpha}^{(2)} \ket{{\psi_{s \alpha}}^{(0)}}
\right) \\
&\cdots
\end{aligned}
\end{equation}
%
So we form
%
\begin{equation}\label{eqn:qmTwoL5a:440}
\begin{aligned}
\ket{A} &=
(H_0 - E_s^{(0)}) \ket{{\psi_{s \alpha}}^{(0)}} \\
\ket{B} &=
(H_0 - E_s^{(0)}) \ket{{\psi_{s \alpha}}^{(1)}}
+(H' - E_{s \alpha}^{(1)}) \ket{{\psi_{s \alpha}}^{(0)}} \\
\ket{C} &=
(H_0 - E_s^{(0)}) \ket{{\psi_{s \alpha}}^{(2)}}
+(H' - E_{s \alpha}^{(1)}) \ket{{\psi_{s \alpha}}^{(1)}}
-E_{s \alpha}^{(2)} \ket{{\psi_{s \alpha}}^{(0)}},
\end{aligned}
\end{equation}
%
and so forth.
%
\paragraph{Zeroth order in \(\lambda\)}
%
Since \(H_0 \ket{{\psi_{s \alpha}}^{(0)}} = E_s^{(0)} \ket{{\psi_{s \alpha}}^{(0)}}\), this first condition on \(\ket{A}\) is not much more than a statement that \(0 - 0 = 0\).
%
\paragraph{First order in \(\lambda\)}
%
How about \(\ket{B} = 0\)?  For this to be zero we require that both of the following are simultaneously zero
%
\begin{equation}\label{eqn:qmTwoL5a:460}
\begin{aligned}
\braket{{\psi_{s \alpha}}^{(0)}}{B} &= 0 \\
\braket{{\psi_{m \beta}}^{(0)}}{B} &= 0, \qquad m \beta \ne s \alpha
\end{aligned}
\end{equation}
%
This first condition is
\begin{equation}\label{eqn:qmTwoL5a:480}
\bra{{\psi_{s \alpha}}^{(0)}} (H' - E_{s \alpha}^{(1)}) \ket{{\psi_{s \alpha}}^{(0)}} = 0.
\end{equation}
%
With
\begin{equation}\label{eqn:qmTwoL5a:500}
\bra{{\psi_{m \beta}}^{(0)}} H' \ket{{\psi_{s \alpha}}^{(0)}} \equiv {H'}_{ms ; \beta \alpha},
\end{equation}
%
or
\begin{equation}\label{eqn:qmTwoL5a:520}
{H'}_{ss ; \alpha \alpha} = E_{s \alpha}^{(1)}.
\end{equation}
%
From the second condition we have
\begin{equation}\label{eqn:qmTwoL5a:540}
0 = \bra{{\psi_{m \beta}}^{(0)}}
(H_0 - E_s^{(0)}) \ket{{\psi_{s \alpha}}^{(1)}}
+\bra{{\psi_{m \beta}}^{(0)}}
(H' - E_{s \alpha}^{(1)}) \ket{{\psi_{s \alpha}}^{(0)}}
\end{equation}
%
Utilizing the Hermitian nature of \(H_0\) we can act backwards on \(\bra{{\psi_m}^{(0)}}\)
%
\begin{equation}\label{eqn:qmTwoL5a:560}
\bra{{\psi_{m \beta}}^{(0)}} H_0
=
E_m^{(0)} \bra{{\psi_{m \beta}}^{(0)}}.
\end{equation}
%
We note that \(\braket{{\psi_{m \beta}}^{(0)}}{{\psi_{s \alpha}}^{(0)}} = 0, m \beta \ne s \alpha\).  We can also expand the \(\braket{{\psi_{m \beta}}^{(0)}}{{\psi_{s \alpha}}^{(1)}}\), which is
%
\begin{equation}\label{eqn:qmTwoL5a:740}
\begin{aligned}
\braket{{\psi_{m \beta}}^{(0)}}{{\psi_{s \alpha}}^{(1)}}
&=
\bra{{\psi_{m \beta}}^{(0)}}
\left(
\sum_{n\delta \ne s\alpha} {\overbar{c}_{ns;\delta \alpha}}^{(1)} \ket{{\psi_{n \delta}}^{(0)}}
\right) \\
\end{aligned}
\end{equation}
%
I found that reducing this sum was not obvious until some actual integers were plugged in.  Suppose that \(s = 3\,1\), and \(m \beta = 2\,2\), then this is
%
\begin{equation}\label{eqn:qmTwoL5a:760}
\begin{aligned}
\braket{{\psi_{2\,2}}^{(0)}}{{\psi_{3\,1}}^{(1)}}
&=
\bra{{\psi_{2\,2}}^{(0)}}
\left(
\sum_{n \delta \in \{1\,1, 1\,2, \cdots, 2\,1, 2\,2, 2\,3, \cdots, 3\,2, 3\,3, \cdots \} } {\overbar{c}_{n 3; \delta 1}}^{(1)} \ket{{\psi_{n \delta}}^{(0)}}
\right) \\
&=
{\overbar{c}_{2\,3 ; 2\,1}}^{(1)} \braket{{\psi_{2\,2}}^{(0)}}{{\psi_{2\,2}}^{(0)}} \\
&=
{\overbar{c}_{2\,3; 2\,1}}^{(1)}.
\end{aligned}
\end{equation}
%
Observe that we can also replace the superscript \((1)\) with \((j)\) in the above manipulation without impacting anything else.  That and putting back in the abstract indices, we have the general result
%
\begin{equation}\label{eqn:qmTwoL5a:580}
\braket{{\psi_{m \beta}}^{(0)}}{{\psi_{s \alpha}}^{(j)}}
=
{\overbar{c}_{ms ; \beta \alpha}}^{(j)}.
\end{equation}
%
Utilizing this gives us, for \(\color{red}{m \beta \ne s \alpha}\)
%
\begin{equation}\label{eqn:qmTwoL5a:600}
0 =
( E_m^{(0)} - E_s^{(0)})
{\overbar{c}_{ms ; \beta \alpha}}^{(1)}
+
{H'}_{ms ; \beta \alpha}
\end{equation}
%
Here we see our first sign of the trouble hinted at in lecture 5.  Just because \(m \beta \ne s \alpha\) does not mean that \(m \ne s\).  For example, with \(m \beta = 1\,1\) and \(s\alpha = 1\,2\) we would have
%
\begin{equation}\label{eqn:qmTwoL5a:620a}
\begin{aligned}
E_{1 2}^{(1)} &= {H'}_{1\,1 ; 2 2} \\
{\overbar{c}_{1\,1 ; 1 2}}^{(1)}
&=
\frac{{H'}_{1\,1 ; 1 2} }
{ E_1^{(0)} - E_1^{(0)} }
\end{aligned}
\end{equation}
%
We have got a \(\color{red}{\text{divide by zero}}\) unless additional restrictions are imposed!

If we return to \eqnref{eqn:qmTwoL5a:600}, we see that, for the result to be valid, when \(m = s\), and there exists degeneracy for the \(s\) state, we require for \(\beta \ne \alpha\)
%
\begin{equation}\label{eqn:qmTwoL5a:610}
{H'}_{ss ; \beta \alpha} = 0
\end{equation}
%
(then \eqnref{eqn:qmTwoL5a:600} becomes a \(0 = 0\) equality, and all is still okay)

And summarizing what we learn from our \(\ket{B} = 0\) conditions we have
%
\begin{equation}\label{eqn:qmTwoL5a:620}
\begin{aligned}
E_{s \alpha}^{(1)} &= {H'}_{ss ; \alpha \alpha} \\
{\overbar{c}_{ms ; \beta \alpha}}^{(1)}
&=
\frac{{H'}_{ms ; \beta \alpha} }
{ E_s^{(0)} - E_m^{(0)} }, \qquad {m \ne s} \\
{H'}_{ss ; \beta \alpha} &= 0, \qquad \beta \alpha \ne 1\,1
\end{aligned}
\end{equation}
%
\paragraph{Second order in \(\lambda\)}
%
Doing the same thing for \(\ket{C} = 0\) we form (or assume)
%
\begin{equation}\label{eqn:qmTwoL5a:640}
\braket{{\psi_{s \alpha}}^{(0)}}{C} = 0
\end{equation}
% not used:
%\braket{{\psi_{m \beta}}^{(0)}}{C} &= 0, \qquad m \ne s
% not used:
%
\begin{equation}\label{eqn:qmTwoL5a:780}
\begin{aligned}
0
&= \braket{{\psi_{s \alpha}}^{(0)}}{C}  \\
&=
\bra{{\psi_{s \alpha}}^{(0)}}
\left(
(H_0 - E_s^{(0)}) \ket{{\psi_{s \alpha}}^{(2)}}
+(H' - E_{s \alpha}^{(1)}) \ket{{\psi_{s \alpha}}^{(1)}}
-E_{s \alpha}^{(2)} \ket{{\psi_{s \alpha}}^{(0)}}
\right) \\
&=
(E_s^{(0)} - E_s^{(0)})
\braket{{\psi_{s \alpha}}^{(0)}}{{\psi_{s \alpha}}^{(2)}}
+
\bra{{\psi_{s \alpha}}^{(0)}}
(H' - E_{s \alpha}^{(1)}) \ket{{\psi_{s \alpha}}^{(1)}}
-E_{s \alpha}^{(2)} \braket{{\psi_{s \alpha}}^{(0)}}{{\psi_{s \alpha}}^{(0)}}
\end{aligned}
\end{equation}
%
We need to know what the \(\braket{{\psi_{s \alpha}}^{(0)}}{{\psi_{s \alpha}}^{(1)}}\) is, and find that it is zero
%
\begin{equation}\label{eqn:qmTwoL5a:660}
\braket{{\psi_{s \alpha}}^{(0)}}{{\psi_{s \alpha}}^{(1)}}
=
\bra{{\psi_{s \alpha}}^{(0)}}
\sum_{n\beta \ne s\alpha} {\overbar{c}_{ns;\beta \alpha}}^{(1)} \ket{{\psi_{n \beta}}^{(0)}} = 0
\end{equation}
%
%Again, suppose that \(s = 3\).  Our sum ranges over all \(n \ne 3\), so all the brakets are zero.
Utilizing that we have
%
\begin{equation}\label{eqn:qmTwoL5a:800}
\begin{aligned}
E_{s \alpha}^{(2)}
&=
\bra{{\psi_{s \alpha}}^{(0)}} H' \ket{{\psi_{s \alpha}}^{(1)}}  \\
&=
\bra{{\psi_{s \alpha}}^{(0)}} H' \sum_{m \beta \ne s \alpha} {\overbar{c}_{ms}}^{(1)} \ket{{\psi_{m \beta}}^{(0)}} \\
&=
\sum_{m \beta \ne s \alpha} {\overbar{c}_{ms ; \beta \alpha}}^{(1)} {H'}_{sm ; \alpha \beta}
\end{aligned}
\end{equation}
%
From \eqnref{eqn:qmTwoL5a:620}, treating the \(\color{red}{m \ne s}\) case carefully, we have
%
\begin{equation}\label{eqn:qmTwoL5a:700}
E_{s \alpha}^{(2)}
=
\sum_{\beta \ne \alpha} {\overbar{c}_{ss ; \beta \alpha}}^{(1)} {H'}_{ss ; \alpha \beta}
+
\sum_{m \beta \ne s \alpha, m \ne s}
\frac{{H'}_{ms ; \beta \alpha} }
{ E_s^{(0)} - E_m^{(0)} }
{H'}_{sm ; \alpha \beta}
\end{equation}
%
Again, only if \(H_{ss ; \alpha \beta} = 0\) for \(\beta \ne \alpha\) do we have a result we can use.  If that is the case, the first sum is killed without a divide by zero, leaving
%
\begin{equation}\label{eqn:qmTwoL5a:700b}
E_{s \alpha}^{(2)}
=
\sum_{m \beta \ne s \alpha, m \ne s}
\frac{\Abs{{H'}_{ms ; \beta \alpha}}^2 }
{ E_s^{(0)} - E_m^{(0)} }.
\end{equation}
%
We can now summarize by forming the first order terms of the perturbed energy and the corresponding kets
%
\begin{equation}\label{eqn:qmTwoL5a:720}
\begin{aligned}
E_{s \alpha} &= E_s^{(0)} + \lambda {H'}_{ss ; \alpha \alpha} + \lambda^2
\sum_{m \ne s, m \beta \ne s \alpha}
\frac{\Abs{{H'}_{ms ; \beta \alpha}}^2 }
{ E_s^{(0)} - E_m^{(0)} }
+ \cdots
\\
\ket{\overbar{\psi}_{s \alpha}} &= \ket{{\psi_{s \alpha}}^{(0)}} + \lambda
\sum_{m \ne s, m \beta \ne s \alpha}
\frac{{H'}_{ms ; \beta \alpha}}
{ E_s^{(0)} - E_m^{(0)} } \ket{{\psi_{m \beta}}^{(0)}}
+ \cdots \\
{H'}_{ss ; \beta \alpha} &= 0, \qquad \beta \alpha \ne 1\,1
\end{aligned}
\end{equation}
%
\paragraph{Notational discrepancy:} OOPS.  It looks like I used different notation than in class for our matrix elements for the placement of the indices.
%
FIXME: looks like the \({c_{ss ; \alpha \alpha'}}^{(1)}\), for \(\alpha \ne \alpha'\) coeffients have been lost track of here?  Do we have to assume those are zero too?  Professor Sipe did not include those in his lecture \eqnref{eqn:qmTwoL5:190}, but I do not see the motivation here for dropping them in this derivation.



%
\paragraph{Diagonalizing the perturbation Hamiltonian}
%
Suppose that we do not have this special zero condition that allows the perturbation treatment to remain valid.  What can we do.  It turns out that we can make use of the fact that the perturbation Hamiltonian is Hermitian, and diagonalize the matrix
%
\begin{equation}\label{eqn:qmTwoL5:110}
\bra{{\psi_{s \alpha}}^{(0)}} H' \ket{{\psi_{s \beta}}^{(0)}}.
\end{equation}
%
In the example of a two fold degeneracy, this amounts to us choosing not to work with the states
%
\begin{equation}\label{eqn:qmTwoL5:130}
\ket{\psi_{s1}^{(0)}}, \ket{\psi_{s2}^{(0)}},
\end{equation}
%
both some linear combinations of the two
\begin{align}\label{eqn:qmTwoL5:150}
\ket{\psi_{sI}^{(0)}} &= a_1 \ket{\psi_{s1}^{(0)}} + b_1 \ket{\psi_{s2}^{(0)}} \\
\ket{\psi_{sII}^{(0)}} &= a_2 \ket{\psi_{s1}^{(0)}} + b_2 \ket{\psi_{s2}^{(0)}}.
\end{align}
%
In this new basis, once found, we have
%
\begin{equation}\label{eqn:qmTwoL5:170}
\bra{{\psi_{s \alpha}}^{(0)}} H' \ket{{\psi_{s \beta}}^{(0)}} = \calH_\alpha \delta_{\alpha \beta}.
\end{equation}
%
Utilizing this to fix the previous, one would get if the analysis was repeated correctly
%
\begin{align}\label{eqn:qmTwoL5:190}
E_{s\alpha} &= E_s^{(0)} + \lambda {H'}_{s\alpha ; s\alpha}
+ \lambda^2
\sum_{m \ne s, \beta}
\frac{\Abs{{H'}_{m \beta ; s \alpha}}^2 }
{ E_s^{(0)} - E_m^{(0)} }
+ \cdots
\\
\ket{\overbar{\psi}_{s \alpha}} &= \ket{{\psi_{s \alpha}}^{(0)}}
+ \lambda
\sum_{m \ne s, \beta}
\frac{{H'}_{m \beta ; s \alpha}}
{ E_s^{(0)} - E_m^{(0)} } \ket{{\psi_{m \beta}}^{(0)}}
+ \cdots
\end{align}
%
FIXME: why do we have second order in \(\lambda\) terms for the energy when we found those exactly by diagonalization?  We found there that the perturbed energy eigenvalues were multivalued with values \(E_{s\alpha} = E_s^{(0)} + \lambda {H'}_{s\beta ; s\beta}\) for all degeneracy indices \(\beta\).  Have to repeat the derivation for these more carefully to understand this apparent discrepancy.

We see that a degenerate state can be split by applying perturbation.

FIXME: diagram.  \(E_s^{(0)}\) as one energy level without perturbation, and as two distinct levels with perturbation.
%
\paragraph{guess} I will bet that this is the origin of the spectral line splitting, especially given that an atom like hydrogen has degenerate states.
%

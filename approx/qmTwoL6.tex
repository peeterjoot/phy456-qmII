%
% Copyright � 2012 Peeter Joot.  All Rights Reserved.
% Licenced as described in the file LICENSE under the root directory of this GIT repository.
%
%
%
%
%\input{../peeter_prologue_print.tex}
%\input{../peeter_prologue_widescreen.tex}
%
%\chapter{PHY456H1F: Quantum Mechanics II.  Lecture 6 (Taught by Prof J.E. Sipe).  Interaction picture}
%\chapter{Interaction picture}
\index{interaction picture}
\label{chap:qmTwoL6}
%\blogpage{http://sites.google.com/site/peeterjoot/math2011/qmTwoL6.pdf}
%\date{Sept 26, 2011}
%
%
\section{Interaction picture.}
\paragraph{Recap.}
%
Recall our table comparing our two interaction pictures
%
\begin{equation}\label{eqn:qmTwoL6:970}
\begin{aligned}
\text{Schr\"{o}dinger picture} &\qquad \text{Heisenberg picture} \\
i \Hbar \frac{d}{dt} \ket{\psi_s(t)} = H \ket{\psi_s(t)} &\qquad i \Hbar \frac{d}{dt} O_H(t) = \antisymmetric{O_H}{H} \\
\bra{\psi_s(t)} O_S \ket{\psi_s(t)} &= \bra{\psi_H} O_H \ket{\psi_H} \\
\ket{\psi_s(0)} &= \ket{\psi_H} \\
O_S &= O_H(0)
\end{aligned}
\end{equation}
%
\paragraph{A motivating example.}
%
While fundamental Hamiltonians are independent of time, in a number of common cases, we can form approximate Hamiltonians that are time dependent.  One such example is that of Coulomb excitations of an atom, as covered in \S 18.3 of the text \citep{desai2009quantum}, and shown in \cref{fig:qmTwoL6:1}.
\imageFigure{../figures/phy456-qmII/qmTwoL6fig1}{Coulomb interaction of a nucleus and heavy atom.}{fig:qmTwoL6:1}{0.2}
%
We consider the interaction of a nucleus with a neutral atom, heavy enough that it can be considered classically.  From the atoms point of view, the effects of the heavy nucleus barrelling by can be described using a time dependent Hamiltonian.  For the atom, that interaction Hamiltonian is
%
\begin{equation}\label{eqn:qmTwoL6:10}
H' = \sum_i \frac{ Z e q_i }{\Abs{\Br_N(t) - \BR_i}}.
\end{equation}
%
Here and \(\Br_N\) is the position vector for the heavy nucleus, and \(\BR_i\) is the position to each charge within the atom, where \(i\) ranges over all the internal charges, positive and negative, within the atom.

Placing the origin close to the atom, we can write this interaction Hamiltonian as
%
\begin{equation}\label{eqn:qmTwoL6:30}
H'(t) = \cancel{\sum_i \frac{Z e q_i}{\Abs{\Br_N(t)}}}
+ \sum_i Z e q_i \BR_i \cdot
\evalbar{\left(
\PD{\Br}{} \inv{\Abs{ \Br_N(t) - \Br}}
\right)}{\Br = 0}
\end{equation}
%
The first term vanishes because the total charge in our neutral atom is zero.  This leaves us with
%
\begin{equation}\label{eqn:qmTwoL6:50}
\begin{aligned}
H'(t)
&=
-\sum_i q_i \BR_i \cdot \evalbar{\left(
-\PD{\Br}{} \frac{ Z e}{\Abs{ \Br_N(t) - \Br}}
\right)}{\Br = 0} \\
&= - \sum_i q_i \BR_i \cdot \BE(t),
\end{aligned}
\end{equation}
%
where \(\BE(t)\) is the electric field at the origin due to the nucleus.
Introducing a dipole moment \textunderline{operator} for the atom
%
\begin{equation}\label{eqn:qmTwoL6:70}
\Bmu = \sum_i q_i \BR_i,
\end{equation}
%
the interaction takes the form
%
\begin{equation}\label{eqn:qmTwoL6:90}
H'(t) = -\Bmu \cdot \BE(t).
\end{equation}
%
Here we have a quantum mechanical operator, and a classical field taken together.  This sort of dipole interaction also occurs when we treat a atom placed into an electromagnetic field, treated classically as depicted in \cref{fig:qmTwoL6:2}.
%
\imageFigure{../figures/phy456-qmII/qmTwoL6fig2}{atom in a field.}{fig:qmTwoL6:2}{0.2}
In the figure, we can use the dipole interaction, provided \(\lambda \gg a\), where \(a\) is the ``width'' of the atom.
%
Because it is great for examples, we will see this dipole interaction a lot.
\paragraph{The interaction picture.}
%
Having talked about both the Schr\"{o}dinger and Heisenberg pictures, we can now move on to describe a hybrid, one where our Hamiltonian has been split into static and time dependent parts
%
\begin{equation}\label{eqn:qmTwoL6:110}
H(t) = H_0 + H'(t).
\end{equation}
%
We will formulate an approach for dealing with problems of this sort called the interaction picture.

This is also covered in \S 3.3 of the text, albeit in a much harder to understand fashion (the text appears to try to not pull the result from a magic hat, but the steps to get to the end result are messy).  It would probably have been nicer to see it this way instead.

In the Schr\"{o}dinger picture our dynamics have the form
%
\begin{equation}\label{eqn:qmTwoL6:130}
i \Hbar \frac{d}{dt} \ket{\psi_s(t)} = H \ket{\psi_s(t)}.
\end{equation}
%
How about the Heisenberg picture?  We look for a solution
%
\begin{equation}\label{eqn:qmTwoL6:400}
\ket{\psi_s(t)} = U(t, t_0) \ket{\psi_s(t_0)}.
\end{equation}
%
We want to find this operator that evolves the state from the state as some initial time \(t_0\), to the arbitrary later state found at time \(t\).  Plugging in we have
%
\begin{equation}\label{eqn:qmTwoL6:420}
i \Hbar \ddt{} U(t, t_0) \ket{\psi_s(t_0)}
=
H(t) U(t, t_0) \ket{\psi_s(t_0)}.
\end{equation}
%
This has to hold for all \(\ket{\psi_s(t_0)}\), and we can equivalently seek a solution of the operator equation
%
\begin{equation}\label{eqn:qmTwoL6:440}
i \Hbar \ddt{} U(t, t_0) = H(t) U(t, t_0),
\end{equation}
%
where
%
\begin{equation}\label{eqn:qmTwoL6:460}
U(t_0, t_0) = I,
\end{equation}
%
the identity for the Hilbert space.
%
Suppose that \(H(t)\) was independent of time.  We could find that
%
\begin{equation}\label{eqn:qmTwoL6:480}
U(t, t_0) = e^{-i H(t - t_0)/\Hbar}.
\end{equation}
%
If \(H(t)\) depends on time could you guess that
%
\begin{equation}\label{eqn:qmTwoL6:500}
U(t, t_0) = e^{-\frac{i}{\Hbar} \int_{t_0}^t H(\tau) d\tau}
\end{equation}
%
holds?  No.  This may be true when \(H(t)\) is a number, but when it is an operator, the Hamiltonian does not necessarily commute with itself at different times
%
\begin{equation}\label{eqn:qmTwoL6:520}
\antisymmetric{H(t')}{H(t'')} \ne 0.
\end{equation}
%
So this is \textunderline{wrong} in general.  As an aside, for numbers, \eqnref{eqn:qmTwoL6:500} can be verified easily.  We have
%
\begin{equation}\label{eqn:qmTwoL6:990}
\begin{aligned}
i \Hbar \left( e^{-\frac{i}{\Hbar} \int_{t_0}^t H(\tau) d\tau} \right)'
&=
i \Hbar \left( -\frac{i}{\Hbar} \right) \left( \int_{t_0}^t H(\tau) d\tau \right)'
e^{-\frac{i}{\Hbar} \int_{t_0}^t H(\tau) d\tau } \\
&=
\left( H(t) \frac{dt}{dt} - H(t_0) \frac{dt_0}{dt} \right)
e^{-\frac{i}{\Hbar} \int_{t_0}^t H(\tau) d\tau}  \\
&=
H(t) U(t, t_0).
\end{aligned}
\end{equation}
%
\paragraph{Expectations.}
%
Suppose that we do find \(U(t, t_0)\).  Then our expectation takes the form
%
\begin{equation}\label{eqn:qmTwoL6:600}
\bra{\psi_s(t)} O_s \ket{\psi_s(t)}
=
\bra{\psi_s(t_0)} U^\dagger(t, t_0) O_s U(t, t_0) \ket{\psi_s(t_0)}.
\end{equation}
%
Put
%
\begin{equation}\label{eqn:qmTwoL6:620}
\ket{\psi_H} = \ket{\psi_s(t_0)},
\end{equation}
%
and form
%
\begin{equation}\label{eqn:qmTwoL6:640}
O_H = U^\dagger(t, t_0) O_s U(t, t_0),
\end{equation}
%
so that our expectation has the familiar representations
%
\begin{equation}\label{eqn:qmTwoL6:660}
\bra{\psi_s(t)} O_s \ket{\psi_s(t)}
=
\bra{\psi_H} O_H \ket{\psi_H}.
\end{equation}
%
\paragraph{New strategy.  Interaction picture.}
%
Let us define
%
\begin{equation}\label{eqn:qmTwoL6:680}
U_I(t, t_0) = e^{\frac{i}{\Hbar} H_0(t - t_0)} U(t, t_0),
\end{equation}
%
or
\begin{equation}\label{eqn:qmTwoL6:700}
U(t, t_0) = e^{-\frac{i}{\Hbar} H_0(t - t_0)} U_I(t, t_0).
\end{equation}
%
Let us see how this works.  We have
%
\begin{equation}\label{eqn:qmTwoL6:1010}
\begin{aligned}
i \Hbar \ddt{U_I}
&=
i \Hbar \ddt{} \left(
e^{\frac{i}{\Hbar} H_0(t - t_0)} U(t, t_0)
\right) \\
&=
-H_0 U(t, t_0)
+
e^{\frac{i}{\Hbar} H_0(t - t_0)} \left( i \Hbar \ddt{} U(t, t_0) \right) \\
&=
-H_0 U(t, t_0)
+
e^{\frac{i}{\Hbar} H_0(t - t_0)} \left( (H + H'(t)) U(t, t_0) \right) \\
&=
e^{\frac{i}{\Hbar} H_0(t - t_0)} H'(t) U(t, t_0) \\
&=
e^{\frac{i}{\Hbar} H_0(t - t_0)} H'(t) e^{-\frac{i}{\Hbar} H_0(t - t_0)} U_I(t, t_0).
\end{aligned}
\end{equation}
%
Define
%
\begin{equation}\label{eqn:qmTwoL6:720}
\overbar{H}'(t) =
e^{\frac{i}{\Hbar} H_0(t - t_0)} H'(t) e^{-\frac{i}{\Hbar} H_0(t - t_0)},
\end{equation}
%
so that our operator equation takes the form
%
\begin{equation}\label{eqn:qmTwoL6:740}
i \Hbar \ddt{} U_I(t, t_0) = \overbar{H}'(t) U_I(t, t_0).
\end{equation}
%
Note that we also have the required identity at the initial time
%
\begin{equation}\label{eqn:qmTwoL6:760}
U_I(t_0, t_0) = I.
\end{equation}
%
Without requiring us to actually find \(U(t, t_0)\) all of the dynamics of the time dependent interaction are now embedded in our operator equation for \(\overbar{H}'(t)\), with all of the simple interaction related to the non time dependent portions of the Hamiltonian left separate.
%
\paragraph{Connection with the Schr\"{o}dinger picture.}
%
In the Schr\"{o}dinger picture we have
%
\begin{equation}\label{eqn:qmTwoL6:1030}
\begin{aligned}
\ket{\psi_s(t)}
&= U(t, t_0)
\ket{\psi_s(t_0)}  \\
&=
e^{-\frac{i}{\Hbar} H_0(t - t_0)} U_I(t, t_0)
\ket{\psi_s(t_0)}.
\end{aligned}
\end{equation}
%
With a definition of the interaction picture ket as
%
\begin{equation}\label{eqn:qmTwoL6:780}
\ket{\psi_I}
= U_I(t, t_0) \ket{\psi_s(t_0)} = U_I(t, t_0) \ket{\psi_H},
\end{equation}
%
the Schr\"{o}dinger picture is then related to the interaction picture by
%
\begin{equation}\label{eqn:qmTwoL6:800}
\ket{\psi_s(t)} = e^{-\frac{i}{\Hbar} H_0(t - t_0)} \ket{\psi_I}.
\end{equation}
%
Also, by multiplying \eqnref{eqn:qmTwoL6:740} by our Schr\"{o}dinger ket, we remove the last vestiges of \(U_I\) and \(U\) from the dynamical equation for our time dependent interaction
%
\begin{equation}\label{eqn:qmTwoL6:820}
i \Hbar \ddt{}
\ket{\psi_I}
= \overbar{H}'(t)
\ket{\psi_I}.
\end{equation}
%
\paragraph{Interaction picture expectation.}
%
Inverting \eqnref{eqn:qmTwoL6:800}, we can form an operator expectation, and relate it the interaction and Schr\"{o}dinger pictures
%
\begin{equation}\label{eqn:qmTwoL6:840}
\bra{\psi_s(t)} O_s \ket{\psi_s(t)} =
\bra{\psi_I}
e^{\frac{i}{\Hbar} H_0(t - t_0)}
O_s
e^{-\frac{i}{\Hbar} H_0(t - t_0)}
\ket{\psi_I} .
\end{equation}
%
With a definition
%
\begin{equation}\label{eqn:qmTwoL6:860}
O_I =
e^{\frac{i}{\Hbar} H_0(t - t_0)}
O_s
e^{-\frac{i}{\Hbar} H_0(t - t_0)},
\end{equation}
%
we have
\begin{equation}\label{eqn:qmTwoL6:880}
\bra{\psi_s(t)} O_s \ket{\psi_s(t)} =
\bra{\psi_I}
O_I
\ket{\psi_I}.
\end{equation}
%
As before, the time evolution of our interaction picture operator, can be found by taking derivatives of \eqnref{eqn:qmTwoL6:860}, for which we find
%
\begin{equation}\label{eqn:qmTwoL6:900}
i \Hbar \ddt{O_I(t)} = \antisymmetric{O_I(t)}{H_0}.
\end{equation}
%
\paragraph{Summarizing the interaction picture.}
%
Given
%
\begin{equation}\label{eqn:qmTwoL6:910}
H(t) = H_0 + H'(t),
\end{equation}
%
and initial time states
\begin{equation}\label{eqn:qmTwoL6:950}
\ket{\psi_I(t_0)} =
\ket{\psi_s(t_0)} = \ket{\psi_H},
\end{equation}
%
we have
\begin{equation}\label{eqn:qmTwoL6:880b}
\bra{\psi_s(t)} O_s \ket{\psi_s(t)} =
\bra{\psi_I}
O_I
\ket{\psi_I},
\end{equation}
%
where
%
\begin{equation}\label{eqn:qmTwoL6:920}
\ket{\psi_I}
= U_I(t, t_0) \ket{\psi_s(t_0)},
\end{equation}
%
and
%
\begin{equation}\label{eqn:qmTwoL6:820b}
i \Hbar \ddt{}
\ket{\psi_I}
= \overbar{H}'(t)
\ket{\psi_I},
\end{equation}
%
or
%
\begin{equation}\label{eqn:qmTwoL6:740b}
\begin{aligned}
i \Hbar \ddt{} U_I(t, t_0) &= \overbar{H}'(t) U_I(t, t_0) \\
U_I(t_0, t_0) &= I.
\end{aligned}
\end{equation}
%
Our interaction picture Hamiltonian is
%
\begin{equation}\label{eqn:qmTwoL6:720b}
\overbar{H}'(t) =
e^{\frac{i}{\Hbar} H_0(t - t_0)} H'(t) e^{-\frac{i}{\Hbar} H_0(t - t_0)},
\end{equation}
%
and for Schr\"{o}dinger operators, independent of time, we have the dynamical equation
%
\begin{equation}\label{eqn:qmTwoL6:900b}
i \Hbar \ddt{O_I(t)} = \antisymmetric{O_I(t)}{H_0}.
\end{equation}
%
\section{Justifying the Taylor expansion above (not class notes).}
%
\paragraph{Multivariable Taylor series.}
\index{Taylor series!multivariable}

As outlined in \S 2.8 (\(8.10\)) of \citep{hestenes1999nfc}, we want to derive the multi-variable Taylor expansion for a scalar valued function of some number of variables
%
\begin{equation}\label{eqn:qmTwoL6:210}
f(\Bu) = f(u^1, u^2, \cdots),
\end{equation}
%
consider the displacement operation applied to the vector argument
%
\begin{equation}\label{eqn:qmTwoL6:230}
f(\Ba + \Bx) = \evalbar{f(\Ba + t \Bx)}{t=1}.
\end{equation}
%
We can Taylor expand a single variable function without any trouble, so introduce
%
\begin{equation}\label{eqn:qmTwoL6:250}
g(t) = f(\Ba + t \Bx),
\end{equation}
%
where
%
\begin{equation}\label{eqn:qmTwoL6:270}
g(1) = f(\Ba + \Bx).
\end{equation}
%
We have
%
\begin{equation}\label{eqn:qmTwoL6:290}
g(t) = g(0)
+ t \evalbar{ \PD{t}{g} }{t = 0}
+ \frac{t^2}{2!} \evalbar{ \PD{t}{g} }{t = 0}
+ \cdots,
\end{equation}
%
so that
%
\begin{equation}\label{eqn:qmTwoL6:310}
g(1) = g(0) +
+ \evalbar{ \PD{t}{g} }{t = 0}
+ \frac{1}{2!} \evalbar{ \PD{t}{g} }{t = 0}
+ \cdots.
\end{equation}
%
The multivariable Taylor series now becomes a plain old application of the chain rule, where we have to evaluate
%
\begin{equation}\label{eqn:qmTwoL6:1050}
\begin{aligned}
\frac{dg}{dt}
&= \ddt{} f(a^1 + t x^1, a^2 + t x^2, \cdots) \\
&= \sum_i \PD{(a^i + t x^i)}{} f(\Ba + t \Bx) \PD{t}{a^i + t x^i},
\end{aligned}
\end{equation}
%
so that
%
\begin{equation}\label{eqn:qmTwoL6:330}
\evalbar{\frac{dg}{dt} }{t=0}
= \sum_i x^i \left(
\evalbar{ \PD{x^i}{f}}{x^i = a^i}
\right).
\end{equation}
%
Assuming an Euclidean space we can write this in the notationally more pleasant fashion using a gradient operator for the space
%
\begin{equation}\label{eqn:qmTwoL6:350}
\evalbar{\frac{dg}{dt} }{t=0} = \evalbar{\Bx \cdot \spacegrad_{\Bu} f(\Bu)}{\Bu = \Ba}.
\end{equation}
%
To handle the higher order terms, we repeat the chain rule application, yielding for example
%
\begin{equation}\label{eqn:qmTwoL6:1070}
\begin{aligned}
\evalbar{\frac{d^2 f(\Ba + t \Bx)}{dt^2} }{t=0}
&=
\evalbar{\ddt{}
\sum_i x^i
\PD{(a^i + t x^i)}{f(\Ba + t \Bx)} }{t=0}\\
&=
\evalbar{\sum_i x^i
\PD{(a^i + t x^i)}{} \ddt{f(\Ba + t \Bx)}}{t=0} \\
&=
\evalbar{(\Bx \cdot \spacegrad_{\Bu})^2 f(\Bu)}{\Bu = \Ba}.
\end{aligned}
\end{equation}
%
Thus the Taylor series associated with a vector displacement takes the tidy form
%
\begin{equation}\label{eqn:qmTwoL6:370}
f(\Ba + \Bx) = \sum_{k=0}^\infty \inv{k!} \evalbar{(\Bx \cdot \spacegrad_{\Bu})^k f(\Bu)}{\Bu = \Ba}.
\end{equation}
%
Even more fancy, we can form the operator equation
%
\begin{equation}\label{eqn:qmTwoL6:390}
f(\Ba + \Bx) = \evalbar{e^{ \Bx \cdot \spacegrad_{\Bu} } f(\Bu)}{\Bu = \Ba}.
\end{equation}
%
Here a dummy variable \(\Bu\) has been retained as an instruction not to differentiate the \(\Bx\) part of the directional derivative in any repeated applications of the \(\Bx \cdot \spacegrad\) operator.

That notational kludge can be removed by swapping \(\Ba\) and \(\Bx\)
%
\begin{equation}\label{eqn:qmTwoL6:390b}
f(\Ba + \Bx) =
%\sum_{k=0}^\infty \inv{k!} \evalbar{(\Ba \cdot \spacegrad_{\Bu})^k f(\Bu)}{\Bu = \Bx}
\sum_{k=0}^\infty \inv{k!} (\Ba \cdot \spacegrad)^k f(\Bx)
=
e^{ \Ba \cdot \spacegrad } f(\Bx),
\end{equation}
%
where \(\spacegrad = \spacegrad_{\Bx} = (\PDi{x^1}{}, \PDi{x^2}{}, ...)\).

Having derived this (or for those with lesser degrees of amnesia, recall it), we can see that \eqnref{eqn:qmTwoL6:30} was a direct application of this, retaining no second order or higher terms.

Our expression used in the interaction Hamiltonian discussion was
%
\begin{equation}\label{eqn:qmTwoL6:170}
\inv{\Abs{\Br - \BR}} \approx \inv{\Abs{\Br}}
+
\BR \cdot \evalbar{\left(
\PD{\BR}{} \inv{\Abs{ \Br - \BR}}
\right)}{\BR = 0}.
\end{equation}
%
which we can see has the same structure as above with some variable substitutions.  Evaluating it we have
%
\begin{equation}\label{eqn:qmTwoL6:1090}
\begin{aligned}
\PD{\BR}{} \inv{\Abs{ \Br - \BR}}
&=
\Be_i \PD{R^i}{} ((x^j - R^j)^2)^{-1/2} \\
&=
\Be_i \left(-\inv{2}\right) 2 (x^j - R^j) \PD{R^i}{(x^j - R^j)} \inv{\Abs{\Br - \BR}^3} \\
&= \frac{\Br - \BR}{
\Abs{\Br - \BR}^3} ,
\end{aligned}
\end{equation}
%
and at \(\BR = 0\) we have
%
\begin{equation}\label{eqn:qmTwoL6:190}
\inv{\Abs{\Br - \BR}} \approx \inv{\Abs{\Br}}
+
\BR \cdot
\frac{\Br}{\Abs{\Br}^3}.
\end{equation}
%
We see in this direction derivative produces the classical electric Coulomb field expression for an electrostatic distribution, once we take the \(\Br/\Abs{\Br}^3\) and multiply it with the \(- Z e\) factor.
%
\paragraph{With algebra.}
%
A different way to justify the expansion of \eqnref{eqn:qmTwoL6:30} is to consider a Clifford algebra factorization (following notation from \citep{doran2003gap}) of the absolute vector difference, where \(\BR\) is considered small.
%
\begin{equation}\label{eqn:qmTwoL6:1110}
\begin{aligned}
\Abs{\Br - \BR}
&= \sqrt{ \left(\Br - \BR\right) \left(\Br - \BR\right) } \\
&= \sqrt{ \gpgradezero{\Br \left(1 - \inv{\Br} \BR\right) \left(1 - \BR \inv{\Br}\right) \Br} } \\
&= \sqrt{ \gpgradezero{\Br^2 \left(1 - \inv{\Br} \BR\right) \left(1 - \BR \inv{\Br}\right) } } \\
&= \Abs{\Br} \sqrt{ 1 - 2 \inv{\Br} \cdot \BR + \gpgradezero{\inv{\Br} \BR \BR \inv{\Br}}} \\
&= \Abs{\Br} \sqrt{ 1 - 2 \inv{\Br} \cdot \BR + \frac{\BR^2}{\Br^2}}
\end{aligned}
\end{equation}
%
Neglecting the \(\BR^2\) term, we can then Taylor series expand this scalar expression
%
\begin{equation}\label{eqn:qmTwoL6:150}
\inv{\Abs{\Br - \BR}}
\approx
\inv{\Abs{\Br}} \left(
1 + \inv{\Br} \cdot \BR
\right)
=
\inv{\Abs{\Br}}
+ \frac{\rcap}{\Br^2} \cdot \BR
=
\inv{\Abs{\Br}}
+ \frac{\Br}{\Abs{\Br}^3} \cdot \BR.
\end{equation}
%
Observe this is what was found with the multivariable Taylor series expansion too.



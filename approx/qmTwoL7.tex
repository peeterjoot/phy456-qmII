%
% Copyright � 2012 Peeter Joot.  All Rights Reserved.
% Licenced as described in the file LICENSE under the root directory of this GIT repository.
%
%
%
%
%\input{../peeter_prologue_print.tex}
%\input{../peeter_prologue_widescreen.tex}
%
%\chapter{Time dependent pertubation}
\index{time dependent pertubation}
%\chapter{PHY456H1F: Quantum Mechanics II.  Lecture 7 (Taught by Prof J.E. Sipe).  Time dependent perturbation}
\label{chap:qmTwoL7}
%\blogpage{http://sites.google.com/site/peeterjoot/math2011/qmTwoL7.pdf}
%\date{Sept 25, 2011}
%
\section{Recap: Interaction picture}
\index{interaction picture}
We will use the interaction picture to examine time dependent perturbations.  We wrote our Schr\"{o}dinger ket in terms of the interaction ket
%
\begin{equation}\label{eqn:qmTwoL7:10}
\ket{\psi}
= e^{-i H_0 (t - t_0)/\Hbar}
\ket{\psi_I(t)},
\end{equation}
%
where
\begin{equation}\label{eqn:qmTwoL7:30}
\ket{\psi_I}
= U_I(t, t_0) \ket{\psi_I(t_0)}.
\end{equation}
%
Our dynamics is given by the operator equation
\begin{equation}\label{eqn:qmTwoL7:50}
i \Hbar \ddt{} U_I(t, t_0) = \overbar{H}'(t) U_I(t, t_0),
\end{equation}
%
where
%
\begin{equation}\label{eqn:qmTwoL7:70}
\overbar{H}'(t) =
e^{\frac{i}{\Hbar} H_0(t - t_0)} H'(t) e^{-\frac{i}{\Hbar} H_0(t - t_0)}.
\end{equation}
%
We can formally solve \eqnref{eqn:qmTwoL7:50} by writing
%
\begin{equation}\label{eqn:qmTwoL7:90}
U_I(t, t_0) = I - \frac{i}{\Hbar} \int_{t_0}^t dt' \overbar{H}'(t') U_I(t', t_0).
\end{equation}
%
This is easy enough to verify by direct differentiation
%
\begin{equation}\label{eqn:qmTwoL7:670}
\begin{aligned}
i \Hbar \ddt{} U_I
&=
\left(\int_{t_0}^t dt' \overbar{H}'(t') U_I(t', t_0) \right)' \\
&=
\overbar{H}'(t) U_I(t, t_0) \frac{dt}{dt}
-
\overbar{H}'(t) U_I(t, t_0) \frac{dt_0}{dt} \\
&=
\overbar{H}'(t) U_I(t, t_0)
\end{aligned}
\end{equation}
%
This is a bit of a chicken and an egg expression, since it is cyclic with a dependency on unknown \(U_I(t', t_0)\) factors.

We start with an initial estimate of the operator to be determined, and iterate.  This can seem like an odd thing to do, but one can find books on just this integral kernel iteration method (like the nice little Dover book \citep{tricomi1985integral} that has sat on my (Peeter's) shelf all lonely so many years).

Suppose for \(t\) near \(t_0\), try
%
\begin{equation}\label{eqn:qmTwoL7:110}
U_I(t, t_0) \approx
I - \frac{i}{\Hbar} \int_{t_0}^t dt' \overbar{H}'(t').
\end{equation}
%
A second order iteration is now possible
%
\begin{equation}\label{eqn:qmTwoL7:130}
\begin{aligned}
U_I(t, t_0)
&\approx
I - \frac{i}{\Hbar} \int_{t_0}^t dt' \overbar{H}'(t') \left(
I - \frac{i}{\Hbar} \int_{t_0}^{t'} dt'' \overbar{H}'(t'').
\right) \\
&=
I - \frac{i}{\Hbar} \int_{t_0}^t dt' \overbar{H}'(t') + \left(\frac{-i}{\Hbar}\right)^2
\int_{t_0}^t dt' \overbar{H}'(t') \int_{t_0}^{t'} dt'' \overbar{H}'(t'')
\end{aligned}
\end{equation}
%
It is possible to continue this iteration, and this approach is considered in some detail in \S 3.3 of the text \citep{desai2009quantum}, and is apparently also the basis for Feynman diagrams.

\section{Time dependent perturbation theory}

As covered in \S 17 of the text, we will split the interaction into time independent and time dependent terms
%
\begin{equation}\label{eqn:qmTwoL7:150}
H(t) = H_0 + H'(t),
\end{equation}
%
and work in the interaction picture with
%
\begin{equation}\label{eqn:qmTwoL7:170}
\ket{\psi_I(t)} = \sum_n \tilde{c}_n(t) \ket{\psi_n^{(0)} }.
\end{equation}
%
Our Schr\"{o}dinger ket is then
%
\begin{equation}\label{eqn:qmTwoL7:190}
\begin{aligned}
\ket{\psi(t)}
&=
e^{-i H_0^{(0)}(t- t_0)/\Hbar}
\ket{\psi_I(t_0) } \\
&=
\sum_n \tilde{c}_n(t)
e^{-i E_n^{(0)}(t- t_0)/\Hbar}
\ket{\psi_n^{(0)} }.
\end{aligned}
\end{equation}
%
With a definition
%
\begin{equation}\label{eqn:qmTwoL7:210}
c_n(t) = \tilde{c}_n(t) e^{i E_n t_0/\Hbar},
\end{equation}
%
(where we leave off the zero superscript for the unperturbed state), our time evolved ket becomes
%
\begin{equation}\label{eqn:qmTwoL7:230}
\ket{\psi(t)}
=
\sum_n c_n(t)
e^{-i E_n t/\Hbar}
\ket{\psi_n^{(0)} }.
\end{equation}
%
We can now plug \eqnref{eqn:qmTwoL7:170} into our evolution equation
%
\begin{equation}\label{eqn:qmTwoL7:690}
\begin{aligned}
i\Hbar \ddt{} \ket{\psi_I(t)}
&=
\overbar{H}'(t) \ket{\psi_I(t)} \\
&=
e^{\frac{i}{\Hbar} H_0(t - t_0)} H'(t) e^{-\frac{i}{\Hbar} H_0(t - t_0)}
\ket{\psi_I(t)},
\end{aligned}
\end{equation}
%
which gives us
%\ket{\psi_I(t)} = \sum_n
%\tilde{c}_n(t) \ket{\psi_n^{(0)} }.
%
\begin{equation}\label{eqn:qmTwoL7:250}
i \Hbar \sum_p \PD{t}{}
\tilde{c}_p(t) \ket{\psi_p^{(0)} }
=
e^{\frac{i}{\Hbar} H_0(t - t_0)} H'(t) e^{-\frac{i}{\Hbar} H_0(t - t_0)}
\sum_n
\tilde{c}_n(t) \ket{\psi_n^{(0)} }.
\end{equation}
%
We can apply the bra \(\bra{\psi_m^{(0)}}\) to this equation, yielding
%
\begin{equation}\label{eqn:qmTwoL7:270}
i \Hbar \PD{t}{}
\tilde{c}_m(t)
=
\sum_n
\tilde{c}_n(t)
e^{\frac{i}{\Hbar} E_m(t - t_0)}
\bra{\psi_m^{(0)}} H'(t)
\ket{\psi_n^{(0)} }
e^{-\frac{i}{\Hbar} E_n(t - t_0)}.
\end{equation}
%
With
%
\begin{equation}\label{eqn:qmTwoL7:290}
\begin{aligned}
\omega_m &= \frac{E_m}{\Hbar} \\
\omega_{mn} &= \omega_m - \omega_n \\
H_{mn}'(t) &= \bra{\psi_m^{(0)}} H'(t) \ket{\psi_n^{(0)} },
\end{aligned}
\end{equation}
%
this is
\begin{equation}\label{eqn:qmTwoL7:310}
i \Hbar \PD{t}{
\tilde{c}_m(t) }
=
\sum_n
\tilde{c}_n(t)
e^{
i \omega_{mn}(t - t_0)}
H_{mn}'(t)
\end{equation}
%
Inverting \eqnref{eqn:qmTwoL7:210} and plugging in
%
\begin{equation}\label{eqn:qmTwoL7:330}
\tilde{c}_n(t) = c_n(t) e^{-i \omega_n t_0},
\end{equation}
%
yields
%
\begin{equation}\label{eqn:qmTwoL7:350}
i \Hbar \PD{t}{
c_m(t)
}
e^{-i \omega_m t_0}
=
\sum_n
c_n(t) e^{-i \omega_n t_0}
e^{i\omega_{mn}t}
e^{-i(\omega_m -\omega_n) t_0}
H_{mn}'(t),
\end{equation}
%
from which we can cancel the exponentials on both sides yielding
\begin{equation}\label{eqn:qmTwoL7:370}
i \Hbar \PD{t}{
c_m(t)
}
=
\sum_n
c_n(t)
e^{i\omega_{mn}t}
H_{mn}'(t)
\end{equation}
%
We are now left with all of our time dependence nicely separated out, with the coefficients \(c_n(t)\) encoding all the non-oscillatory time evolution information
%
\begin{equation}\label{eqn:qmTwoL7:390}
\begin{aligned}
H &= H_0 + H'(t) \\
\ket{\psi(t)} &= \sum_n c_n(t) e^{-i\omega_n t} \ket{\psi_n^{(0)}} \\
i \Hbar \dot{c}_m &= \sum_n H_{mn}'(t) e^{i \omega_{mn} t} c_n(t)
\end{aligned}
\end{equation}
%
\section{Perturbation expansion}

We now introduce our \(\lambda\) parametrization
%
\begin{equation}\label{eqn:qmTwoL7:410}
H'(t) \rightarrow \lambda H'(t),
\end{equation}
%
and hope for convergence, or at least something that at least has well defined asymptotic behavior.  We have
%
\begin{equation}\label{eqn:qmTwoL7:430}
i \Hbar \dot{c}_m = \lambda \sum_n H_{mn}'(t) e^{i \omega_{mn} t} c_n(t),
\end{equation}
%
and try
%
\begin{equation}\label{eqn:qmTwoL7:450}
c_m(t) = c_m^{(0)}(t) + \lambda c_m^{(1)}(t) + \lambda^2 c_m^{(2)}(t) + \cdots
\end{equation}
%
Plugging in, we have
%
\begin{equation}\label{eqn:qmTwoL7:470}
i \Hbar
\sum_k
\lambda^k \dot{c}_m^{(k)}(t)
=
\sum_{n,p} H_{mn}'(t) e^{i \omega_{mn} t}
\lambda^{p+1} c_n^{(p)}(t).
\end{equation}
%
As before, for equality, we treat this as an equation for each \(\lambda^k\).  Expanding explicitly for the first few powers, gives us
%
\begin{equation}\label{eqn:qmTwoL7:710}
\begin{aligned}
0
&= \lambda^0 \left( i \Hbar \dot{c}_m^{(0)}(t) - 0 \right) \\
&+ \lambda^1 \left( i \Hbar \dot{c}_m^{(1)}(t) -
\sum_{n} H_{mn}'(t) e^{i \omega_{mn} t}
c_n^{(0)}(t)
\right) \\
&+ \lambda^2 \left( i \Hbar \dot{c}_m^{(2)}(t) -
\sum_{n} H_{mn}'(t) e^{i \omega_{mn} t}
c_n^{(1)}(t)
\right) \\
&\vdots
\end{aligned}
\end{equation}
%
Suppose we have a set of energy levels as depicted in \cref{fig:qmTwoL7:1}.
\imageFigure{../figures/phy456-qmII/qmTwoL7fig1}{Perturbation around energy level s}{fig:qmTwoL7:1}{0.3}
With \(c_n^{(i)} = 0\) before the perturbation for all \(i \ge 1, n\) and \(c_m^{(0)} = \delta_{ms}\), we can proceed iteratively, solving each equation, starting with
%
\begin{equation}\label{eqn:qmTwoL7:490}
i \Hbar \dot{c}_m^{(1)} = H_{ms}'(t) e^{i \omega_{ms} t}
\end{equation}
%
\makeexample{Slow nucleus passing an atom}{l7:ex1}{
%
\begin{equation}\label{eqn:qmTwoL7:510}
H'(t) = - \Bmu \cdot \BE(t)
\end{equation}
%
with
\begin{equation}\label{eqn:qmTwoL7:530}
H_{ms}' = -\Bmu_{ms} \cdot \BE(t),
\end{equation}
%
where
\begin{equation}\label{eqn:qmTwoL7:550}
\Bmu_{ms} =
\bra{\psi_m^{(0)}}
\Bmu
\ket{\psi_s^{(0)}}.
\end{equation}
%
Using our previous nucleus passing an atom example, as depicted in \cref{fig:qmTwoL7:2}.
\imageFigure{../figures/phy456-qmII/qmTwoL7fig2}{Slow nucleus passing an atom}{fig:qmTwoL7:2}{0.3}
We have
%
\begin{equation}\label{eqn:qmTwoL7:570}
\Bmu = \sum_i q_i \BR_i,
\end{equation}
%
the dipole moment for each of the charges in the atom.  We will have fields as depicted in \cref{fig:qmTwoL7:3}.
\imageFigure{../figures/phy456-qmII/qmTwoL7fig3}{Fields for nucleus atom example}{fig:qmTwoL7:3}{0.3}
%FIXME: think through.
}
\makeexample{Electromagnetic wave pulse interacting with an atom}{l7:ex2}{
Consider a EM wave pulse, perhaps Gaussian, of the form depicted in \cref{fig:qmTwoL7:4}.
\imageFigure{../figures/phy456-qmII/qmTwoL7fig4}{Atom interacting with an EM pulse}{fig:qmTwoL7:4}{0.3}
%
\begin{equation}\label{eqn:qmTwoL7:590}
E_y(t) = e^{-t^2/T^2} \cos(\omega_0 t).
\end{equation}
%
As we learned very early, perhaps sitting on our mother's knee, we can solve the differential equation \eqnref{eqn:qmTwoL7:490} for the first order perturbation, by direct integration
%
\begin{equation}\label{eqn:qmTwoL7:510b}
c_m^{(1)}(t) =
\inv{i \Hbar} \int_{-\infty}^t
H_{ms}'(t') e^{i \omega_{ms} t'} dt'.
\end{equation}
%
Here the perturbation is assumed equal to zero at \(-\infty\).  Suppose our electric field is specified in terms of a Fourier transform
%
\begin{equation}\label{eqn:qmTwoL7:530b}
\BE(t) = \int_{-\infty}^\infty \frac{d \omega}{2\pi} \BE(\omega) e^{-i \omega t},
\end{equation}
%
so
%
\begin{equation}\label{eqn:qmTwoL7:550b}
c_m^{(1)}(t) =
\frac{\Bmu_{ms}}{2 \pi i \Hbar} \cdot
\int_{-\infty}^\infty
\int_{-\infty}^t
\BE(\omega)
e^{i (\omega_{ms} -\omega) t'} dt' d\omega.
\end{equation}
%
From this, ``after the perturbation'', as \(t \rightarrow \infty\) we find
%
\begin{equation}\label{eqn:qmTwoL7:730}
\begin{aligned}
c_m^{(1)}(\infty)
&=
\frac{\Bmu_{ms}}{2 \pi i \Hbar} \cdot
\int_{-\infty}^\infty
\int_{-\infty}^\infty
\BE(\omega)
e^{i (\omega_{ms} -\omega) t'} dt' d\omega \\
&=
\frac{\Bmu_{ms}}{i \Hbar} \cdot
\int_{-\infty}^\infty
\BE(\omega)
\delta(\omega_{ms} - \omega)
d\omega
\end{aligned}
\end{equation}
%
since we identify
%
\begin{equation}\label{eqn:qmTwoL7:570b}
\inv{2 \pi}
\int_{-\infty}^\infty
e^{i (\omega_{ms} -\omega) t'} dt' \equiv \delta(\omega_{ms} - \omega)
\end{equation}
%
Thus the steady state first order perturbation coefficient is
%
\begin{equation}\label{eqn:qmTwoL7:590b}
c_m^{(1)}(\infty)
=
\frac{\Bmu_{ms}}{i \Hbar} \cdot
\BE(\omega_{ms}).
\end{equation}
%
\paragraph{Frequency symmetry for the Fourier spectrum of a real field}
%
We will look further at this next week, but we first require an intermediate result from transform theory.  Because our field is real, we have
%
\begin{equation}\label{eqn:qmTwoL7:610}
\BE^\conj(t) = \BE(t)
\end{equation}
%
so
%
\begin{equation}\label{eqn:qmTwoL7:750}
\begin{aligned}
\BE^\conj(t)
&= \int \frac{d\omega}{2 \pi} \BE^\conj(\omega) e^{i \omega t} \\
&= \int \frac{d\omega}{2 \pi} \BE^\conj(-\omega) e^{-i \omega t} \\
\end{aligned}
\end{equation}
%
and thus
%
\begin{equation}\label{eqn:qmTwoL7:630}
\BE(\omega) = \BE^\conj(-\omega),
\end{equation}
%
and
\begin{equation}\label{eqn:qmTwoL7:650}
\Abs{\BE(\omega)}^2 = \Abs{\BE(-\omega)}^2.
\end{equation}
%
We will see shortly what the point of this aside is.
}

\shipoutAnswer

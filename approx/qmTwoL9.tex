%
% Copyright � 2012 Peeter Joot.  All Rights Reserved.
% Licenced as described in the file LICENSE under the root directory of this GIT repository.
%

%
%
%\input{../peeter_prologue_print.tex}
%\input{../peeter_prologue_widescreen.tex}

%\chapter{PHY456H1F: Quantum Mechanics II.  Lecture 9 (Taught by Prof J.E. Sipe).  Adiabatic perturbation theory (cont.)}
\index{adiabatic perturbation}
%\chapter{Adiabatic perturbation theory (cont.), and Fermi's golden rule}
\label{chap:qmTwoL9}
\blogpage{http://sites.google.com/site/peeterjoot/math2011/qmTwoL9.pdf}
%\date{Oct 5, 2011}





\section{Adiabatic perturbation theory (cont.)}

We were working through Adiabatic time dependent perturbation (as also covered in \S 17.5.2 of the text \citep{desai2009quantum}.)

Utilizing an expansion
%
\begin{equation}\label{eqn:qmTwoL9:10}
\begin{aligned}
\ket{\psi(t)} &= \sum_n c_n(t) e^{- i \omega_n^{(0)} t} \ket{\psi_n^{(0)} } \\
&= \sum_n b_n(t) \ket{\hat{\psi}_n(t)},
\end{aligned}
\end{equation}
%
where
%
\begin{equation}\label{eqn:qmTwoL9:30}
H(t) \ket{\hat{\psi}_s(t)} = E_s(t) \ket{\hat{\psi}_s(t)}
\end{equation}
%
and found
%
\begin{equation}\label{eqn:qmTwoL9:50}
\ddt{b_s(t)} =
-i \left(
\omega_s(t) - \Gamma_s(t)
\right) b_s(t)
-
\sum_{n \ne s} b_n(t)
\bra{\hat{\psi}_s(t)}
\ddt{} \ket{\hat{\psi}_n(t)}
\end{equation}
%
where
%
\begin{equation}\label{eqn:qmTwoL9:70}
\Gamma_s(t) =
i \bra{\hat{\psi}_s(t)} \ddt{} \ket{\hat{\psi}_s(t)}
\end{equation}
%
Look for a solution of the form
%
\begin{equation}\label{eqn:qmTwoL9:90}
\begin{aligned}
b_s(t) &= \overbar{b}_s(t) e^{-i \int_0^t dt' (\omega_s(t') - \Gamma_s(t'))} \\
&=
\overbar{b}_s(t) e^{-i \gamma_s(t)}
\end{aligned}
\end{equation}
%
where
\begin{equation}\label{eqn:qmTwoL9:110}
\gamma_s(t) =
\int_0^t dt' (\omega_s(t') - \Gamma_s(t')).
\end{equation}
%
Taking derivatives of \(\overbar{b}_s\) and after a bit of manipulation we find that things conveniently cancel
%
\begin{equation}\label{eqn:qmTwoL9:330}
\begin{aligned}
\ddt{\overbar{b}_s(t)}
&= \ddt{} \left( b_s(t) e^{i \gamma_s(t) } \right) \\
&=
\ddt{b_s(t)} e^{i \gamma_s(t) } +
b_s(t) \ddt{} e^{i \gamma_s(t) }  \\
&=
\ddt{b_s(t)} e^{i \gamma_s(t) } +
b_s(t) i (\omega_s(t) - \Gamma_s(t)) e^{i \gamma_s(t) }.
\end{aligned}
\end{equation}
%
We find
%
\begin{equation}\label{eqn:qmTwoL9:350}
\begin{aligned}
\ddt{\overbar{b}_s(t)}
e^{-i \gamma_s(t)}
&=
\ddt{b_s(t)} + i b_s(t) (\omega_s(t) - \Gamma_s(t))  \\
&=
\cancel{i b_s(t) (\omega_s(t) - \Gamma_s(t)) }
-\cancel{i \left(
\omega_s(t) - \Gamma_s(t)
\right) b_s(t)}
-
\sum_{n \ne s} b_n(t)
\bra{\hat{\psi}_s(t)}
\ddt{} \ket{\hat{\psi}_n(t)},
\end{aligned}
\end{equation}
%
so
%
\begin{equation}\label{eqn:qmTwoL9:370}
\begin{aligned}
\ddt{\overbar{b}_s(t)}
&=
-
\sum_{n \ne s} b_n(t)
e^{i \gamma_s(t)}
\bra{\hat{\psi}_s(t)}
\ddt{} \ket{\hat{\psi}_n(t)} \\
&=
-
\sum_{n \ne s} \overbar{b}_n(t)
e^{i (\gamma_s(t) - \gamma_n(t))}
\bra{\hat{\psi}_s(t)}
\ddt{} \ket{\hat{\psi}_n(t)}.
\end{aligned}
\end{equation}
%
With a last bit of notation
%
\begin{equation}\label{eqn:qmTwoL9:130}
\gamma_{sn}(t) = \gamma_s(t) - \gamma_n(t)),
\end{equation}
%
the problem is reduced to one involving only the sums over the \(n \ne s\) terms, and where all the dependence on \(\bra{\hat{\psi}_s(t)} \ddt{} \ket{\hat{\psi}_s(t)}\) has been nicely isolated in a phase term
%
\begin{equation}\label{eqn:qmTwoL9:150}
\ddt{\overbar{b}_s(t)}
=
-
\sum_{n \ne s} \overbar{b}_n(t)
e^{i \gamma_{sn}(t) }
\bra{\hat{\psi}_s(t)}
\ddt{} \ket{\hat{\psi}_n(t)}.
\end{equation}
%
\paragraph{Looking for an approximate solution}
%
\paragraph{Try}: An approximate solution
%
\begin{equation}\label{eqn:qmTwoL9:170}
\overbar{b}_n(t) =
\delta_{nm}
\end{equation}
%
For \(s = m\) this is okay, since we have \(\ddt{\delta_{ns}} = 0\) which is consistent with
%
\begin{equation}\label{eqn:qmTwoL9:190}
\sum_{n \ne s} \delta_{ns} ( \cdots ) = 0
\end{equation}
%
However, for \(s \ne m\) we get
%
\begin{equation}\label{eqn:qmTwoL9:390}
\begin{aligned}
\ddt{\overbar{b}_s(t)}
&=
-
\sum_{n \ne s}
\delta_{nm}
e^{i \gamma_{sn}(t) }
\bra{\hat{\psi}_s(t)}
\ddt{} \ket{\hat{\psi}_n(t)} \\
&=
-
e^{i \gamma_{sm}(t) }
\bra{\hat{\psi}_s(t)}
\ddt{} \ket{\hat{\psi}_m(t)} \\
\end{aligned}
\end{equation}
%
But
%
\begin{equation}\label{eqn:qmTwoL9:210}
\gamma_{sm}(t) = \int_0^t dt' \left( \inv{\Hbar}( E_s(t') - E_m(t') ) - \Gamma_s(t') + \Gamma_m(t') \right)
\end{equation}
%
FIXME: I think we argued in class that the \(\Gamma\) contributions are negligible.  Why was that?

Now, are energy levels will have variation with time, as illustrated in \cref{fig:qmTwoL9:1}.
\pdfTexFigure{../figures/phy456-qmII/qmTwoL9fig1.pdf_tex}{Energy level variation with time}{fig:qmTwoL9:1}{0.2}
Perhaps unrealistically, suppose that our energy levels have some ``typical'' energy difference \(\Delta E\), so that
%
\begin{equation}\label{eqn:qmTwoL9:230}
\gamma_{sm}(t) \approx \frac{\Delta E}{\Hbar} t \equiv \frac{t}{\tau},
\end{equation}
%
or
%
\begin{equation}\label{eqn:qmTwoL9:250}
\tau = \frac{\Hbar}{\Delta E}
\end{equation}
%
Suppose that \(\tau\) is much less than a typical time \(T\) over which instantaneous quantities (wavefunctions and brakets) change.  After a large time \(T\)
%
\begin{equation}\label{eqn:qmTwoL9:270}
e^{i \gamma_{sm}(t)} \approx e^{i T/\tau}
\end{equation}
%
so we have our phase term whipping around really fast, as illustrated in \cref{fig:qmTwoL9:2}.

\pdfTexFigure{../figures/phy456-qmII/qmTwoL9fig2.pdf_tex}{Phase whipping around}{fig:qmTwoL9:2}{0.3}

So, while \(\bra{\hat{\psi}_s(t)} \ddt{} \ket{\hat{\psi}_m(t)}\) is moving really slow, but our phase space portion is changing really fast.  The key to the approximate solution is factoring out this quickly changing phase term.
%
\paragraph{Note} \(\Gamma_s(t)\) is called the ``Berry'' phase \citep{wiki:GeometricPhase}, whereas the \(E_s(t')/\Hbar\) part is called the geometric phase, and can be shown to have a geometric interpretation.
%
To proceed we can introduce \(\lambda\) terms, perhaps
%
\begin{equation}\label{eqn:qmTwoL9:290}
\overbar{b}_s(t) = \delta_{ms} + \lambda \overbar{b}^{(1)}_s(t) + \cdots
\end{equation}
%
and
\begin{equation}\label{eqn:qmTwoL9:310}
- \sum_{n \ne s} e^{i \gamma_{sn}(t)} \lambda (\cdots)
\end{equation}
%
This \(\lambda\) approximation and a similar Taylor series expansion in time have been explored further in \ref{chap:adiabaticApprox}.
%
\paragraph{Degeneracy}
Suppose we have some branching of energy levels that were initially degenerate, as illustrated in \cref{fig:qmTwoL9:3}.
\pdfTexFigure{../figures/phy456-qmII/qmTwoL9fig3.pdf_tex}{Degenerate energy level splitting}{fig:qmTwoL9:3}{0.3}
We have a necessity to choose states properly so there is a continuous evolution in the instantaneous eigenvalues as \(H(t)\) changes.
%\paragraph{Question: A physical  example?}
%FIXME: Prof Sipe to ponder and revisit.

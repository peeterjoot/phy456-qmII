%
% Copyright � 2012 Peeter Joot.  All Rights Reserved.
% Licenced as described in the file LICENSE under the root directory of this GIT repository.
%

%
%
%\input{../peeter_prologue_print.tex}
%\input{../peeter_prologue_widescreen.tex}

\index{WKB method}
\index{Stark shift}
%\chapter{WKB method and Stark shift}
%\chapter{PHY456H1F: Quantum Mechanics II.  Recitation 3 (Taught by Mr. Federico Duque Gomez).  WKB method and Stark shift}
\label{chap:qmTwoR3}
\blogpage{http://sites.google.com/site/peeterjoot/math2011/qmTwoR3.pdf}
%\date{Oct 28, 2011}





% wkb
\makeexample{Infinite well potential}{r3:ex1}{

Consider the potential
%
\begin{equation}\label{eqn:qmTwoR3:10}
V(x) =
\left\{
\begin{array}{l l}
v(x) & \quad \mbox{if \(x \in [0,a]\)} \\
\infty & \quad \mbox{otherwise} \\
\end{array}
\right.
\end{equation}
%
as illustrated in \cref{fig:qmTwoR3:qmTwoR3fig1}.
\imageFigure{../figures/phy456-qmII/qmTwoR3fig1}{Arbitrary potential in an infinite well}{fig:qmTwoR3:qmTwoR3fig1}{0.3}
Inside the well, we have
%
\begin{equation}\label{eqn:qmTwoR3:30}
\psi(x) = \inv{\sqrt{k(x)}} \left(
C_{+} e^{i \int_0^x k(x') dx'}
+C_{-} e^{-i \int_0^x k(x') dx'}
\right)
\end{equation}
%
where
%
\begin{equation}\label{eqn:qmTwoR3:50}
k(x) = \inv{\Hbar} \sqrt{ 2m( E - v(x) }
\end{equation}
%
With
\begin{equation}\label{eqn:qmTwoR3:70}
\phi(x) = e^{\int_0^x k(x') dx'}
\end{equation}
%
We have
%
\begin{equation}\label{eqn:qmTwoR3:250}
\begin{aligned}
\psi(x)
&= \inv{\sqrt{k(x)}} \left(
C_{+}(\cos \phi + i\sin\phi) + C_{-}(\cos\phi - i \sin\phi)
\right) \\
&= \inv{\sqrt{k(x)}} \left(
(C_{+} + C_{-})\cos \phi + i(C_{+} - C_{-}) \sin\phi
\right) \\
&= \inv{\sqrt{k(x)}} \left(
(C_{+} + C_{-})\cos \phi + i(C_{+} - C_{-}) \sin\phi
\right) \\
&\equiv
\inv{\sqrt{k(x)}} \left(
C_2 \cos \phi + C_1 \sin\phi
\right),
\end{aligned}
\end{equation}
%
Where
\begin{equation}\label{eqn:qmTwoR3:85}
\begin{aligned}
C_2 &= C_{+} + C_{-} \\
C_1 &= i( C_{+} - C_{-})
\end{aligned}
\end{equation}
%
Setting boundary conditions we have
%
\begin{equation}\label{eqn:qmTwoR3:90}
\phi(0) = 0
\end{equation}
%
Noting that we have \(\phi(0) = 0\), we have
%
\begin{equation}\label{eqn:qmTwoR3:110}
\inv{\sqrt{k(0)}} C_2 = 0
\end{equation}
%
So
%
\begin{equation}\label{eqn:qmTwoR3:130}
\psi(x)
\sim
\inv{\sqrt{k(x)}} \sin\phi
\end{equation}
%
At the other boundary
%
\begin{equation}\label{eqn:qmTwoR3:150}
\psi(a) = 0
\end{equation}
%
So we require
%
\begin{equation}\label{eqn:qmTwoR3:170}
\sin \phi(a) = \sin(n \pi)
\end{equation}
%
or
%
\begin{equation}\label{eqn:qmTwoR3:190}
\inv{\Hbar} \int_0^a \sqrt{2 m (E - v(x')} dx' = n \pi
\end{equation}
%
This is called the Bohr-Sommerfeld condition.
%
\paragraph{Check} with \(v(x) = 0\).
%
We have
%
\begin{equation}\label{eqn:qmTwoR3:210}
\inv{\Hbar} \sqrt{2m E} a = n \pi
\end{equation}
%
or
%
\begin{equation}\label{eqn:qmTwoR3:230}
E = \inv{2m} \left(\frac{n \pi \Hbar}{a}\right)^2
\end{equation}
}


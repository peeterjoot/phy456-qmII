%
% Copyright � 2013 Peeter Joot.  All Rights Reserved.
% Licenced as described in the file LICENSE under the root directory of this GIT repository.
%
%
%Time independent perturbation theory
\makeexample{Stark Shift}{r3:ex2}{
Reading: \S 16.5 of \citep{desai2009quantum}.
%
\begin{equation}\label{eqn:qmTwoR3Stark:250}
H = H_0 + \lambda H'
\end{equation}
%
\begin{equation}\label{eqn:qmTwoR3Stark:270}
H' = e \calE_z \hat{Z}
\end{equation}
%
where \(\calE_z\) is the electric field.

To first order we have

%{(1)} = {(1)}
\begin{equation}\label{eqn:qmTwoR3Stark:290}
\ket{\psi_\alpha^{(1)}} = \ket{\psi_\alpha^{(0)}}
+
\sum_{\beta \ne \alpha} \frac{
\ket{\psi_\beta^{(0)}} \bra{\psi_\beta^{(0)}} H' \ket{\psi_\alpha^{(0)}}
}{
E_\alpha^{(0)}
-E_\beta^{(0)}
}
\end{equation}
%
and
%
\begin{equation}\label{eqn:qmTwoR3Stark:310}
E_\alpha^{(1)} =
\bra{\psi_\alpha^{(0)}} H' \ket{\psi_\alpha^{(0)}}
\end{equation}
%
With the default basis \(\{\ket{\psi_\beta^{(0)}}\}\), and \(n=2\) we have a 4 fold degeneracy
%
\begin{equation}\label{eqn:qmTwoR3Stark:610}
\begin{aligned}
l,m &= 0,0 \\
l,m &= 1,-1 \\
l,m &= 1,0 \\
l,m &= 1,+1
\end{aligned}
\end{equation}
%
but can diagonalize as follows
%
\begin{equation}\label{eqn:qmTwoR3Stark:570}
\begin{bmatrix}
\text{nlm} & 200 & 210 & 211 & 21\,-1 \\
200    & 0 & \Delta & 0 & 0 \\
210    & \Delta & 0 & 0 & 0 \\
211    & 0 & 0 & 0 & 0 \\
21\,-1 & 0 & 0 & 0 & 0
\end{bmatrix}
\end{equation}
%
FIXME: show.

where
\begin{equation}\label{eqn:qmTwoR3Stark:590}
\Delta = -3 e \calE_z a_0
\end{equation}
%
We have a split of energy levels as illustrated in \cref{fig:qmTwoR3Stark:qmTwoR3fig2}.
\imageFigure{../figures/phy456-qmII/qmTwoR3fig2}{Energy level splitting}{fig:qmTwoR3Stark:qmTwoR3fig2}{0.3}
%
Observe the embedded Pauli matrix (FIXME: missed the point of this?)
%
\begin{equation}\label{eqn:qmTwoR3Stark:330}
\sigma_x = \PauliX
\end{equation}
%
Proper basis for perturbation (FIXME:check) is then
%
\begin{equation}\label{eqn:qmTwoR3Stark:350}
\left\{
\inv{\sqrt{2}}
(
\ket{2,0,0}
\pm
\ket{2,1,0}
),
\ket{2, 1, \pm 1}
\right\}
\end{equation}
%
and our result is
%
\begin{equation}\label{eqn:qmTwoR3Stark:370}
\ket{\psi_{\alpha, n=2}^{(1)}} =
\ket{\psi_{\alpha}^{(0)}}
+\sum_{\beta \notin \text{degenerate subspace}} \frac{
\ket{\psi_\beta^{(0)}} \bra{\psi_\beta^{(0)}} H' \ket{\psi_\alpha^{(0)}}
}{
E_\alpha^{(0)}
-E_\beta^{(0)}
}
\end{equation}
}


%
% Copyright � 2012 Peeter Joot.  All Rights Reserved.
% Licenced as described in the file LICENSE under the root directory of this GIT repository.
%
%
\label{chap:adiabaticApproximation}
%
%\blogpage{http://sites.google.com/site/peeterjoot2/math2011/adiabaticApproximation.pdf}
%\date{Dec 11, 2011}
%
\paragraph{Motivation.}
%
In class we were shown an adiabatic approximation where we started with (or worked our way towards) a representation of the form
%
\begin{equation}\label{eqn:adiabaticApproximation:10}
\ket{\psi} = \sum_k c_k(t) e^{-i \int_0^t (\omega_k(t') - \Gamma_k(t')) dt' } \ket{\psi_k(t)}
\end{equation}
%
where \(\ket{\psi_k(t)}\) were normalized energy eigenkets for the (slowly) evolving Hamiltonian
%
\begin{equation}\label{eqn:adiabaticApproximation:30}
H(t) \ket{\psi_k(t)} = E_k(t) \ket{\psi_k(t)}
\end{equation}
%
In the problem sets we were shown a different adiabatic approximation, where are starting point is
%
\begin{equation}\label{eqn:adiabaticApproximation:50}
\ket{\psi(t)} = \sum_k c_k(t) \ket{\psi_k(t)}.
\end{equation}
%
For completeness, here is a walk through of the general amplitude derivation that is been used.
%
\paragraph{Guts.}
%
We operate with our energy identity once again
%
\begin{equation}\label{eqn:adiabaticApproximation:230}
\begin{aligned}
0
&=
\left(H - i \Hbar \ddt{} \right) \sum_k c_k \ket{k} \\
&=
\sum_k c_k E_k \ket{k} - i \Hbar c_k' \ket{k} - i \Hbar c_k \ket{k'} ,
\end{aligned}
\end{equation}
%
where
%
\begin{equation}\label{eqn:adiabaticApproximation:70}
\ket{k'} = \ddt{} \ket{k}.
\end{equation}
%
Bra'ing with \(\bra{m}\), and split the sum into \(k = m\) and \(k \ne m\) parts
%
\begin{equation}\label{eqn:adiabaticApproximation:90}
0 =
c_m E_m - i \Hbar c_m'
- i \Hbar c_m \braket{m}{m'}
- i \Hbar \sum_{k \ne m} c_k \braket{m}{k'}
\end{equation}
%
Again writing
%
\begin{equation}\label{eqn:adiabaticApproximation:110}
\Gamma_m = i \braket{m}{m'}
\end{equation}
%
We have
%
\begin{equation}\label{eqn:adiabaticApproximation:130}
c_m' = \inv{i \Hbar} c_m (E_m - \Hbar \Gamma_m) - \sum_{k \ne m} c_k \braket{m}{k'},
\end{equation}
%
In this form we can make an ``Adiabatic'' approximation, dropping the \(k \ne m\) terms, and integrate
%
\begin{equation}\label{eqn:adiabaticApproximation:150}
\int \frac{d c_m'}{c_m} = \inv{i \Hbar} \int_0^t (E_m(t') - \Hbar \Gamma_m(t')) dt'
\end{equation}
%
or
%
\begin{equation}\label{eqn:adiabaticApproximation:170}
c_m(t) = A \exp\left(
\inv{i \Hbar} \int_0^t (E_m(t') - \Hbar \Gamma_m(t')) dt'
\right).
\end{equation}
%
Evaluating at \(t = 0\), fixes the integration constant for
%
\begin{equation}\label{eqn:adiabaticApproximation:190}
c_m(t) = c_m(0) \exp\left(
\inv{i \Hbar} \int_0^t (E_m(t') - \Hbar \Gamma_m(t')) dt'
\right).
\end{equation}
%
Observe that this is very close to the starting point of the adiabatic approximation we performed in class since we end up with
%
\begin{equation}\label{eqn:adiabaticApproximation:210}
\ket{\psi} = \sum_k c_k(0) e^{-i \int_0^t (\omega_k(t') - \Gamma_k(t')) dt' } \ket{k(t)},
\end{equation}
%
So, to perform the more detailed approximation, that started with \eqnref{eqn:adiabaticApproximation:10}, where we ended up with all the cross terms that had both \(\omega_k\) and Berry phase \(\Gamma_k\) dependence, we have only to generalize by replacing \(c_k(0)\) with \(c_k(t)\).




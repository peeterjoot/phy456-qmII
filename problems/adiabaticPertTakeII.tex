%
% Copyright � 2012 Peeter Joot.  All Rights Reserved.
% Licenced as described in the file LICENSE under the root directory of this GIT repository.
%
%
\label{chap:adiabaticPertTakeII}
%
%\blogpage{http://sites.google.com/site/peeterjoot/math2011/adiabaticPertTakeII.pdf}
%\date{Oct 27, 2011}
%
Professor Sipe's adiabatic perturbation and that of the text \citep{desai2009quantum} in \S 17.5.1 and \S 17.5.2 use different notation for \(\gamma_m\) and take a slightly different approach.  We can find Prof Sipe's final result with a bit less work, if a hybrid of the two methods is used.
%
Our starting point is the same, we have a time dependent slowly varying Hamiltonian
%
\begin{equation}\label{eqn:adiabaticPertTakeII:10}
H = H(t),
\end{equation}
%
where our perturbation starts at some specific time from a given initial state
%
\begin{equation}\label{eqn:adiabaticPertTakeII:30}
H(t) = H_0, \qquad t \le 0.
\end{equation}
%
We assume that instantaneous eigenkets can be found, satisfying
%
\begin{equation}\label{eqn:adiabaticPertTakeII:50}
H(t) \ket{n(t)} = E_n(t) \ket{n(t)}
\end{equation}
%
Here I will use \(\ket{n} \equiv \ket{n(t)}\) instead of the \(\ket{\hat{\psi}_n(t)}\) that we used in class because its easier to write.

Now suppose that we have some arbitrary state, expressed in terms of the instantaneous basis kets \(\ket{n}\)
%
\begin{equation}\label{eqn:adiabaticPertTakeII:70}
\ket{\psi} = \sum_n \overbar{b}_n(t) e^{-i\alpha_n + i \beta_n} \ket{n},
\end{equation}
%
where
%
\begin{equation}\label{eqn:adiabaticPertTakeII:90}
\alpha_n(t) = \inv{\Hbar} \int_0^t dt' E_n(t').
\end{equation}
%
Here I have used \(\beta_n\) instead of \(\gamma_n\) (as in the text) to avoid conflicting with the lecture notes, where this \(\beta_n\) is a factor to be determined.


For this state, we have at the time just before the perturbation
%
\begin{equation}\label{eqn:adiabaticPertTakeII:110}
\ket{\psi(0)} = \sum_n \overbar{b}_n(0) e^{-i\alpha_n(0) + i \beta_n(0)} \ket{n(0)}.
\end{equation}
%
The question to answer is: How does this particular state evolve?

Another question, for those that do not like sneaky bastard derivations, is where did that magic factor of \(e^{-i\alpha_n}\) come from in our superposition state? We will see after we start taking derivatives that this is what we need to cancel the \(H(t)\ket{n}\) in Schr\"{o}dinger's equation.

Proceeding to plug into the evolution identity we have
%
\begin{equation}\label{eqn:adiabaticPertTakeII:230}
\begin{aligned}
0 &=
\bra{m} \left( i \Hbar \ddt{} - H(t) \right) \ket{\psi} \\
&=
\bra{m} \left(
\sum_n
e^{-i \alpha_n + i \beta_n}
(i \Hbar) \left(
\ddt{\overbar{b}_n}
+ \overbar{b}_n \left(-i \cancel{\frac{E_n}{\Hbar}} + i \dot{\beta}_m \right)
\right) \ket{n}
+ i \Hbar \overbar{b}_n \ddt{} \ket{n}
- \cancel{E_n \overbar{b}_n \ket{n}}
\right)
\\
&=
e^{-i \alpha_m + i \beta_m}
(i \Hbar)
\ddt{\overbar{b}_m}
+
e^{-i \alpha_m + i \beta_m}
(i \Hbar)
i \dot{\beta}_m \overbar{b}_m
+ i \Hbar \sum_n \overbar{b}_n \bra{m} \ddt{} \ket{n}
e^{-i \alpha_n + i \beta_n} \\
&\sim
\ddt{\overbar{b}_m}
+
i \dot{\beta}_m \overbar{b}_m
+
\sum_n
e^{-i \alpha_n + i \beta_n}
e^{i \alpha_m - i \beta_m}
\overbar{b}_n \bra{m} \ddt{} \ket{n} \\
&=
\ddt{\overbar{b}_m}
+
i \dot{\beta}_m \overbar{b}_m
+
\overbar{b}_m \bra{m} \ddt{} \ket{m}
+
\sum_{n \ne m}
e^{-i \alpha_n + i \beta_n}
e^{i \alpha_m - i \beta_m}
\overbar{b}_n \bra{m} \ddt{} \ket{n}
\end{aligned}
\end{equation}
%
We are free to pick \(\beta_m\) to kill the second and third terms
%
\begin{equation}\label{eqn:adiabaticPertTakeII:130}
0 =
i \dot{\beta}_m \overbar{b}_m
+
\overbar{b}_m \bra{m} \ddt{} \ket{m},
\end{equation}
%
or
%
\begin{equation}\label{eqn:adiabaticPertTakeII:150}
\dot{\beta}_m
=
i \bra{m} \ddt{} \ket{m},
\end{equation}
%
which after integration is
%
\begin{equation}\label{eqn:adiabaticPertTakeII:170}
\beta_m(t)
=
i \int_0^t dt' \bra{m(t')} \frac{d}{dt'} \ket{m(t)}.
\end{equation}
%
In the lecture notes this was written as
%
\begin{equation}\label{eqn:adiabaticPertTakeII:150b}
\Gamma_m(t) = i \bra{m(t)} \ddt{} \ket{m(t)}
\end{equation}
%
so that
%
\begin{equation}\label{eqn:adiabaticPertTakeII:150c}
\beta_m(t) = \int_0^t dt' \Gamma_m(t').
\end{equation}
%
As in class we can observe that this is a purely real function.  We are left with
%
\begin{equation}\label{eqn:adiabaticPertTakeII:190}
\ddt{\overbar{b}_m}
=
-
\sum_{n \ne m} \overbar{b}_n
e^{-i \alpha_{nm} + i \beta_{nm}}
\bra{m} \ddt{} \ket{n}
,
\end{equation}
%
where
%
\begin{equation}\label{eqn:adiabaticPertTakeII:210}
\begin{aligned}
\alpha_{nm} &= \alpha_{n} -\alpha_m \\
\beta_{nm} &= \beta_{n} -\beta_m
\end{aligned}
\end{equation}
%
The task is now to find solutions for these \(\overbar{b}_m\) coefficients, and we can refer to the class notes for that without change.



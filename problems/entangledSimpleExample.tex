%
% Copyright � 2012 Peeter Joot.  All Rights Reserved.
% Licenced as described in the file LICENSE under the root directory of this GIT repository.
%
%
\label{chap:entangledSimpleExample}
%
%\blogpage{http://sites.google.com/site/peeterjoot/math2011/entangledSimpleExample.pdf}
%\date{Oct 9, 2011}
%
On the quiz we were given a three state system \(\ket{1}, \ket{2}\) and \(\ket{3}\), and a two state system \(\ket{a}, \ket{b}\), and were asked to show that the composite system can be entangled.  I had trouble with this, having not seen any examples of this and subsequently filing away entanglement in the ``abstract stuff that has no current known application'' bit bucket, and then forgetting about it.  Let us generate a concrete example of entanglement, and consider the very simplest direct product spaces.
%
What is the simplest composite state that we can create?  Suppose we have a pair of two state systems, say,
%
\begin{equation}\label{eqn:entangledSimpleExample:10}
\begin{aligned}
\ket{1}
&=
\begin{bmatrix}
1 \\
0
\end{bmatrix} \in H_1 \\
\ket{2}
&=
\begin{bmatrix}
0 \\
1
\end{bmatrix} \in H_1,
\end{aligned}
\end{equation}
%
and
%
\begin{equation}\label{eqn:entangledSimpleExample:30}
\begin{aligned}
\braket{x}{+}
&=
\frac{e^{i k x}}{\sqrt{2\pi}},
\text{where}\, \ket{+} \in H_2 \\
\braket{x}{+}
&=
\frac{e^{-i k x}}{\sqrt{2\pi}}
\text{where}\, \ket{-} \in H_2.
\end{aligned}
\end{equation}
%
We can now enumerate the space of possible operators
%
\begin{equation}\label{eqn:entangledSimpleExample:90}
\begin{aligned}
A &\in
a_{11++} \ket{1}\bra{1} \otimes \ket{+}\bra{+}
\,+ a_{11+-} \ket{1}\bra{1} \otimes \ket{+}\bra{-} \\
&+ a_{11-+} \ket{1}\bra{1} \otimes \ket{-}\bra{+}
\,+ a_{11--} \ket{1}\bra{1} \otimes \ket{-}\bra{-} \\
&+ a_{12++} \ket{1}\bra{2} \otimes \ket{+}\bra{+}
\,+ a_{12+-} \ket{1}\bra{2} \otimes \ket{+}\bra{-} \\
&+ a_{12-+} \ket{1}\bra{2} \otimes \ket{-}\bra{+}
\,+ a_{12--} \ket{1}\bra{2} \otimes \ket{-}\bra{-} \\
&+ a_{21++} \ket{2}\bra{1} \otimes \ket{+}\bra{+}
\,+ a_{21+-} \ket{2}\bra{1} \otimes \ket{+}\bra{-} \\
&+ a_{21-+} \ket{2}\bra{1} \otimes \ket{-}\bra{+}
\,+ a_{21--} \ket{2}\bra{1} \otimes \ket{-}\bra{-} \\
&+ a_{22++} \ket{2}\bra{2} \otimes \ket{+}\bra{+}
\,+ a_{22+-} \ket{2}\bra{2} \otimes \ket{+}\bra{-} \\
&+ a_{22-+} \ket{2}\bra{2} \otimes \ket{-}\bra{+}
\,+ a_{22--} \ket{2}\bra{2} \otimes \ket{-}\bra{-}.
\end{aligned}
\end{equation}
%
We can also enumerate all the possible states, some of these can be entangled
%
\begin{equation}\label{eqn:entangledSimpleExample:110}
%\begin{aligned}
\ket{\psi} \in h_{1+} \ket{1} \otimes \ket{+}
+ h_{1-} \ket{1} \otimes \ket{-}
+ h_{2+} \ket{2} \otimes \ket{+}
+ h_{2-} \ket{2} \otimes \ket{-}.
%\end{aligned}
\end{equation}
%
Finally, we can enumerate all the possible product states
%
\begin{dmath}\label{eqn:entangledSimpleExample:130}
%\begin{aligned}
\ket{\psi} \in
 (a_{i} \ket{i}) \otimes (b_{\beta} \ket{\beta})
=
 a_{1} b_{+} \ket{1} \otimes \ket{+}
\,+ a_{1} b_{-} \ket{1} \otimes \ket{-}
\,+ a_{2} b_{+} \ket{2} \otimes \ket{+}
\,+ a_{2} b_{-} \ket{2} \otimes \ket{-}.
%\end{aligned}
\end{dmath}
%
In this simpler example, we have the same dimensionality for both the sets of direct product kets and the ones formed by arbitrary superposition of the composite ket basis elements, but that does not mean that this rules out entanglement.
%
Suppose that, as the product of some operator, we end up with a ket
%
\begin{equation}\label{eqn:entangledSimpleExample:50}
\ket{\psi} = \ket{1} \otimes \ket{+} \,+ \ket{2} \otimes \ket{-}.
\end{equation}
%
Does this have a product representation of the following form
%
\begin{equation}\label{eqn:entangledSimpleExample:70}
\ket{\psi} = (a_{i} \ket{i}) \otimes (b_{\beta} \ket{\beta}) = a_{i} b_{\beta} \ket{i} \otimes \ket{\beta}?
\end{equation}
%
For this to be true we would require
\begin{equation}\label{eqn:entangledSimpleExample:150}
\begin{aligned}
a_1 b_{+} &= 1 \\
a_2 b_{-} &= 1 \\
a_1 b_{-} &= 0 \\
a_2 b_{+} &= 0.
\end{aligned}
\end{equation}
%
However, we can not find a solution to this set of equations.  We require one of \(a_1 = 0\) or \(b_{-} = 0\) for the third equality, but such zeros generate contradictions for one of the first pair of equations.

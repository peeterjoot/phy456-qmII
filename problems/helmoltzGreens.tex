%
% Copyright � 2012 Peeter Joot.  All Rights Reserved.
% Licenced as described in the file LICENSE under the root directory of this GIT repository.
%

\label{chap:helmholtzGreens}

%\blogpage{http://sites.google.com/site/peeterjoot2/math2011/helmholtzGreens.pdf}
%\date{Dec 12, 2011}
%
\paragraph{Motivation}
%
In class this week, looking at an instance of the Helmholtz equation
%
\begin{equation}\label{eqn:helmholtzGreens:10}
\left( \spacegrad^2 + \Bk^2\right) \psi_\Bk(\Br) = s(\Br).
\end{equation}
%
We were told that the Green's function
%
\begin{equation}\label{eqn:helmholtzGreens:30}
\left( \spacegrad^2 + \Bk^2\right) G^0(\Br, \Br') = \delta(\Br- \Br')
\end{equation}
%
that can be used to solve for a particular solution this differential equation via convolution
%
\begin{equation}\label{eqn:helmholtzGreens:50}
\psi_\Bk(\Br) = \int G^0(\Br, \Br') s(\Br') d^3 \Br',
\end{equation}
%
had the value
%
\begin{equation}\label{eqn:helmholtzGreens:70}
G^0(\Br, \Br') = - \inv{4 \pi} \frac{e^{i k \Abs{\Br - \Br'}} }{\Abs{\Br - \Br'}}.
\end{equation}
%
Let us try to verify this.

%\section{Guts}

Application of the Helmholtz differential operator \(\spacegrad^2 + \Bk^2\) on the presumed solution gives
%
\begin{equation}\label{eqn:helmholtzGreens:90}
\left(\spacegrad^2 + \Bk^2\right) \psi_\Bk(\Br) =
- \inv{4 \pi}
\int \left( \spacegrad^2 + \Bk^2 \right)
\frac{e^{i k \Abs{\Br - \Br'}} }{\Abs{\Br - \Br'}}
s(\Br') d^3 \Br'.
\end{equation}
%
% http://tex.stackexchange.com/questions/57947/tex-capacity-exceeded-when-flexisym-scrreport-hyperref-and-section-heading-wi/57950#57950
%\subsection{When \(\Br \ne \Br'\)}
%\subsection{When \texorpdfstring{\(\Br \ne \Br'\)}{r not equal to r prime}}
\paragraph{When \(\Br \ne \Br'\)}
%
To proceed we will need to evaluate
%
\begin{equation}\label{eqn:helmholtzGreens:110}
\spacegrad^2 \frac{e^{i k \Abs{\Br - \Br'}} }{\Abs{\Br - \Br'}}.
\end{equation}
%
Writing \(\mu = \Abs{\Br - \Br'}\) we start with the computation of
%
\begin{equation}\label{eqn:helmoltzGreens:350}
\begin{aligned}
\PD{x}{} \frac{e^{i k \mu} }{\mu}
&=
\PD{x}{\mu} \left( \frac{i k}{\mu} - \inv{\mu^2} \right) e^{i k \mu} \\
&=
\PD{x}{\mu} \left( i k - \inv{\mu} \right) \frac{e^{i k \mu}}{\mu}
\end{aligned}
\end{equation}
%
We see that we will have
%
\begin{equation}\label{eqn:helmholtzGreens:130}
\spacegrad \frac{e^{i k \mu} }{\mu} = \left( i k - \inv{\mu} \right) \frac{e^{i k \mu}}{\mu} \spacegrad \mu.
\end{equation}
%
Taking second derivatives with respect to \(x\) we find
%
\begin{equation}\label{eqn:helmoltzGreens:370}
\begin{aligned}
\PDSq{x}{} \frac{e^{i k \mu} }{\mu}
&=
\PDSq{x}{\mu} \left( i k - \inv{\mu} \right) \frac{e^{i k \mu}}{\mu}
+\PD{x}{\mu} \PD{x}{\mu} \inv{\mu^2} \frac{e^{i k \mu}}{\mu}
+\left( \PD{x}{\mu} \right)^2 \left( i k - \inv{\mu} \right)^2 \frac{e^{i k \mu}}{\mu} \\
&=
\PDSq{x}{\mu} \left( i k - \inv{\mu} \right) \frac{e^{i k \mu}}{\mu}
+\left( \PD{x}{\mu} \right)^2
\left( -k^2 - \frac{ 2 i k }{\mu} + \frac{2}{\mu^2} \right)
\frac{e^{i k \mu}}{\mu}.
\end{aligned}
\end{equation}
%
Our Laplacian is then
%
\begin{equation}\label{eqn:helmholtzGreens:150}
\spacegrad^2
\frac{e^{i k \mu} }{\mu} =
\left( i k - \inv{\mu} \right) \frac{e^{i k \mu}}{\mu} \spacegrad^2 \mu
+
\left( -k^2 - \frac{ 2 i k }{\mu} + \frac{2}{\mu^2} \right)
\frac{e^{i k \mu}}{\mu} (\spacegrad \mu)^2.
\end{equation}
%
Now lets calculate the derivatives of \(\mu\).  Working on \(x\) again, we have
%
\begin{equation}\label{eqn:helmoltzGreens:390}
\begin{aligned}
\PD{x}{} \mu
&=
\PD{x}{} \sqrt{
(x - x')^2
+(y - y')^2
+(z - z')^2
} \\
&=
\inv{2} 2 (x - x')
\inv{\sqrt{
(x - x')^2
+(y - y')^2
+(z - z')^2
}} \\
&=
\frac{x - x'}{\mu}.
\end{aligned}
\end{equation}
%
So we have
%
\begin{equation}\label{eqn:helmholtzGreens:170}
\begin{aligned}
\spacegrad \mu &= \frac{\Br - \Br'}{\mu} \\
(\spacegrad \mu)^2 &= 1
\end{aligned}
\end{equation}
%
Taking second derivatives with respect to \(x\) we find
%
\begin{equation}\label{eqn:helmoltzGreens:410}
\begin{aligned}
\PDSq{x}{} \mu
&= \PD{x}{}
\frac{x - x'}{\mu} \\
&=
\frac{1}{\mu}
- (x - x') \PD{x}{\mu} \inv{\mu^2}
\\
&=
\frac{1}{\mu}
- (x - x') \frac{x - x'}{\mu} \inv{\mu^2}
\\
&=
\frac{1}{\mu}
- (x - x')^2 \inv{\mu^3}.
\end{aligned}
\end{equation}
%
So we find
%
\begin{equation}\label{eqn:helmholtzGreens:190}
\spacegrad^2 \mu =
\frac{3}{\mu}
- \inv{\mu},
\end{equation}
%
or
%
\begin{equation}\label{eqn:helmholtzGreens:210}
\spacegrad^2 \mu = \frac{2}{\mu}.
\end{equation}
%
Inserting this and \((\spacegrad \mu)^2\) into \eqnref{eqn:helmholtzGreens:150} we find
%
\begin{equation}\label{eqn:helmholtzGreens:220}
\begin{aligned}
\spacegrad^2
\frac{e^{i k \mu} }{\mu}
&=
\left( i k - \inv{\mu} \right) \frac{e^{i k \mu}}{\mu} \frac{2}{\mu}
+
\left( -k^2 - \frac{ 2 i k }{\mu} + \frac{2}{\mu^2} \right)
\frac{e^{i k \mu}}{\mu}
&=
-k^2 \frac{e^{i k \mu}}{\mu}
\end{aligned}
\end{equation}
%
This shows us that provided \(\Br \ne \Br'\) we have
%
\begin{equation}\label{eqn:helmholtzGreens:35}
\left(\spacegrad^2 + \Bk^2\right) G^0(\Br, \Br') = 0.
\end{equation}
%
%\subsection{\texorpdfstring{In the neighborhood of \(\Abs{\Br - \Br'} < \epsilon\)}{When r is close to r prime}}
\paragraph{In the neighborhood of \(\Abs{\Br - \Br'} < \epsilon\)}
%
Having shown that we end up with zero everywhere that \(\Br \ne \Br'\) we are left to consider a neighborhood of the volume surrounding the point \(\Br\) in our integral.  Following the Coulomb treatment in \S 2.2 of \citep{schwartz1987pe} we use a spherical volume element centered around \(\Br\) of radius \(\epsilon\), and then convert a divergence to a surface area to evaluate the integral away from the problematic point
%
\begin{equation}\label{eqn:helmoltzGreens:50}
-\inv{4\pi} \int_{\text{all space}} \left(\spacegrad^2 + \Bk^2\right) \frac{e^{i k \Abs{\Br - \Br'}}}{\Abs{\Br - \Br'}} s(\Br') d^3 \Br'
=
-\inv{4\pi} \int_{\Abs{\Br - \Br'} < \epsilon} \left(\spacegrad^2 + \Bk^2\right) \frac{e^{i k \Abs{\Br - \Br'}}}{\Abs{\Br - \Br'}} s(\Br') d^3 \Br'
\end{equation}
%
We make the change of variables \(\Br' = \Br + \Ba\).  We add an explicit \(\Br\) suffix to our Laplacian at the same time to remind us that it is taking derivatives with respect to the coordinates of \(\Br = (x, y, z)\), and not the coordinates of our integration variable \(\Ba = (a_x, a_y, a_z)\).  Assuming sufficient continuity and ``well behavedness'' of \(s(\Br')\) we will be able to pull it out of the integral, giving
%
\begin{equation}\label{eqn:helmoltzGreens:430}
\begin{aligned}
-\inv{4\pi} \int_{\Abs{\Br - \Br'} < \epsilon} \left( \spacegrad_\Br^2 + \Bk^2 \right) \frac{e^{i k \Abs{\Br - \Br'}}}{\Abs{\Br - \Br'}} s(\Br') d^3 \Br'
&=
-\frac{1}{4\pi} \int_{\Abs{\Ba} < \epsilon} \left( \spacegrad_\Br^2 + \Bk^2 \right) \frac{e^{i k \Abs{\Ba}}}{\Abs{\Ba}} s(\Br + \Ba) d^3 \Ba \\
&=
-\frac{s(\Br)}{4\pi} \int_{\Abs{\Ba} < \epsilon} \left( \spacegrad_\Br^2 + \Bk^2 \right) \frac{e^{i k \Abs{\Ba}}}{\Abs{\Ba}} d^3 \Ba
\end{aligned}
\end{equation}
%
Recalling the dependencies on the derivatives of \(\Abs{\Br - \Br'}\) in our previous gradient evaluations, we note that we have
%
\begin{equation}\label{eqn:shortTimeDepPertubation:300}
\begin{aligned}
\spacegrad_\Br \Abs{\Br - \Br'} &= -\spacegrad_\Ba \Abs{\Ba} \\
\left(\spacegrad_\Br \Abs{\Br - \Br'} \right)^2 &= \left( \spacegrad_\Ba \Abs{\Ba} \right)^2 \\
\spacegrad_\Br^2 \Abs{\Br - \Br'} &= \spacegrad_\Ba^2 \Abs{\Ba},
\end{aligned}
\end{equation}
%
so with \(\Ba = \Br - \Br'\), we can rewrite our Laplacian as
%
\begin{equation}\label{eqn:shortTimeDepPertubation:310}
\spacegrad_\Br^2 \frac{e^{i k \Abs{\Br - \Br'}}}{\Abs{\Br - \Br'}}
=
\spacegrad_\Ba^2 \frac{e^{i k \Abs{\Ba}}}{\Abs{\Ba}}
=
\spacegrad_\Ba \cdot \left(\spacegrad_\Ba \frac{e^{i k \Abs{\Ba}}}{\Abs{\Ba}} \right)
\end{equation}
%
This gives us
%
\begin{equation}\label{eqn:helmoltzGreens:450}
\begin{aligned}
-\frac{s(\Br)}{4\pi}
\int_{\Abs{\Ba} < \epsilon} (\spacegrad_\Ba^2 + \Bk^2) \frac{e^{i k \Abs{\Ba}}}{\Abs{\Ba}} d^3 \Ba
&=
-\frac{s(\Br)}{4\pi}
\int_{dV} \spacegrad_\Ba \cdot \left( \spacegrad_\Ba \frac{e^{i k \Abs{\Ba}}}{\Abs{\Ba}} \right) d^3 \Ba
-\frac{s(\Br)}{4\pi} \int_{dV}
\Bk^2 \frac{e^{i k \Abs{\Ba}}}{\Abs{\Ba}} d^3 \Ba  \\
&=
-\frac{s(\Br)}{4\pi}
\int_{dA} \left( \spacegrad_\Ba \frac{e^{i k \Abs{\Ba}}}{\Abs{\Ba}} \right) \cdot \acap d^2 \Ba
-\frac{s(\Br)}{4\pi}
\int_{dV}
\Bk^2 \frac{e^{i k \Abs{\Ba}}}{\Abs{\Ba}} d^3 \Ba
\end{aligned}
\end{equation}
%
To complete these evaluations, we can now employ a spherical coordinate change of variables.  Let us do the \(\Bk^2\) volume integral first.  We have
%
\begin{equation}\label{eqn:helmoltzGreens:470}
\begin{aligned}
\int_{dV}
\Bk^2 \frac{e^{i k \Abs{\Ba}}}{\Abs{\Ba}} d^3 \Ba
&=
\int_{a = 0}^\epsilon \int_{\theta = 0}^\pi \int_{\phi=0}^{2\pi}
\Bk^2 \frac{e^{i k a}}{a} a^2 da \sin\theta d\theta d\phi \\
&=
4\pi k^2
\int_{a = 0}^\epsilon
a e^{i k a} da  \\
&=
4\pi
\int_{u = 0}^{k\epsilon}
u e^{i u} du  \\
&=
4\pi
{\left.
(-i u + 1) e^{i u} \right\vert}_0^{k \epsilon} \\
&=
4 \pi \left( (-i k \epsilon + 1)e^{i k \epsilon} - 1 \right)
\end{aligned}
\end{equation}
%
To evaluate the surface integral we note that we will require only the radial portion of the gradient, so have
%
\begin{equation}\label{eqn:helmoltzGreens:490}
\begin{aligned}
\left( \spacegrad_\Ba \frac{e^{i k \Abs{\Ba}}}{\Abs{\Ba}} \right) \cdot \acap
&=
\left( \acap \PD{a}{} \frac{e^{i k a}}{a} \right) \cdot \acap \\
&=
\PD{a}{} \frac{e^{i k a}}{a} \\
&=
\left( i k \inv{a} - \inv{a^2} \right)
e^{i k a} \\
&=
\left( i k a - 1 \right)
\frac{e^{i k a}}{a^2}
\end{aligned}
\end{equation}
%
Our area element is \(a^2 \sin\theta d\theta d\phi\), so we are left with
%
\begin{equation}\label{eqn:shortTimeDepPertubation:320}
\begin{aligned}
\int_{dA} \left( \spacegrad_\Ba \frac{e^{i k \Abs{\Ba}}}{\Abs{\Ba}} \right) \cdot \acap d^2 \Ba
&=
\evalbar{
\int_{\theta = 0}^\pi \int_{\phi=0}^{2\pi}
\left( i k a - 1 \right)
\frac{e^{i k a}}{a^2}
a^2 \sin\theta d\theta d\phi
}{a = \epsilon}
\\
&=
4 \pi
\left( i k \epsilon - 1 \right) e^{i k \epsilon}.
\end{aligned}
\end{equation}
Putting everything back together we have
\begin{equation}\label{eqn:helmoltzGreens:510}
\begin{aligned}
-\inv{4\pi} \int_{\text{all space}} \left(\spacegrad^2 + \Bk^2\right) \frac{e^{i k \Abs{\Br - \Br'}}}{\Abs{\Br - \Br'}} s(\Br') d^3 \Br'
&=
-s(\Br)
\left(
(-i k \epsilon + 1)e^{i k \epsilon} - 1
+\left( i k \epsilon - 1 \right) e^{i k \epsilon}
\right) \\
&=
-s(\Br)
\left(
(-i k \epsilon + 1 + i k \epsilon - 1 )e^{i k \epsilon} - 1
\right)
\end{aligned}
\end{equation}
But this is just
\boxedEquation{eqn:shortTimeDepPertubation:330}{
-\inv{4\pi} \int_{\text{all space}} \left(\spacegrad^2 + \Bk^2\right) \frac{e^{i k \Abs{\Br - \Br'}}}{\Abs{\Br - \Br'}} s(\Br') d^3 \Br' = s(\Br).
}
This completes the desired verification of the Green's function for the Helmholtz operator.  Observe the perfect cancellation here, so the limit of \(\epsilon \rightarrow 0\) can be independent of how large \(k\) is made.  You have to complete the integrals for both the Laplacian and the \(\Bk^2\) portions of the integrals and add them, \textunderline{before} taking any limits, or else you will get into trouble (as I did in my first attempt).

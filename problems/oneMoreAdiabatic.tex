%
% Copyright � 2012 Peeter Joot.  All Rights Reserved.
% Licenced as described in the file LICENSE under the root directory of this GIT repository.
%
%
\label{chap:oneMoreAdiabatic}
%
%\blogpage{http://sites.google.com/site/peeterjoot2/math2011/oneMoreAdiabatic.pdf}
%\date{Dec 8, 2011}
%
\section{Motivation.}
%
I liked one of the adiabatic perturbation derivations that I did to review the material, and am recording it for reference.
\section{Build up.}
In time dependent perturbation we started after noting that our ket in the interaction picture, for a Hamiltonian \(H = H_0 + H'(t)\), took the form
%
\begin{equation}\label{eqn:oneMoreAdiabatic:10}
\ket{\alpha_S(t)}
= e^{-i H_0 t/\Hbar} \ket{\alpha_I(t)} = e^{-i H_0 t/\Hbar} U_I(t) \ket{\alpha_I(0)}.
\end{equation}
%
Here we have basically assumed that the time evolution can be factored into a portion dependent on only the static portion of the Hamiltonian, with some other operator \(U_I(t)\), providing the remainder of the time evolution.  From \eqnref{eqn:oneMoreAdiabatic:10} that operator \(U_I(t)\) is found to behave according to
%
\begin{equation}\label{eqn:oneMoreAdiabatic:30}
i \Hbar \ddt{U_I} = e^{i H_0 t/\Hbar} H'(t) e^{-i H_0 t/\Hbar} U_I,
\end{equation}
%
but for our purposes we just assumed it existed, and used this for motivation.  With the assumption that the interaction picture kets can be written in terms of the basis kets for the system at \(t=0\) we write our Schr\"{o}dinger ket as
%
\begin{equation}\label{eqn:oneMoreAdiabatic:40}
\ket{\psi}
= \sum_k e^{-i H_0 t/\Hbar} a_k(t) \ket{k}
= \sum_k e^{-i \omega_k t/\Hbar} a_k(t) \ket{k},
\end{equation}
%
where \(\ket{k}\) are the energy eigenkets for the initial time equation problem
%
\begin{equation}\label{eqn:oneMoreAdiabatic:50}
H_0 \ket{k} = E_k^0 \ket{k}.
\end{equation}
%
\section{Adiabatic case.}
\index{adiabatic perturbation}

For the adiabatic problem, we assume the system is changing very slowly, as described by the instantaneous energy eigenkets
%
\begin{equation}\label{eqn:oneMoreAdiabatic:60}
H(t) \ket{k(t)} = E_k(t) \ket{k(t)}.
\end{equation}
%
Can we assume a similar representation to \eqnref{eqn:oneMoreAdiabatic:40} above, but allow \(\ket{k}\) to vary in time?  This does not quite work since \(\ket{k(t)}\) are no longer eigenkets of \(H_0\)
%
\begin{equation}\label{eqn:oneMoreAdiabatic:10b}
\ket{\psi}
= \sum_k e^{-i H_0 t/\Hbar} a_k(t) \ket{k(t)}
\ne \sum_k e^{-i \omega_k t} a_k(t) \ket{k(t)}.
\end{equation}
%
Operating with \(e^{i H_0 t/\Hbar}\) does not give the proper time evolution of \(\ket{k(t)}\), and we will in general have a more complex functional dependence in our evolution operator for each \(\ket{k(t)}\).  Instead of an \(\omega_k t\) dependence in this time evolution operator let us assume we have some function \(\alpha_k(t)\) to be determined, and can write our ket as
%
\begin{equation}\label{eqn:oneMoreAdiabatic:70}
\ket{\psi}
= \sum_k e^{-i \alpha_k(t)} a_k(t) \ket{k(t)}.
\end{equation}
%
Operating on this with our energy operator equation we have
%
\begin{equation}\label{eqn:oneMoreAdiabatic:300}
\begin{aligned}
0
&=
\left(H - i \Hbar \ddt{} \right) \ket{\psi} \\
&=
\left(H - i \Hbar \ddt{} \right) \sum_k e^{-i \alpha_k} a_k \ket{k} \\
&=
\sum_k e^{-i \alpha_k(t)}
\left(
\left(
E_k a_k
-i \Hbar(-i \alpha_k' a_k + a_k')
\right)
\ket{k}
-i \Hbar
a_k \ket{k'}
\right)
\\
\end{aligned}
\end{equation}
%
Here I have written \(\ket{k'} = d\ket{k}/dt\).  In our original time dependent perturbation the \(-i \alpha_k'\) term was \(-i \omega_k\), so this killed off the \(E_k\).  If we assume this still kills off the \(E_k\), we must have
%
\begin{equation}\label{eqn:oneMoreAdiabatic:80}
\alpha_k = \inv{\Hbar} \int_0^t E_k(t') dt',
\end{equation}
%
and are left with
%
\begin{equation}\label{eqn:oneMoreAdiabatic:90}
0
=
\sum_k e^{-i \alpha_k(t)}
\left(
a_k' \ket{k}
+
a_k \ket{k'}
\right).
\end{equation}
%
Bra'ing with \(\bra{m}\) we have
%
\begin{equation}\label{eqn:oneMoreAdiabatic:100}
0
=
e^{-i \alpha_m(t)}
a_m'
+
e^{-i \alpha_m(t)}
a_m \braket{m}{m'}
+
\sum_{k \ne m} e^{-i \alpha_k(t)}
a_k \braket{m}{k'},
\end{equation}
%
or
%
\begin{equation}\label{eqn:oneMoreAdiabatic:110}
a_m'
+
a_m \braket{m}{m'}
=
-
\sum_{k \ne m} e^{-i \alpha_k(t)} e^{i \alpha_m(t)}
a_k \braket{m}{k'},
\end{equation}
%
The LHS is a perfect differential if we introduce an integration factor \(e^{\int_0^t \braket{m}{m'}}\), so we can write
%
\begin{equation}\label{eqn:oneMoreAdiabatic:130}
e^{-\int_0^t \braket{m}{m'}} ( a_m e^{\int_0^t \braket{m}{m'} } )'
=
-
\sum_{k \ne m} e^{-i \alpha_k(t)} e^{i \alpha_m(t)}
a_k \braket{m}{k'},
\end{equation}
%
This suggests that we want to form a new function
%
\begin{equation}\label{eqn:oneMoreAdiabatic:150}
b_m = a_m e^{\int_0^t \braket{m}{m'} }
\end{equation}
%
or
%
\begin{equation}\label{eqn:oneMoreAdiabatic:170}
a_m = b_m e^{-\int_0^t \braket{m}{m'} }
\end{equation}
%
Plugging this into our assumed representation we have a more concrete form
%
\begin{equation}\label{eqn:oneMoreAdiabatic:70b}
\ket{\psi}
= \sum_k e^{- \int_0^t dt' ( i \omega_k + \braket{k}{k'} ) } b_k(t) \ket{k(t)}.
\end{equation}
%
Writing
%
\begin{equation}\label{eqn:oneMoreAdiabatic:180}
\Gamma_k = i \braket{k}{k'},
\end{equation}
%
this becomes
%
\begin{equation}\label{eqn:oneMoreAdiabatic:70c}
\ket{\psi}
= \sum_k e^{- i\int_0^t dt' ( \omega_k - \Gamma_k ) } b_k(t) \ket{k(t)}.
\end{equation}
%
\paragraph{A final pass}
%
Now that we have what appears to be a good representation for any given state if we wish to examine the time evolution, let us start over, reapplying our instantaneous energy operator equality
%
\begin{equation}\label{eqn:oneMoreAdiabatic:320}
\begin{aligned}
0
&=
\left(H - i \Hbar \ddt{} \right)
\ket{\psi}  \\
&=
\left(H - i \Hbar \ddt{} \right)
\sum_k e^{- i\int_0^t dt' ( \omega_k - \Gamma_k ) } b_k \ket{k} \\
&=
- i \Hbar
\sum_k
e^{- i\int_0^t dt' ( \omega_k - \Gamma_k ) }
\left(
i \Gamma_k
b_k \ket{k}
+
b_k' \ket{k}
+
b_k \ket{k'}
\right).
\end{aligned}
\end{equation}
%
Bra'ing with \(\bra{m}\) we find
%
\begin{equation}\label{eqn:oneMoreAdiabatic:340}
\begin{aligned}
0
&=
e^{- i\int_0^t dt' ( \omega_m - \Gamma_m ) }
i \Gamma_m
b_m
+
e^{- i\int_0^t dt' ( \omega_m - \Gamma_m ) }
b_m' \\
&+
e^{- i\int_0^t dt' ( \omega_m - \Gamma_m ) }
b_m \braket{m}{m'}
+
\sum_{k \ne m}
e^{- i\int_0^t dt' ( \omega_k - \Gamma_k ) }
b_k \braket{m}{k'}
\end{aligned}
\end{equation}
%
Since \(i \Gamma_m = \braket{m}{m'}\) the first and third terms cancel leaving us just
%
\begin{equation}\label{eqn:oneMoreAdiabatic:200}
b_m'
=-
\sum_{k \ne m}
e^{- i\int_0^t dt' ( \omega_{km} - \Gamma_{km} ) }
b_k \braket{m}{k'},
\end{equation}
%
where \(\omega_{km} = \omega_k - \omega_m\) and \(\Gamma_{km} = \Gamma_k - \Gamma_m\).
%
\section{Summary.}
%
We assumed that a ket for the system has a representation in the form
%
\begin{equation}\label{eqn:oneMoreAdiabatic:220}
\ket{\psi}
= \sum_k e^{- i \alpha_k(t) } a_k(t) \ket{k(t)},
\end{equation}
%
where \(a_k(t)\) and \(\alpha_k(t)\) are given or to be determined.  Application of our energy operator identity provides us with an alternate representation that simplifies the results
%
\begin{equation}\label{eqn:oneMoreAdiabatic:220b}
\ket{\psi}
= \sum_k e^{- i\int_0^t dt' ( \omega_k - \Gamma_k ) } b_k(t) \ket{k(t)}.
\end{equation}
%
With
%
\begin{equation}\label{eqn:oneMoreAdiabatic:260}
\begin{aligned}
\ket{m'} &= \ddt{} \ket{m} \\
\Gamma_k &= i \braket{m}{m'} \\
\omega_{km} &= \omega_k - \omega_m \\
\Gamma_{km} &= \Gamma_k - \Gamma_m
\end{aligned}
\end{equation}
%
we find that our dynamics of the coefficients are related by
%
\begin{equation}\label{eqn:oneMoreAdiabatic:280}
b_m'
=-
\sum_{k \ne m}
e^{- i\int_0^t dt' ( \omega_{km} - \Gamma_{km} ) }
b_k \braket{m}{k'},
\end{equation}
%



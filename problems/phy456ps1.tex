%
% Copyright � 2012 Peeter Joot.  All Rights Reserved.
% Licenced as described in the file LICENSE under the root directory of this GIT repository.
%

\label{chap:phy456ps1}
%\blogpage{http://sites.google.com/site/peeterjoot/math2011/phy456ps1.pdf}
%%\date{Sept 12, 2011}

\makeoproblem{Harmonic oscillator}{pr:phy456ps1:1}{2011 ps1/p1}{
Let \(H_o\) indicate the Hamiltonian of a 1D harmonic oscillator with mass \(m\) and frequency \(\omega\)

\begin{equation}\label{eqn:phy456ps1:99}
H_o = \frac{P^2}{2m} + \inv{2} m \omega^2 X^2
\end{equation}
%
and denote the energy eigenstates by \(\ket{n}\), where \(n\) is the eigenvalue of the number operator.

\makesubproblem{Find \(\bra{n} X^4 \ket{n}\)}{pr:phy456ps1:1:a}
\makesubproblem{Quadratic pertubation}{pr:phy456ps1:1:b}

Find the ground state energy of the Hamiltonian \(H = H_o + \gamma X^2\).  You may assume \(\gamma > 0\). [Hint: This is not a trick question.]

\makesubproblem{linear pertubation}{pr:phy456ps1:1:c}

Find the ground state energy of the Hamiltonian \(H = H_o - \alpha X\). [Hint: This is a bit harder than \partref{pr:phy456ps1:1:b} but not much. Try "completing the square."]
} % makeoproblem

\makeanswer{pr:phy456ps1:1}{

\makeSubAnswer{\(X^4\)}{pr:phy456ps1:1:a}

Working through \ref{chap:phy456ps1SHO} we have now got enough context to attempt the first part of the question, calculation of

\begin{equation}\label{eqn:phy456ps1:320}
\bra{n} X^4 \ket{n}
\end{equation}
%
We have calculated things like this before, such as
\begin{equation}\label{eqn:phy456ps1:860}
\begin{aligned}
\bra{n} X^2 \ket{n}
&=
\frac{\Hbar}{2 m \omega} \bra{n} (a + a^\dagger)^2 \ket{n}
\end{aligned}
\end{equation}
%
To continue we need an exact relation between \(\ket{n}\) and \(\ket{n \pm 1}\).  Recall that \(a \ket{n}\) was an eigenstate of \(a^\dagger a\) with eigenvalue \(n - 1\).  This implies that the eigenstates \(a \ket{n}\) and \(\ket{n-1}\) are proportional

\begin{equation}\label{eqn:phy456ps1:340}
a \ket{n} = c_n \ket{n - 1},
\end{equation}
%
or
\begin{equation}\label{eqn:phy456ps1:880}
\begin{aligned}
\bra{n} a^\dagger a \ket{n} &= \Abs{c_n}^2 \braket{n - 1}{n-1} = \Abs{c_n}^2 \\
n \braket{n}{n} &= \\
n &=
\end{aligned}
\end{equation}
%
so that

\begin{equation}\label{eqn:phy456ps1:380}
a \ket{n} = \sqrt{n} \ket{n - 1}.
\end{equation}
%
Similarly let

\begin{equation}\label{eqn:phy456ps1:400}
a^\dagger \ket{n} = b_n \ket{n + 1},
\end{equation}
%
or
\begin{equation}\label{eqn:phy456ps1:900}
\begin{aligned}
\bra{n} a a^\dagger \ket{n} &= \Abs{b_n}^2 \braket{n - 1}{n-1} = \Abs{b_n}^2 \\
\bra{n} (1 + a^\dagger a) \ket{n} &= \\
1 + n &=
\end{aligned}
\end{equation}
%
so that

\begin{equation}\label{eqn:phy456ps1:440}
a^\dagger \ket{n} = \sqrt{n+1} \ket{n + 1}.
\end{equation}
%
We can now return to \eqnref{eqn:phy456ps1:320}, and find

\begin{equation}\label{eqn:phy456ps1:920}
\begin{aligned}
\bra{n} X^4 \ket{n}
&=
\frac{\Hbar^2}{4 m^2 \omega^2} \bra{n} (a + a^\dagger)^4 \ket{n}
\end{aligned}
\end{equation}
%
Consider half of this braket

\begin{equation}\label{eqn:phy456ps1:940}
\begin{aligned}
(a + a^\dagger)^2 \ket{n}
&=
\left( a^2 + (a^\dagger)^2 + a^\dagger a + a a^\dagger \right) \ket{n} \\
&=
\left( a^2 + (a^\dagger)^2 + a^\dagger a + (1 + a^\dagger a) \right) \ket{n} \\
&=
\left( a^2 + (a^\dagger)^2 + 1 + 2 a^\dagger a \right) \ket{n} \\
&=
\sqrt{n-1}\sqrt{n-2} \ket{n-2}
+
\sqrt{n+1}\sqrt{n+2} \ket{n + 2}
+
\ket{n}
+  2 n \ket{n}
\end{aligned}
\end{equation}
%
Squaring, utilizing the Hermitian nature of the \(X\) operator %, we have for \(n > 2\)

\begin{equation}\label{eqn:phy456ps1:500}
\bra{n} X^4 \ket{n}
=
\frac{\Hbar^2}{4 m^2 \omega^2}
\left(
(n-1)(n-2) + (n+1)(n+2) + (1 + 2n)^2
\right)
=
\frac{\Hbar^2}{4 m^2 \omega^2}
\left( 6 n^2 + 4 n + 5 \right)
\end{equation}
%
%It also looks like we can drop the \(n > 2\) restriction since the \(\sqrt{n-1}\) and \(\sqrt{n-2}\) factors kill off the
\makeSubAnswer{Quadratic ground state}{pr:phy456ps1:1:b}

Find the ground state energy of the Hamiltonian \(H = H_0 + \gamma X^2\) for \(\gamma > 0\).

The new Hamiltonian has the form

\begin{equation}\label{eqn:phy456ps1:520}
H = \frac{P^2}{2m} + \inv{2} m \left(\omega^2 + \frac{2 \gamma}{m} \right) X^2 =
\frac{P^2}{2m} + \inv{2} m {\omega'}^2 X^2,
\end{equation}
%
where
\begin{equation}\label{eqn:phy456ps1:540}
\omega' = \sqrt{ \omega^2 + \frac{2 \gamma}{m} }
\end{equation}
%
The energy states of the Hamiltonian are thus

\begin{equation}\label{eqn:phy456ps1:560}
E_n = \Hbar \sqrt{ \omega^2 + \frac{2 \gamma}{m} } \left( n + \inv{2} \right)
\end{equation}
%
and the ground state of the modified Hamiltonian \(H\) is thus

\begin{equation}\label{eqn:phy456ps1:580}
E_0 = \frac{\Hbar}{2} \sqrt{ \omega^2 + \frac{2 \gamma}{m} }
\end{equation}
%
\makeSubAnswer{Linear ground state}{pr:phy456ps1:1:c}

Find the ground state energy of the Hamiltonian \(H = H_0 - \alpha X\).

With a bit of play, this new Hamiltonian can be factored into

\begin{equation}\label{eqn:phy456ps1:590}
H
= \Hbar \omega \left( b^\dagger b + \inv{2} \right) - \frac{\alpha^2}{2 m \omega^2}
= \Hbar \omega \left( b b^\dagger - \inv{2} \right) - \frac{\alpha^2}{2 m \omega^2},
\end{equation}
%
where

\begin{equation}\label{eqn:phy456ps1:600}
\begin{aligned}
b &= \sqrt{\frac{m \omega}{2\Hbar}} X + \frac{i P}{\sqrt{2 m \Hbar \omega}} - \frac{\alpha}{\omega \sqrt{ 2 m \Hbar \omega }} \\
b^\dagger &= \sqrt{\frac{m \omega}{2\Hbar}} X - \frac{i P}{\sqrt{2 m \Hbar \omega}} - \frac{\alpha}{\omega \sqrt{ 2 m \Hbar \omega }}.
\end{aligned}
\end{equation}
%
From \eqnref{eqn:phy456ps1:590} we see that we have the same sort of commutator relationship as in the original Hamiltonian

\begin{equation}\label{eqn:phy456ps1:610}
\antisymmetric{b}{b^\dagger} = 1,
\end{equation}
%
and because of this, all the preceding arguments follow unchanged with the exception that the energy eigenstates of this Hamiltonian are shifted by a constant

\begin{equation}\label{eqn:phy456ps1:620}
H \ket{n} = \left( \Hbar \omega \left( n + \inv{2} \right) - \frac{\alpha^2}{2 m \omega^2} \right) \ket{n},
\end{equation}
%
where the \(\ket{n}\) states are simultaneous eigenstates of the \(b^\dagger b\) operator

\begin{equation}\label{eqn:phy456ps1:630}
b^\dagger b \ket{n} = n \ket{n}.
\end{equation}
%
The ground state energy is then
\begin{equation}\label{eqn:phy456ps1:640}
E_0 = \frac{\Hbar \omega }{2} - \frac{\alpha^2}{2 m \omega^2}.
\end{equation}
%
This makes sense.  A translation of the entire position of the system should not effect the energy level distribution of the system, but we have set our reference potential differently, and have this constant energy adjustment to the entire system.
} % makeanswer

\makeoproblem{Expectation values for position operators for spinless hydrogen}{pr:phy456ps1:2}{2011 ps1/p2}{
Show that for all energy eigenstates \(\ket{\Phi_{nlm}}\) of the (spinless) hydrogen atom, where as usual \(n\), \(l\), and \(m\) are respectively the principal, azimuthal, and magnetic quantum numbers, we have

\begin{equation}\label{eqn:phy456ps1:98}
\bra{\Phi_{nlm}}
X
\ket{\Phi_{nlm}}
=
\bra{\Phi_{nlm}}
Y
\ket{\Phi_{nlm}}
=
\bra{\Phi_{nlm}}
Z
\ket{\Phi_{nlm}}
= 0
\end{equation}
%
[Hint: Take note of the parity of the spherical harmonics (see "quick summary" notes on the spherical harmonics).]
} % makeoproblem

\makeanswer{pr:phy456ps1:2}{
%We are asked to show that for any eigenkets of the hydrogen atom \(\ket{\Phi_{nlm}}\) we have
%
%\begin{equation}\label{eqn:phy456ps1:700}
%\bra{\Phi_{nlm}} X \ket{\Phi_{nlm}}
%=
%\bra{\Phi_{nlm}} Y \ket{\Phi_{nlm}}
%=
%\bra{\Phi_{nlm}} Z \ket{\Phi_{nlm}}.
%\end{equation}
%
The summary sheet provides us with the wavefunction
\begin{equation}\label{eqn:phy456ps1:720}
\braket{\Br}{\Phi_{nlm}} =
\frac{2}{n^2 a_0^{3/2}} \sqrt{\frac{(n-l-1)!}{(n+l)!)^3}} F_{nl}\left( \frac{2r}{n a_0} \right) Y_l^m(\theta, \phi),
\end{equation}
%
where \(F_{nl}\) is a real valued function defined in terms of Lagueere polynomials.  Working with the expectation of the \(X\) operator to start with we have

\begin{equation}\label{eqn:phy456ps1:960}
\begin{aligned}
\bra{\Phi_{nlm}} X \ket{\Phi_{nlm}}
&=
\int
\braket{\Phi_{nlm}}{\Br'} \bra{\Br'} X \ket{\Br} \braket{\Br}{\Phi_{nlm}} d^3 \Br d^3 \Br' \\
&=
\int
\braket{\Phi_{nlm}}{\Br'} \delta(\Br - \Br') r \sin\theta \cos\phi \braket{\Br}{\Phi_{nlm}} d^3 \Br d^3 \Br' \\
&=
\int
\Phi_{nlm}^\conj(\Br) r \sin\theta \cos\phi \Phi_{nlm}(\Br) d^3 \Br \\
&\sim
\int r^2 dr \Abs{ F_{nl}\left(\frac{2 r}{ n a_0} \right)}^2 r
\int \sin\theta d\theta d\phi
{Y_l^m}^\conj(\theta, \phi) \sin\theta \cos\phi Y_l^m(\theta, \phi) \\
\end{aligned}
\end{equation}
%
Recalling that the only \(\phi\) dependence in \(Y_l^m\) is \(e^{i m \phi}\) we can perform the \(d\phi\) integration directly, which is

\begin{equation}\label{eqn:phy456ps1:740}
\int_{\phi=0}^{2\pi} \cos\phi d\phi e^{-i m \phi} e^{i m \phi} = 0.
\end{equation}
%
We have the same story for the \(Y\) expectation which is

\begin{equation}\label{eqn:phy456ps1:760}
\bra{\Phi_{nlm}} X \ket{\Phi_{nlm}}
\sim
\int r^2 dr \Abs{F_{nl}\left( \frac{2 r}{ n a_0} \right)}^2 r
\int \sin\theta d\theta d\phi
{Y_l^m}^\conj(\theta, \phi) \sin\theta \sin\phi Y_l^m(\theta, \phi).
\end{equation}
%
Our \(\phi\) integral is then just

\begin{equation}\label{eqn:phy456ps1:780}
\int_{\phi=0}^{2\pi} \sin\phi d\phi e^{-i m \phi} e^{i m \phi} = 0,
\end{equation}
%
also zero.  The \(Z\) expectation is a slightly different story.  There we have

\begin{equation}\label{eqn:phy456ps1:800}
\begin{aligned}
\bra{\Phi_{nlm}} Z \ket{\Phi_{nlm}}
&\sim
\int dr \Abs{F_{nl}\left( \frac{2 r}{ n a_0} \right)}^2 r^3  \\
&\quad \int_0^{2\pi} d\phi
\int_0^\pi \sin \theta d\theta
\left( \sin\theta \right)^{-2m}
\left( \frac{d^{l - m}}{d (\cos\theta)^{l-m}} \sin^{2l}\theta \right)^2
\cos\theta.
\end{aligned}
\end{equation}
%
Within this last integral we can make the substitution

\begin{equation}\label{eqn:phy456ps1:820}
\begin{aligned}
u &= \cos\theta \\
\sin\theta d\theta &= - d(\cos\theta) = -du \\
u &\in [1, -1],
\end{aligned}
\end{equation}
%
and the integral takes the form
\begin{equation}\label{eqn:phy456ps1:840}
-\int_{-1}^1
(-du)
\inv{(1 - u^2)^m}
\left( \frac{d^{l-m}}{d u^{l -m }} (1 - u^2)^l
\right)^2 u.
\end{equation}
%
Here we have the product of two even functions, times one odd function (\(u\)), over a symmetric interval, so the end result is zero, completing the problem.

I was not able to see how to exploit the parity result suggested in the problem, but it was not so bad to show these directly.
} % makeanswer

\makeproblem{Angular momentum operator}{pr:phy456ps1:3}{2011 ps1/p3}{
Working with the appropriate expressions in \it{Cartesian components}, confirm that \(L_i \ket{\psi} = 0\) for each component of angular momentum \(L_i\), if \(\braket{\Br}{\psi} = \psi(\Br)\) is in fact only a function of \(r = \Abs{\Br}\).
} % makeoproblem

\makeanswer{pr:phy456ps1:3}{
In order to proceed, we will have to consider a matrix element, so that we can operate on \(\ket{\psi}\) in position space.  For that matrix element, we can proceed to insert complete states, and reduce the problem to a question of wavefunctions.  That is

\begin{equation}\label{eqn:phy456ps1:980}
\begin{aligned}
\bra{\Br} L_i \ket{\psi}
&=
\int d^3 \Br' \bra{\Br} L_i \ket{\Br'} \braket{\Br'}{\psi} \\
&=
\int d^3 \Br' \bra{\Br} \epsilon_{i a b} X_a P_b \ket{\Br'} \braket{\Br'}{\psi} \\
&=
-i \Hbar \epsilon_{i a b} \int d^3 \Br' x_a \bra{\Br} \PD{X_b}{\psi(\Br')} \ket{\Br'}  \\
&=
-i \Hbar \epsilon_{i a b} \int d^3 \Br' x_a \PD{x_b}{\psi(\Br')} \braket{\Br}{\Br'}  \\
&=
-i \Hbar \epsilon_{i a b} \int d^3 \Br' x_a \PD{x_b}{\psi(\Br')} \delta^3(\Br - \Br') \\
&=
-i \Hbar \epsilon_{i a b} x_a \PD{x_b}{\psi(\Br)}
\end{aligned}
\end{equation}
%
With \(\psi(\Br) = \psi(r)\) we have

\begin{equation}\label{eqn:phy456ps1:1000}
\begin{aligned}
\bra{\Br} L_i \ket{\psi}
&=
-i \Hbar \epsilon_{i a b} x_a \PD{x_b}{\psi(r)}  \\
&=
-i \Hbar \epsilon_{i a b} x_a \PD{x_b}{r} \frac{d\psi(r)}{dr}  \\
&=
-i \Hbar \epsilon_{i a b} x_a \inv{2} 2 x_b \inv{r} \frac{d\psi(r)}{dr}  \\
\end{aligned}
\end{equation}
%
We are left with an sum of a symmetric product \(x_a x_b\) with the antisymmetric tensor \(\epsilon_{i a b}\) so this is zero for all \(i \in [1,3]\).
} % makeanswer

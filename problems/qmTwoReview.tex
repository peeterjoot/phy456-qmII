%
% Copyright � 2012 Peeter Joot.  All Rights Reserved.
% Licenced as described in the file LICENSE under the root directory of this GIT repository.
%
%
\label{chap:qmTwoReview}
%
%\blogpage{http://sites.google.com/site/peeterjoot/math2011/qmTwoReview.pdf}
%\date{Nov 6, 2011}
%
\section{Motivation.}
Here I will summarize what I had put on a cheat sheet for the tests or exam, if one would be allowed.  While I can derive these results, memorization unfortunately appears required for good test performance in this class, and this will give me a good reference of what to memorize.

This set of review notes covers all the approximation methods we covered except for Fermi's golden rule.
\section{Variational method.}
\index{variational method}
We can find an estimate of our ground state energy using
\boxedEquation{eqn:qmTwoReview:310}{
\frac{
\bra{\Psi} H \ket{\Psi}
}{
\braket{\Psi}{\Psi}
}
\ge E_0.
}
\section{Time independent perturbation.}
\index{time independent perturbation}
Given a perturbed Hamiltonian and an associated solution for the unperturbed state
\boxedEquation{eqn:qmTwoReview:330}{
\begin{aligned}
H &= H_0 + \lambda H', \qquad \lambda \in [0,1] \\
H_0 \ket{{\psi_{m\alpha}}^{(0)}} &= {E_m}^{(0)} \ket{{\psi_{m\alpha}}^{(0)}},
\end{aligned}
}
we assume a power series solution for the energy
\begin{equation}\label{eqn:qmTwoReview:350}
E_m = {E_m}^{(0)} + \lambda {E_m}^{(1)} + \lambda^2 {E_m}^{(2)} + \cdots
\end{equation}
For a non-degenerate state \(\ket{\psi_m} = \ket{\psi_{m1}}\), with an unperturbed value of \(\ket{\psi_{m}^{(0)}} = \ket{\psi_{m1}^{(0)}}\), we seek a power series expansion of this ket in the perturbed system
\begin{equation}\label{eqn:qmTwoReview:130}
\begin{aligned}
\ket{\psi_m} &=
\sum_{n,\alpha} {c_{n\alpha;m}}^{(0)} \ket{{\psi_{n\alpha}}^{(0)}}
+
\lambda
\sum_{n,\alpha} {c_{n\alpha;m}}^{(1)} \ket{{\psi_{n\alpha}}^{(0)}}
+
\lambda^2
\sum_{n,\alpha} {c_{n\alpha;m}}^{(2)} \ket{{\psi_{n\alpha}}^{(0)}}
+ \cdots \\
&\propto
\ket{{\psi_m}^{(0)}}
+
\lambda
\sum_{n \ne m, \alpha} {\overbar{c}_{n\alpha;m}}^{(1)} \ket{{\psi_{n\alpha}}^{(0)}}
+
\lambda^2
\sum_{n \ne m, \alpha} {\overbar{c}_{n\alpha;m}}^{(2)} \ket{{\psi_{n\alpha}}^{(0)}}
+ \cdots
\end{aligned}
\end{equation}
%
Any states \(n \ne m\) are allowed to have degeneracy.  For this case, we found to second order in energy and first order in the kets
\boxedEquation{eqn:qmTwoReview:370}{
\begin{aligned}
E_m &= E_m^{(0)} + \lambda {H_{m1;m1}}' + \lambda^2
\sum_{n \ne m, \alpha}
\frac{\Abs{{H_{n\alpha;m1}}'}^2 }
{ E_m^{(0)} - E_n^{(0)} }
+ \cdots
\\
\ket{\psi_m} &\propto \ket{{\psi_m}^{(0)}} + \lambda
\sum_{n \ne m, \alpha}
\frac{{H_{n\alpha;m1}}'}
{ E_m^{(0)} - E_n^{(0)} } \ket{{\psi_{n\alpha}}^{(0)}}
+ \cdots \\
H_{n\alpha;s\beta}' &=
\bra{{\psi_{n\alpha}}^{(0)}}
H'
\ket{{\psi_{s\beta}}^{(0)}}.
\end{aligned}
}
\section{Degeneracy.}
\index{degeneracy}
When the initial energy eigenvalue \(E_m\) has a degeneracy \(\gamma_m > 1\) we use a different approach to compute the perturbed energy eigenkets and perturbed energy eigenvalues.  Writing the kets as \(\ket{m\alpha}\), then we assume that the perturbed ket is a superposition of the kets in the degenerate energy level
\begin{equation}\label{eqn:qmTwoReview:750}
\ket{m \alpha}' = \sum_i c_i \ket{m i}.
\end{equation}
We find that we must have
\begin{equation}\label{eqn:qmTwoReview:770}
\left( (E^0 - E)I + \lambda \begin{bmatrix} H_{mi;mj}' \end{bmatrix} \right)
\begin{bmatrix}
c_1 \\
c_2 \\
\vdots \\
c_{\gamma_m}
\end{bmatrix}
= 0.
\end{equation}
%
Diagonalizing this matrix \(\begin{bmatrix} H_{mi;mj}' \end{bmatrix}\) (a subset of the complete \(H'\) matrix element)
%
\begin{equation}\label{eqn:qmTwoReview:790}
\begin{bmatrix}
\bra{m i} H' \ket{m j}
\end{bmatrix}
= U_m
\begin{bmatrix}
\delta_{ij}
\calH_{m,i}'
\end{bmatrix}
 U_m^\dagger,
\end{equation}
we find, by taking the determinant, that the perturbed energy eigenvalues are in the set
\boxedEquation{eqn:qmTwoReview:810}{
E = E_m^0 + \lambda \calH_{m,i}', \quad i \in [1, \gamma_m].
}
To compute the perturbed kets we must work in a basis for which the block diagonal matrix elements are diagonal for all \(m\), as in
\begin{equation}\label{eqn:qmTwoReview:790b}
\begin{bmatrix}
\bra{m i} H' \ket{m j}
\end{bmatrix}
=
\begin{bmatrix}
\delta_{ij}
\calH_{m,i}'
\end{bmatrix}.
\end{equation}
If that is not the case, then the unitary matrices of \eqnref{eqn:qmTwoReview:790} can be computed, and the matrix
\begin{equation}\label{eqn:qmTwoReview:230}
U =
\begin{bmatrix}
U_1 &   	&  	 & \\
    & U_2 	&  	 & \\
    &     	& \ddots & \\
    &     	&        & U_N \\
\end{bmatrix},
\end{equation}
can be formed.  The kets
\begin{equation}\label{eqn:qmTwoReview:850}
\ket{\overline{m \alpha}} = U^\dagger \ket{m \alpha},
\end{equation}
will still be energy eigenkets of the unperturbed Hamiltonian
\begin{equation}\label{eqn:qmTwoReview:870}
H_0 \ket{\overline{m \alpha}} = E_m^0 \ket{\overline{m \alpha}},
\end{equation}
but also ensure that the partial diagonalization condition of \eqnref{eqn:qmTwoReview:790} is satisfied.  In this basis, dropping overbars, the first order perturbation results found previously for perturbation about a non-degenerate state also hold, allowing us to write
\boxedEquation{eqn:qmTwoReview:830}{
\ket{s \alpha}' = \ket{s \alpha}
+ \lambda \sum_{m \ne s, \beta} \frac{{H'}_{m \beta ; s \alpha}}{ E_s^{(0)} - E_m^{(0)} } \ket{m \beta}
+ \cdots
}
\section{Interaction picture.}
\index{interaction picture}
We split of the Hamiltonian into time independent and time dependent parts, and also factorize the time evolution operator
\boxedEquation{eqn:qmTwoReview:390}{
\begin{aligned}
H &= H_0 + H_I(t) \\
\ket{\alpha_S} &= e^{-i H_0 t/\Hbar } \ket{\alpha_I(t)} = e^{-i H_0 t/\Hbar } U_I(t) \ket{\alpha_I(0)} .
\end{aligned}
}
Plugging into Schr\"{o}dinger's equation we find
\boxedEquation{eqn:qmTwoReview:410}{
\begin{aligned}
i \Hbar \ddt{} \ket{\alpha_I(t)} &= H_I(t) \ket{\alpha_I(t)} \\
i \Hbar \ddt{U_I} &= H_I' U_I \\
H_I'(t) &= e^{i H_0 t/\Hbar } H_I(t) e^{-i H_0 t/\Hbar }
\end{aligned}
}
\section{Time dependent perturbation.}
\index{time dependent perturbation}
We moved on to time dependent perturbations of the form
\boxedEquation{eqn:qmTwoReview:430}{
\begin{aligned}
H(t) &= H_0 + H'(t) \\
H_0 \ket{\psi_n^{(0)} } &= \Hbar \omega_n \ket{\psi_n^{(0)} },
\end{aligned}
}
where \(\Hbar \omega_n\) are the energy eigenvalues, and \(\ket{\psi_n^{(0)} }\) the energy eigenstates of the unperturbed Hamiltonian.
Use of the interaction picture led quickly to the problem of seeking the coefficients describing the perturbed state
\begin{equation}\label{eqn:qmTwoReview:450}
\ket{\psi(t)} = \sum_n c_n(t) e^{-i \omega_n t} \ket{\psi_n^{(0)} },
\end{equation}
and plugging in we found
\boxedEquation{eqn:qmTwoReview:470}{
\begin{aligned}
i \Hbar \dot{c}_s &= \sum_n H_{sn}'(t) e^{i \omega_{sn} t} c_n(t) \\
\omega_{sn} &= \omega_s - \omega_n \\
H_{sn}'(t) &= \bra{\psi_s^{(0)}} H'(t) \ket{\psi_n^{(0)} },
\end{aligned}
}
\paragraph{Perturbation expansion in series}
Introducing a \(\lambda\) parametrized dependence in the perturbation above, and assuming a power series expansion of our coefficients
\boxedEquation{eqn:qmTwoReview:490}{
\begin{aligned}
H'(t) &\rightarrow \lambda H'(t) \\
c_s(t) &= c_s^{(0)}(t) + \lambda c_s^{(1)}(t) + \lambda^2 c_s^{(2)}(t) + \cdots
\end{aligned}
}
we found, after equating powers of \(\lambda\) a set of coupled differential equations
\begin{equation}\label{eqn:qmTwoReview:510}
\begin{aligned}
i \Hbar \dot{c}_s^{(0)}(t) &= 0  \\
i \Hbar \dot{c}_s^{(1)}(t) &= \sum_{n} H_{sn}'(t) e^{i \omega_{sn} t} c_n^{(0)}(t) \\
i \Hbar \dot{c}_s^{(2)}(t) &= \sum_{n} H_{sn}'(t) e^{i \omega_{sn} t} c_n^{(1)}(t) \\
&\vdots
\end{aligned}
\end{equation}
Of particular value was the expansion, assuming that we started with an initial state in energy level \(m\) before the perturbation was ``turned on'' (ie: \(\lambda = 0\)).
\begin{equation}\label{eqn:qmTwoReview:530}
\ket{\psi(t)} = e^{-i \omega_m t} \ket{\psi_m^{(0)} },
\end{equation}
so that \(c_n^{(0)}(t) = \delta_{nm}\).  We then found a first order approximation for the transition probability coefficient of
\boxedEquation{eqn:qmTwoReview:550}{
i \Hbar \dot{c}_m^{(1)} = H_{ms}'(t) e^{i \omega_{ms} t}
}
\section{Sudden perturbations.}
\index{sudden perturbation}
The idea here is that we integrate Schr\"{o}dinger's equation over the small interval containing the changing Hamiltonian
\begin{equation}\label{eqn:qmTwoReview:570}
\ket{\psi(t)} = \ket{\psi(t_0)} + \inv{i\Hbar} \int_{t_0}^t H(t') \ket{\psi(t')} dt',
\end{equation}
and find
\boxedEquation{eqn:qmTwoReview:590}{
\ket{\psi_{\text{after}}} = \ket{\psi_{\text{before}}}.
}
An implication is that, say, we start with a system measured in a given energy, that same system after the change to the Hamiltonian will then be in a state that is now a superposition of eigenkets from the new Hamiltonian.
\section{Adiabatic perturbations.}
\index{adiabatic perturbation}
Given a Hamiltonian that turns on slowly at \(t=0\), a set of instantaneous eigenkets for the duration of the time dependent interval, and a representation in terms of the instantaneous eigenkets
\boxedEquation{eqn:qmTwoReview:610}{
\begin{aligned}
H(t) &= H_0, \qquad t \le 0 \\
H(t) \ket{\psihat_n(t)} &= E_n(t) \ket{\psihat_n(t)} \\
\ket{\psi} &= \sum_n \overbar{b}_n(t) e^{-i\alpha_n + i \beta_n} \ket{\psihat_n} \\
\alpha_n(t) &= \inv{\Hbar} \int_0^t dt' E_n(t').
\end{aligned}
}
Plugging into Schr\"{o}dinger's equation we find
\boxedEquation{eqn:qmTwoReview:630}{
\begin{aligned}
\ddt{\overbar{b}_m} &= - \sum_{n \ne m} \overbar{b}_n e^{-i \gamma_{nm} } \bra{\psihat_m(t)} \ddt{} \ket{\psihat_n(t)}  \\
\gamma_{nm}(t) &= \alpha_n(t) - \alpha_m(t) - (\beta_n(t) - \beta_m(t)) \\
\beta_n(t) &= \int_0^t dt' \Gamma_n(t') \\
\Gamma_n(t) &= i \bra{\psihat_n(t)} \ddt{} \ket{\psihat_n(t)}.
\end{aligned}
}
Here \(\Gamma_n(t)\) is called the Berry phase.
\paragraph{Evolution of a given state}
Given a system initially measured with energy \(E_m(0)\) before the time dependence is ``turned on''
\boxedEquation{eqn:qmTwoReview:650}{
\ket{\psi(0)} = \ket{\psihat_m(0)},
}
we find that the first order Taylor series expansion for the transition probability coefficients are
\boxedEquation{eqn:qmTwoReview:890}{
\barb_s(t) = \delta_{sm} - t (1 - \delta_{sm}) \bra{\psihat_s(0)} \evalbar{\ddt{} \ket{\psihat_m(t)}}{t=0}.
}
If we introduce a \(\lambda\) perturbation, separating all the (slowly changing) time dependent part of the Hamiltonian \(H'\) from the non time dependent parts \(H_0\) as in
\begin{equation}\label{eqn:qmTwoReview:910}
H(t) = H_0 + \lambda H'(t),
\end{equation}
then we find our perturbed coefficients are
\boxedEquation{eqn:qmTwoReview:930}{
\begin{aligned}
\barb_s(t)
&=
\delta_{ms}(1 + \lambda \text{constant}) \\
&\quad-
(1-\delta_{ms}) \lambda
\int_0^t dt'
e^{i \gamma_{sm}(t') } \bra{\psihat_s(t')} \frac{d}{dt'} \ket{\psihat_m(t')}
\end{aligned}
}
\section{WKB.}
\index{WKB}
We write Schr\"{o}dinger's equation as
\boxedEquation{eqn:qmTwoReview:670}{
\begin{aligned}
0 &= \frac{d^2 U}{dx^2} + k^2 U \\
k^2 &= -\kappa^2 = \frac{2m (E - V)}{\Hbar},
\end{aligned}
}
and seek solutions of the form \(U \propto e^{i\phi}\).  Schr\"{o}dinger's equation takes the form
\begin{equation}\label{eqn:qmTwoReview:690}
- (\phi'(x))^2 + i \phi''(x) + k^2(x) = 0.
\end{equation}
Initially setting \(\phi'' = 0\) we refine our approximation to find
\begin{equation}\label{eqn:qmTwoReview:710}
\phi'(x)
= k(x) \sqrt{ 1 + i \frac{k'(x)}{k^2(x)} }.
\end{equation}
To first order, this gives us
\boxedEquation{eqn:qmTwoReview:730}{
U(x) \propto \inv{\sqrt{k(x)}} e^{\pm i \int dx k(x)}.
}
What we did not cover in class, but required in the problems was the Bohr-Sommerfeld condition described in \S 24.1.2 of the text \citep{desai2009quantum}.
\boxedEquation{eqn:qmTwoReview:730b}{
\int_{x_1}^{x_2} dx \sqrt{ 2m (E - V(x))} = \left( n + \inv{2} \right) \pi \Hbar.
}
This was found from the WKB connection formulas, themselves found my some Bessel function arguments that I have to admit that I did not understand.

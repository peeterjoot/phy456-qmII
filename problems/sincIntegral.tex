%
% Copyright � 2012 Peeter Joot.  All Rights Reserved.
% Licenced as described in the file LICENSE under the root directory of this GIT repository.
%
%
%\input{../blogpost.tex}
%\renewcommand{\basename}{sincIntegral}
%\renewcommand{\dirname}{notes/phy456/}
%\newcommand{\dateintitle}{}
%\newcommand{\keywords}{}
%\date{Dec 10, 2011}
%
%\input{../peeter_prologue_print2.tex}
%
%\renewcommand{\onlineurl}{http://sites.google.com/site/peeterjoot2/math2011/sincIntegral.pdf}
%\beginArtNoToc
%
%\generatetitle{Evaluating the squared sinc integral}
%\label{chap:sincIntegral}
%
\makeproblem{Sinc squared integral.}{problem:sincIntegral:1}{
In the Fermi's golden rule lecture we used the result for the integral of the squared \(\sinc\) function, that is
%\begin{equation}\label{eqn:sincIntegral:10}
\begin{equation*}
   \int_{-\infty}^\infty \frac{\sin^2 (x\Abs{\mu})}{x^2} dx = \pi \Abs{\mu}.
\end{equation*}
Prove this.
} % problem
\makeanswer{problem:sincIntegral:1}{
We want to evaluate
%
\begin{equation}\label{eqn:sincIntegral:10}
\int_{-\infty}^\infty \frac{\sin^2 (x\Abs{\mu})}{x^2} dx,
\end{equation}
for which we must remind ourselves about the appropriate contours required to do the evaluation.  
%
We make a few change of variables
%
\begin{equation}\label{eqn:sincIntegral:110}
\begin{aligned}
\int_{-\infty}^\infty \frac{\sin^2 (x\Abs{\mu})}{x^2} dx
&=
\Abs{\mu} \int_{-\infty}^\infty \frac{\sin^2 (y)}{y^2} dy \\
&=
-i \Abs{\mu} \int_{-\infty}^\infty \frac{(e^{iy} - e^{-iy})^2}{(2 i y)^2} i dy \\
&=
-\frac{i \Abs{\mu}}{4} \int_{-i\infty}^{i\infty} \frac{e^{2z} + e^{-2z} - 2}{z^2} dz.
\end{aligned}
\end{equation}
%
Now we pick a contour that is distorted to one side of the origin as in \cref{fig:sincIntegral:sincSquaredContour}.
\imageFigure{../figures/phy456-qmII/sincSquaredContour}{Contour distorted to one side of the double pole at the origin.}{fig:sincIntegral:sincSquaredContour}{0.2}
%
We employ Jordan's theorem (\S 8.12 \citep{lepage1980cva}) now to pick the contours for each of the integrals since we need to ensure the \(e^{\pm z}\) terms converges as \(R \rightarrow \infty\) for the \(z = R e^{i\theta}\) part of the contour.  We can write
%
\begin{equation}\label{eqn:sincIntegral:30}
\int_{-\infty}^\infty \frac{\sin^2 (x\Abs{\mu})}{x^2} dx
=
-\frac{i \Abs{\mu}}{4} \left(
\int_{C_0 + C_2} \frac{e^{2z}}{z^2} dz
+\int_{C_0 + C_1} \frac{e^{-2z}}{z^2} dz
-\int_{C_0 + C_1} \frac{2}{z^2} dz
\right).
\end{equation}
%
The second two integrals both surround no poles (the contours were strategically placed to cowardly avoid the pole of the integrand at the origin), so we have only the first to deal with
%
\begin{equation}\label{eqn:sincIntegral:130}
\begin{aligned}
\int_{C_0 + C_2} \frac{e^{2z}}{z^2} dz
&= 2 \pi i \inv{1!} \evalbar{ \frac{d}{dz} e^{2z}}{z=0} \\
&= 4 \pi i.
\end{aligned}
\end{equation}
%
Putting everything back together we have
%
\begin{equation}\label{eqn:sincIntegral:50}
\int_{-\infty}^\infty \frac{\sin^2 (x\Abs{\mu})}{x^2} dx
=
-\frac{i \Abs{\mu}}{4} 4 \pi i
= \pi \Abs{\mu}.  \qquad \qedmarker
\end{equation}
%
\paragraph{On the cavalier choice of contours.}
%
The choice of which contours to pick above may seem pretty arbitrary, but they are for good reason.  Suppose you picked \(C_0 + C_1\) for the first integral.  On the big \(C_1\) arc, then with a \(z = R e^{i \theta}\) substitution we have
\begin{dmath}\label{eqn:sincIntegral:70}
\Abs{
\int_{C_1} \frac{e^{2 z}}{z^2} dz
}
=
\Abs{
\int_{\theta = \pi/2}^{-\pi/2}
\frac{
e^{ 2 R (\cos\theta + i \sin\theta) }
}{
R^2 e^{ 2 i \theta}
}
R i e^{i \theta} d\theta
}
=
\frac{1}{R}
\Abs{
\int_{\theta = \pi/2}^{-\pi/2}
e^{ 2 R (\cos\theta + i \sin\theta) }
e^{-i \theta} d\theta
}
\le
\frac{1}{R}
\int_{\theta = -\pi/2}^{\pi/2}
\Abs{
e^{ 2 R \cos\theta }
}
d\theta
\le
\frac{\pi e^{2 R}}{R}.
\end{dmath}
This clearly doesn't have the zero convergence property that we desire.  We need to pick the \(C_2\) contour for the first (positive exponent) integral since in that \([\pi/2, 3\pi/2]\) range, \(\cos\theta\) is always negative.  We can however, use the \(C_1\) contour for the second (negative exponent) integral.  Explicitly, again by example, using \(C_2\) contour for the first integral, over that portion of the arc we have
\begin{dmath}\label{eqn:sincIntegral:90}
\Abs{
\int_{C_2} \frac{e^{2 z}}{z^2} dz
}
=
\Abs{
\int_{\theta = \pi/2}^{3 \pi/2}
\frac{
e^{ 2 R (\cos\theta + i \sin\theta) }
}{
R^2 e^{ 2 i \theta}
}
R i e^{i \theta} d\theta
}
=
\frac{1}{R}
\Abs{
\int_{\theta = \pi/2}^{3 \pi/2}
e^{ 2 R (\cos\theta + i \sin\theta) }
e^{-i \theta} d\theta
}
\le
\frac{1}{R}
\int_{\theta = \pi/2}^{3 \pi/2}
\Abs{
e^{ 2 R \cos\theta }
d\theta
}
\approx
\frac{1}{R}
\int_{\theta = \pi/2}^{3 \pi/2}
\Abs{
e^{ -2 R }
d\theta
}
=
\frac{\pi e^{-2 R} }{R}.
\end{dmath}
} % answer
%\EndArticle

%
% Copyright � 2012 Peeter Joot.  All Rights Reserved.
% Licenced as described in the file LICENSE under the root directory of this GIT repository.
%
%
\label{chap:variationHarmonicOscillator}
%
%\blogpage{http://sites.google.com/site/peeterjoot/math2011/variationHarmonicOscillator.pdf}
%\date{Oct 3, 2011}
%
\subsubsection{Recap.  Variational method to find the ground state energy}
%
Problem 3 of \S 24.4 in the text \citep{desai2009quantum} is an interesting one.  It asks to use the variational method to find the ground state energy of a one dimensional harmonic oscillator Hamiltonian.

Somewhat unexpectedly, once I take derivatives equate to zero, I find that the variational parameter beta becomes imaginary?

I tried this twice on paper and pencil, both times getting the same thing.  This seems like a noteworthy problem, and one worth reflecting on a bit.

%
\subsubsection{Recap.  The variational method}
%
Given any, not necessarily normalized wavefunction, with a series representation specified using the energy eigenvectors for the space
%
\begin{equation}\label{eqn:variationHarmonicOscillator:10}
\ket{\psi} = \sum_m c_{m} \ket{\psi_m},
\end{equation}
%
where
%
\begin{equation}\label{eqn:variationHarmonicOscillator:30}
H \ket{\psi_m} = E_m \ket{\psi_m},
\end{equation}
%
and
%
\begin{equation}\label{eqn:variationHarmonicOscillator:50}
\braket{\psi_m}{\psi_n} = \delta_{mn}.
\end{equation}
%
We can perform an energy expectation calculation with respect to this more general state
%
\begin{equation}\label{eqn:variationHarmonicOscillator:491}
\begin{aligned}
\bra{\psi} H \ket{\psi}
&=
\sum_m c_{m}^\conj \bra{\psi_m}
H
\sum_n c_{n} \ket{\psi_n} \\
&=
\sum_m c_{m}^\conj \bra{\psi_m}
\sum_n c_{n} E_n \ket{\psi_n} \\
&=
\sum_{m,n} c_{m}^\conj c_n E_n \braket{\psi_m}{\psi_n} \\
&
\sum_{m} \Abs{c_{m}}^2 E_m \\
&\ge
\sum_{m} \Abs{c_{m}}^2 E_0 \\
&=
E_0 \braket{\psi}{\psi}
\end{aligned}
\end{equation}
%
This allows us to form an estimate of the ground state energy for the system, by using any state vector formed from a superposition of energy eigenstates, by simply calculating
%
\begin{equation}\label{eqn:variationHarmonicOscillator:70}
E_0 \le \frac{\bra{\psi} H \ket{\psi}}{ \braket{\psi}{\psi} }.
\end{equation}
%
One of the examples in the text is to use this to find an approximation of the ground state energy for the Helium atom Hamiltonian
%
\begin{equation}\label{eqn:variationHarmonicOscillator:90}
H =
-\frac{\Hbar^2}{2m} \left(
\spacegrad_1^2
+\spacegrad_1^2\right) - 2 e^2 \left( \inv{r_1} + \inv{r_2} \right) + \frac{e^2}{\Abs{\Br_1 - \Br_2}}.
\end{equation}
%
This calculation is performed using a trial function that was a solution of the interaction free Hamiltonian
%
\begin{equation}\label{eqn:variationHarmonicOscillator:110}
\phi = \frac{Z^3}{\pi a_0^3} e^{-Z (r_1 + r_2)/a_0 }.
\end{equation}
%
This is despite the fact that this is not a solution to the interaction Hamiltonian.  The end result ends up being pretty close to the measured value (although there is a pesky error in the book that appears to require a compensating error somewhere else).

Part of the variational technique used in that problem, is to allow Z to vary, and then once the normalized expectation is computed, set the derivative of that equal to zero to calculate the trial wavefunction as a parameter of Z that has the lowest energy eigenstate for a function of that form.  We find considering the Harmonic oscillator that this final variation does not necessarily produce meaningful results.
%
\subsubsection{The Harmonic oscillator variational problem}
%
The problem asks for the use of the trial wavefunction
%
\begin{equation}\label{eqn:variationHarmonicOscillator:130}
\phi = e^{-\beta \Abs{x}},
\end{equation}
%
to perform the variational calculation above for the Harmonic oscillator Hamiltonian, which has the one dimensional position space representation
%
\begin{equation}\label{eqn:variationHarmonicOscillator:150}
H = -\frac{\Hbar^2}{2m} \frac{d^2}{dx^2} + \inv{2} m \omega^2 x^2.
\end{equation}
%
We can find the normalization easily
%
\begin{equation}\label{eqn:variationHarmonicOscillator:511}
\begin{aligned}
\braket{\phi}{\phi}
&= \int_{-\infty}^\infty e^{- 2 \beta \Abs{x}} dx \\
&= 2 \inv{2 \beta} \int_{0}^\infty e^{- 2 \beta x} 2 \beta dx \\
&= 2 \inv{2 \beta} \int_{0}^\infty e^{- u} du \\
&= \inv{\beta}
\end{aligned}
\end{equation}
%
Using integration by parts, we find for the energy expectation
%
\begin{equation}\label{eqn:variationHarmonicOscillator:531}
\begin{aligned}
\bra{\phi} H \ket{\phi}
&=
\int_{-\infty}^\infty dx
e^{- \beta \Abs{x}}
\left( -\frac{\Hbar^2}{2m} \frac{d^2}{dx^2} + \inv{2} m \omega^2 x^2 \right)
e^{- \beta \Abs{x}}  \\
&=
\lim_{\epsilon \rightarrow 0}
\left(
\int_{-\infty}^{-\epsilon}
+
\int_{-\epsilon}^\epsilon
+
\int_{\epsilon}^\infty
\right)
dx
e^{- \beta \Abs{x}}
\left( -\frac{\Hbar^2}{2m} \frac{d^2}{dx^2} + \inv{2} m \omega^2 x^2 \right)
e^{- \beta \Abs{x}}  \\
&=
2 \int_{0}^\infty dx
e^{ - 2 \beta x }
\left( -\frac{\Hbar^2 \beta^2}{2m} + \inv{2} m \omega^2 x^2 \right)
-
\frac{\Hbar^2}{2m}
\lim_{\epsilon \rightarrow 0}
\int_{-\epsilon}^\epsilon
dx
e^{- \beta \Abs{x}}
\frac{d^2}{dx^2}
e^{- \beta \Abs{x}}
\end{aligned}
\end{equation}
%
The first integral we can do
%
\begin{equation}\label{eqn:variationHarmonicOscillator:551}
\begin{aligned}
2 \int_{0}^\infty dx
e^{- 2 \beta x}
\left( -\frac{\Hbar^2 \beta^2}{2m} + \inv{2} m \omega^2 x^2 \right)
&=
-\frac{\Hbar^2 \beta^2}{m}
\int_{0}^\infty dx e^{- 2 \beta x}
+ m \omega^2
 \int_{0}^\infty dx x^2 e^{- 2 \beta x}  \\
&=
-\frac{\Hbar^2 \beta}{2 m}
\int_{0}^\infty du e^{- u}
+ \frac{m \omega^2 }{8 \beta^3}
 \int_{0}^\infty du u^2 e^{- u}  \\
&=
-\frac{\beta \Hbar^2}{2m} + \frac{m \omega^2}{4 \beta^3}
\end{aligned}
\end{equation}
%
A naive evaluation of this integral requires the origin to be avoided where the derivative of \(\Abs{x}\) becomes undefined.  This also provides a nice way to evaluate this integral because we can double the integral and half the range, eliminating the absolute value.

However, can we assume that the remaining integral is zero?

I thought that we could, but the end result is curious.  I also verified my calculation symbolically in \nbref{24.4.3_attempt_with_mathematica.nb}, but found that Mathematica required some special hand holding to deal with the origin.  Initially I coded this by avoiding the origin as above, but later switched to \(\Abs{x} = \sqrt{x^2}\) which Mathematica treats more gracefully.
%\href{https://github.com/peeterjoot/physicsplay/tree/master/notes/phy456/24.4.3 attempt with mathematica.nb}{using Mathematica}.

Without that last integral, involving our singular \(\Abs{x}'\) and \(\Abs{x}''\) terms, our ground state energy estimation, parameterized by \(\beta\) is
%
\begin{equation}\label{eqn:variationHarmonicOscillator:190}
E[\beta] = -\frac{\beta^2 \Hbar^2}{2m} + \frac{m \omega^2}{4 \beta^2}.
\end{equation}
%
Observe that if we set the derivative of this equal to zero to find the ``best'' beta associated with this trial function
%
\begin{equation}\label{eqn:variationHarmonicOscillator:210}
0 = \PD{\beta}{E} = -\frac{\beta \Hbar^2}{2m} - \frac{m \omega^2}{2 \beta^3}
\end{equation}
%
we find that the parameter beta that best minimizes this ground state energy function is complex with value
%
\begin{equation}\label{eqn:variationHarmonicOscillator:230}
\beta^2 = \pm \frac{i m \omega}{\sqrt{2} \Hbar}.
\end{equation}
%
It appears at first glance that we can not minimize \eqnref{eqn:variationHarmonicOscillator:190} to find a best ground state energy estimate associated with the trial function \eqnref{eqn:variationHarmonicOscillator:130}.  We do however, know the exact ground state energy \(\Hbar \omega/2\) for the Harmonic oscillator.  Is is possible to show that for all \(\beta^2\) we have
%
\begin{equation}\label{eqn:variationHarmonicOscillator:250}
\frac{\Hbar \omega}{2} \le -\frac{\beta^2 \Hbar^2}{2m} + \frac{m \omega^2}{4 \beta^2}
\end{equation}
%
?  This inequality would be expected if we can assume that the trial wavefunction has a Fourier series representation utilizing the actual energy eigenfunctions for the system.

The resolution to this question is avoided once we include the singularity.  This is explored in the last part of these notes.
%
\subsubsection{Is our trial function representable?}
%
I thought perhaps that since the trial wave function for this problem lies outside the span of the Hilbert space that describes the solutions to the Harmonic oscillator.  Another thing of possible interest is the trouble near the origin for this wave function, when operated on by \(P^2/2m\), and this has been (incorrectly assumed to have zero contribution above).

I had initially thought that part of the value of this variational method was that we can use it despite not even knowing what the exact solution is (and in the case of the Helium atom, I believe it was stated in class that an exact closed form solution is not even known).  This makes me wonder what restrictions must be imposed on the trial solutions to get a meaningful answer from the variational calculation?

Suppose that the trial wavefunction is not representable in the solution space.  If that is the case, we need to adjust the treatment to account for that.  Suppose we have
%
\begin{equation}\label{eqn:variationHarmonicOscillator:270}
\ket{\phi} = \sum_n c_n \ket{\psi_n} + c_\perp \ket{\psi_\perp}.
\end{equation}
%
where \(\ket{\psi_\perp}\) is unknown, and presumed not orthogonal to any of the energy eigenkets.  We can still calculate the norm of the trial function
%
\begin{equation}\label{eqn:variationHarmonicOscillator:571}
\begin{aligned}
\braket{\phi}{\phi}
&=
\sum_{n,m} \braket{ c_n \psi_n + c_\perp \psi_\perp}{ c_m \psi_m + c_\perp \psi_\perp} \\
&=
\sum_n \Abs{c_n}^2
+ c_n^\conj c_\perp
\braket{\psi_n}{\psi_\perp}
+ c_n c_\perp^\conj \braket{\psi_\perp}{\psi_n}
+ \Abs{c_\perp}^2
\braket{\psi_\perp}{\psi_\perp} \\
&=
\braket{\psi_\perp}{\psi_\perp} +
\sum_n \Abs{c_n}^2 + 2 \Real \left(c_n^\conj c_\perp \braket{\psi_n}{\psi_\perp} \right).
\end{aligned}
\end{equation}
%
Similarly we can calculate the energy expectation for this unnormalized state and find
%
\begin{equation}\label{eqn:variationHarmonicOscillator:591}
\begin{aligned}
\bra{\phi} H \ket{\phi}
&=
\sum_{n,m} \bra{ c_n \psi_n + c_\perp \psi_\perp} H \ket{ c_m \psi_m + c_\perp \psi_\perp} \\
&=
\sum_n \Abs{c_n}^2 E_n
+ c_n^\conj c_\perp E_n
\braket{\psi_n}{\psi_\perp}
+ c_n c_\perp^\conj E_n \braket{\psi_\perp}{\psi_n}
+ \Abs{c_\perp}^2
\bra{\psi_\perp} H \ket{\psi_\perp}
%&=
%\braket{\psi_\perp} H {\psi_\perp} +
%\sum_n \Abs{c_n}^2 + 2 \Real \left(c_n^\conj c_\perp \braket{\psi_n}{\psi_\perp} \right).
\end{aligned}
\end{equation}
%
Our normalized energy expectation is therefore the considerably messier
%
\begin{equation}\label{eqn:variationHarmonicOscillator:290}
\begin{aligned}
\frac{\bra{\phi} H \ket{\phi}}{
\braket{\phi}{\phi}
}
&=
\frac{
\sum_n \Abs{c_n}^2 E_n
+ c_n^\conj c_\perp E_n
\braket{\psi_n}{\psi_\perp}
+ c_n c_\perp^\conj E_n \braket{\psi_\perp}{\psi_n}
+ \Abs{c_\perp}^2
\bra{\psi_\perp} H \ket{\psi_\perp}
}
{
\braket{\psi_\perp}{\psi_\perp} +
\sum_m \Abs{c_m}^2 + 2 \Real \left(c_m^\conj c_\perp \braket{\psi_m}{\psi_\perp} \right)
} \\
&\ge
\frac{
\sum_n \Abs{c_n}^2 E_0
+ c_n^\conj c_\perp E_n
\braket{\psi_n}{\psi_\perp}
+ c_n c_\perp^\conj E_n \braket{\psi_\perp}{\psi_n}
+ \Abs{c_\perp}^2
\bra{\psi_\perp} H \ket{\psi_\perp}
}
{
\braket{\psi_\perp}{\psi_\perp} +
\sum_m \Abs{c_m}^2 + 2 \Real \left(c_m^\conj c_\perp \braket{\psi_m}{\psi_\perp} \right)
}
\end{aligned}
\end{equation}
%
With a requirement to include the perpendicular cross terms the norm does not just cancel out, leaving us with a clean estimation of the ground state energy.  In order to utilize this variational method, we implicitly have an assumption that the \(\braket{\psi_\perp}{\psi_\perp}\) and \(\braket{\psi_m}{\psi_\perp}\) terms in the denominator are sufficiently small that they can be neglected.
%
\subsubsection{Calculating the Fourier terms}
%
In order to see how much a problem representing this trial function in the Harmonic oscillator wavefunction solution space, we can just calculate the Fourier fit.

Our first few basis functions, with \(\alpha = \sqrt{m \omega/\Hbar}\) are
%
\begin{equation}\label{eqn:variationHarmonicOscillator:611}
\begin{aligned}
u_0 &= \sqrt{\frac{\alpha}{\sqrt{\pi}}} e^{-\alpha^2 x^2/2} \\
u_1 &= \sqrt{\frac{\alpha}{2 \sqrt{\pi}}} (2 \alpha x) e^{-\alpha^2 x^2/2} \\
u_2 &= \sqrt{\frac{\alpha}{8 \sqrt{\pi}}} (4 \alpha^2 x^2 - 2) e^{-\alpha^2 x^2/2}
\end{aligned}
\end{equation}
%
In general our wavefunctions are
%
\begin{equation}\label{eqn:variationHarmonicOscillator:631}
\begin{aligned}
u_n &= N_n H_n(\alpha x) e^{-\alpha^2 x^2/2} \\
N_n &= \sqrt{
\frac{\alpha}{\sqrt{\pi} 2^n n!}
} \\
H_n(\eta) &= (-1)^n e^{\eta^2} \frac{d^n}{d\eta^n} e^{-\eta^2}
\end{aligned}
\end{equation}
%
From which we find
%
\begin{equation}\label{eqn:variationHarmonicOscillator:471}
\psi(x) = e^{-\alpha^2 x^2/2} (N_n)^2 H_n(\alpha x) \int_{-\infty}^\infty H_n(\alpha x) e^{-\alpha^2 x^2/2} \psi(x) dx
\end{equation}
%
Our wave function, with \(\beta=1\) is plotted in \cref{fig:variationHarmonicOscillator:expMinusBetsAbsX}.
\imageFigure{../figures/phy456-qmII/expMinusBetsAbsX}{Exponential trial function with absolute exponential die off}{fig:variationHarmonicOscillator:expMinusBetsAbsX}{0.2}
The zeroth order fitting using the Gaussian exponential is found to be
%
\begin{equation}\label{eqn:variationHarmonicOscillator:310}
\psi_0(x) = \sqrt{2 \beta}
\erfc\left(\frac{\beta }{\sqrt{2} \alpha }\right)
e^{- \alpha^2 x^2/2 +\beta^2/(2 \alpha^2)}
\end{equation}
%
With \(\alpha = \beta = 1\), this is plotted in \cref{fig:variationHarmonicOscillator:expMinusBetsAbsXfirstOrderFitting} and can be seen to match fairly well
%
\imageFigure{../figures/phy456-qmII/expMinusBetsAbsXfirstOrderFitting}{First ten orders, fitting harmonic oscillator wavefunctions to this trial function}{fig:variationHarmonicOscillator:expMinusBetsAbsXfirstOrderFitting}{0.2}
%
The higher order terms get small fast, but we can see in \cref{fig:variationHarmonicOscillator:expMinusBetsAbsXtenthOrderFitting}, where a tenth order fitting is depicted that it would take a number of them to get anything close to the sharp peak that we have in our exponential trial function.
%
\imageFigure{../figures/phy456-qmII/expMinusBetsAbsXtenthOrderFitting}{Tenth order harmonic oscillator wavefunction fitting}{fig:variationHarmonicOscillator:expMinusBetsAbsXtenthOrderFitting}{0.2}
%
Note that all the brakets of even orders in \(n\) with the trial function are zero, which is why the tenth order approximation is only a sum of six terms.

Details for this harmonic oscillator wavefunction fitting can be found in \nbref{gaussian_fitting_for_abs_function.nb} can be found separately, calculated using a Mathematica worksheet.

The question of interest is why we can approximate the trial function so nicely (except at the origin) even with just a first order approximation (polynomial times Gaussian functions where the polynomials are Hankel functions), and we can get an exact value for the lowest energy state using the first order approximation of our trial function, why do we get garbage from the variational method, where enough terms are implicitly included that the peak should be sharp.  It must therefore be important to consider the origin, but how do we give some meaning to the derivative of the absolute value function?  The key (supplied when asking Professor Sipe in office hours for the course) is to express the absolute value function in terms of Heavyside step functions, for which the derivative can be identified as the delta function.
%
\subsubsection{Correcting, treating the origin this way}
%
Here is how we can express the absolute value function using the Heavyside step
%
\begin{equation}\label{eqn:variationHarmonicOscillator:330}
\Abs{x} = x \theta(x) - x \theta(-x),
\end{equation}
%
where the step function is zero for \(x < 0\) and one for \(x > 0\) as plotted in \cref{fig:variationHarmonicOscillator:stepFunction}.
%
\imageFigure{../figures/phy456-qmII/stepFunction}{Unit step function.}{fig:variationHarmonicOscillator:stepFunction}{0.2}
%
Expressed this way, with the identification \(\theta'(x) = \delta(x)\), we have for the derivative of the absolute value function
%
\begin{equation}\label{eqn:variationHarmonicOscillator:651}
\begin{aligned}
\Abs{x}'
&= x' \theta(x) - x' \theta(-x) + x \theta'(x) - x \theta'(-x) \\
&= \theta(x) - \theta(-x) + x \delta(x) + x \delta(-x) \\
&= \theta(x) - \theta(-x) + x \delta(x) + x \delta(x) \\
\end{aligned}
\end{equation}
%
Observe that we have our expected unit derivative for \(x > 0\), and \(-1\) derivative for \(x < 0\).  At the origin our \(\theta\) contributions vanish, and we are left with
%
\begin{equation}\label{eqn:variationHarmonicOscillator:671}
\begin{aligned}
\evalbar{\Abs{x}' }{x=0}
= 2 \evalbar{x \delta(x)}{x=0} \\
\end{aligned}
\end{equation}
%
We have got zero times infinity here, so how do we give meaning to this?  As with any delta functional, we have got to apply it to a well behaved (square integrable) test function \(f(x)\) and integrate.  Doing so we have
%
\begin{equation}\label{eqn:variationHarmonicOscillator:691}
\begin{aligned}
\int_{-\infty}^\infty dx \Abs{x}' f(x)
&= 2 \int_{-\infty}^\infty dx x \delta(x) f(x) \\
&= 2 (0) f(0)
\end{aligned}
\end{equation}
%
This equals zero for any well behaved test function \(f(x)\).  Since the delta function only picks up the contribution at the origin, we can therefore identify \(\Abs{x}'\) as zero at the origin.

Using the same technique, we can express our trial function in terms of steps
%
\begin{equation}\label{eqn:variationHarmonicOscillator:350}
\psi = e^{-\beta \Abs{x}} = \theta(x) e^{-\beta x} + \theta(-x) e^{\beta x}.
\end{equation}
%
This we can now take derivatives of, even at the origin, and find
%
\begin{equation}\label{eqn:variationHarmonicOscillator:711}
\begin{aligned}
\psi'
&= \theta'(x) e^{-\beta x} + \theta'(-x) e^{\beta x} -\beta \theta(x) e^{-\beta x} + \beta \theta(-x) e^{\beta x} \\
&= \delta(x) e^{-\beta x} - \delta(-x) e^{\beta x} -\beta \theta(x) e^{-\beta x} + \beta \theta(-x) e^{\beta x} \\
&= \cancel{\delta(x) e^{-\beta x} - \delta(x) e^{\beta x}} -\beta \theta(x) e^{-\beta x} + \beta \theta(-x) e^{\beta x} \\
&= \beta \left(
-\theta(x) e^{-\beta x} + \theta(-x) e^{\beta x}
\right)
\end{aligned}
\end{equation}
%
Taking second derivatives we find
%
\begin{equation}\label{eqn:variationHarmonicOscillator:731}
\begin{aligned}
\psi''
&= \beta \left(
-\theta'(x) e^{-\beta x} + \theta'(-x) e^{\beta x}
+\beta \theta(x) e^{-\beta x} + \beta \theta(-x) e^{\beta x}
\right) \\
&=
\beta \left(
-\delta(x) e^{-\beta x} - \delta(-x) e^{\beta x}
+\beta \theta(x) e^{-\beta x} + \beta \theta(-x) e^{\beta x}
\right) \\
&= \beta^2 \psi - 2 \beta \delta(x)
\end{aligned}
\end{equation}
%
Now application of the Hamiltonian operator on our trial function gives us
%
\begin{equation}\label{eqn:variationHarmonicOscillator:370}
H \psi = -\frac{\Hbar^2}{2m} \left( \beta^2 \psi - 2 \delta(x) \right) + \inv{2} m \omega^2 x^2 \psi,
\end{equation}
%
so
%
\begin{equation}\label{eqn:variationHarmonicOscillator:751}
\begin{aligned}
\bra{\psi} H \ket{\psi} &=
\int_{-\infty}^\infty
\left(
-\frac{\Hbar^2 \beta^2}{2m} + \inv{2} m \omega^2 x^2
\right) e^{-2 \beta \Abs{x} }
+ \frac{\Hbar^2 \beta}{m}
\int_{-\infty}^\infty \delta(x)
e^{- \beta \Abs{x}} \\
&=
-\frac{\beta \Hbar^2}{2m} + \frac{m \omega^2}{4 \beta^3} + \frac{\Hbar^2 \beta}{m} \\
&=
\frac{\beta \Hbar^2}{2m} + \frac{m \omega^2}{4 \beta^3}.
\end{aligned}
\end{equation}
%
Normalized we have
%
\begin{equation}\label{eqn:variationHarmonicOscillator:390}
E[\beta] = \frac{\bra{\psi} H \ket{\psi}}{\braket{\psi}{\psi}} = \frac{\beta^2 \Hbar^2}{2m} + \frac{m \omega^2}{4 \beta^2}.
\end{equation}
%
This is looking much more promising.  We will have the sign alternation that we require to find a positive, non-complex, value for \(\beta\) when \(E[\beta]\) is minimized.  That is
%
\begin{equation}\label{eqn:variationHarmonicOscillator:410}
0 = \PD{\beta}{E} =
\frac{\beta \Hbar^2}{m} - \frac{m \omega^2}{2 \beta^3},
\end{equation}
%
so the extremum is found at
%
\begin{equation}\label{eqn:variationHarmonicOscillator:430}
\beta^4 = \frac{m^2 \omega^2}{2 \Hbar^2}.
\end{equation}
%
Plugging this back in we find that our trial function associated with the minimum energy (unnormalized still) is
%
\begin{equation}\label{eqn:variationHarmonicOscillator:450}
\psi = e^{-\sqrt{\frac{m \omega x^2}{\sqrt{2} \Hbar}}},
\end{equation}
%
and that energy, after substitution, is
%
\begin{equation}\label{eqn:variationHarmonicOscillator:470}
E[\beta_{\text{min}}] = \frac{\Hbar \omega}{2} \sqrt{2}
\end{equation}
%
We have something that is \(1.4 \times\) the true ground state energy, but is at least a ball park value.  However, to get this result, we have to be very careful to treat our point of singularity.  A derivative that we would call undefined in first year calculus, is not only defined, but required, for this treatment to work!

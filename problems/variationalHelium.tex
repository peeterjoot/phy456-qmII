%
% Copyright � 2012 Peeter Joot.  All Rights Reserved.
% Licenced as described in the file LICENSE under the root directory of this GIT repository.
%
%
%
%
%\input{../peeter_prologue_print.tex}
%\input{../peeter_prologue_widescreen.tex}
%
\label{chap:variationalHelium}
%
%\blogpage{http://sites.google.com/site/peeterjoot/math2011/variationalHelium.pdf}
%\date{Sept 29, 2011}
%
\makeproblem{Helium atom, variational method, first steps.}{problem:variationalHelium:1}{
Verify (24.69) from \citep{desai2009quantum}, calculating the required six fold integral.
} % problem

\makeanswer{problem:variationalHelium:1}{
%
To verify (24.69) from \citep{desai2009quantum}, a six fold integral is required
%
\begin{equation}\label{eqn:variationalHelium:60}
\begin{aligned}
\expectation{
-\frac{\Hbar^2}{2m}
\left( \spacegrad_1^2 + \spacegrad_2^2\right)
}
&=
-\frac{\Hbar^2}{2m}
\int
dr_1 d \Omega_1 r_1^2
dr_2 d \Omega_2 r_2^2
\frac{ Z^6}{\pi^2 a_0^6 }
e^{-(r_1 + r_2) Z/a_0}\\
&\quad \left(
\frac{2}{r_1} \PD{r_1}{}
+ \PDSq{r_1}{}
+\frac{2}{r_2} \PD{r_2}{}
+ \PDSq{r_2}{}
\right)
e^{-(r_1 + r_2) Z/a_0} \\
&=
-\frac{\Hbar^2}{2m}
\frac{ Z^6}{\pi^2 a_0^6 }
(4 \pi)^2
\int
dr_1
dr_2
r_1^2
r_2^2
e^{-(r_1 + r_2) Z/a_0} \\
&\quad \left(
\frac{2}{r_1} \PD{r_1}{}
+ \PDSq{r_1}{}
+\frac{2}{r_2} \PD{r_2}{}
+ \PDSq{r_2}{}
\right)
e^{-(r_1 + r_2) Z/a_0}.
\end{aligned}
\end{equation}
%
Making a change of variables
%
\begin{equation}\label{eqn:variationalHelium:80}
\begin{aligned}
x &= \frac{Z r_1 }{a_0} \\
y &= \frac{Z r_2 }{a_0},
\end{aligned}
\end{equation}
%
we have
%
\begin{equation}\label{eqn:variationalHelium:100}
\begin{aligned}
%\expectation{
\lexpectation
%\Biggl\langle
-\frac{\Hbar^2}{2m} &
\lr{ \spacegrad_1^2 + \spacegrad_2^2 }
%\Biggr\rangle
\rexpectation
%}
\\
&=
-\frac{8 \Hbar^2}{m}
\frac{ Z^2}{ a_0^2 }
\int
dx dy
x^2 y^2
e^{-x - y}
\left(
\frac{2}{x} \PD{x}{}
+ \PDSq{x}{}
+\frac{2}{y} \PD{y}{}
+ \PDSq{y}{}
\right)
e^{-x - y}
 \\
&=
-\frac{8 \Hbar^2}{m}
\frac{ Z^2}{ a_0^2 }
\int
dx dy
x^2 y^2
e^{-x - y}
\left(
-\frac{2}{x}
+ 1
-\frac{2}{y}
+
1
\right)
e^{-x - y}
 \\
&=
\frac{16 \Hbar^2}{m}
\frac{ Z^2}{ a_0^2 }
\int
dx dy
x^2 y^2
e^{-2 x - 2 y}
\left(
\frac{1}{x}
+\frac{1}{y}
- 1
\right)
 \\
&=
\frac{\Hbar^2}{m}
\frac{ Z^2}{ a_0^2 }.
\end{aligned}
\end{equation}
%
With
%
\begin{equation}\label{eqn:desaiCh24:20}
a_0 = \frac{\Hbar^2}{m e^2}.
\end{equation}
%
We have the result from the text
%
\begin{equation}\label{eqn:desaiCh24:40}
\expectation{
-\frac{\Hbar^2}{2m}
\left( \spacegrad_1^2 + \spacegrad_2^2\right)
}
=
\frac{ Z^2 e^2}{ a_0 }.
\end{equation}
%
Verification of (24.70) follows in a similar fashion.  We have
%
\begin{equation}\label{eqn:variationalHelium:120}
\begin{aligned}
\lexpectation
%\expectation{
2 e^2 &
\lr{ \inv{r_1} + \inv{r_2} }
%}
\rexpectation \\
&=
2 e^2 \frac{Z^6}{ \pi^2 a_0^6} (4 \pi)^2 \int e^{-2 (r_1 + r_2) Z/a_0 } r_1^2 r_2^2 dr_1 dr_2
\left( \inv{r_1} + \inv{r_2} \right)
\\
&=
32 e^2 \frac{Z}{ a_0} \int e^{-2 x - 2 y} x^2 y^2 dx dy
\left( \inv{x} + \inv{y} \right)
\\
&=
4 e^2 \frac{Z}{ a_0}.
\end{aligned}
\end{equation}
} % answer
%
\makeproblem{Helium expectation.}
{problem:variationalHelium:2}{
In \S 24.2.1 of the text \citep{desai2009quantum} is an expectation value calculation associated with the Helium atom.  Show that equation (24.76) is wrong.
} % problem
\makeanswer{problem:variationalHelium:2}{
The following hand calculation and a Mathematica calculation
shows that \S 24.2.1 \citep{desai2009quantum} is wrong, provided no
compensating error was made by me.
Start with
%
\begin{equation}\label{eqn:variationalHelium:140}
\begin{aligned}
\expectation{\frac{e^2}{\Abs{\Br_1 - \Br_2}}}
&=
\left( \frac{Z^3}{\pi a_0^3}\right)^2 e^2
\int d^3 k d^3 r_1 d^3 r_2 \inv{2 \pi^2 k^2} e^{i \Bk \cdot (\Br_1 - \Br_2) } e^{ -2 Z (r_1 + r_2)/a_0} \\
&=
\left( \frac{Z^3}{\pi a_0^3}\right)^2 e^2
\inv{2 \pi^2}
\int d^3 k \inv{k^2}
\int d^3 r_1
e^{i \Bk \cdot \Br_1 } e^{ -2 Z r_1 /a_0}
\, \times \\ &\quad
\int
d^3 r_2
e^{-i \Bk \cdot \Br_2 } e^{ -2 Z r_2/a_0}.
\end{aligned}
\end{equation}
%
To evaluate the two last integrals, I figure the author has aligned the axis for the \(d^3 r_1\) volume elements to make the integrals easier.  Specifically, for the first so that \(\Bk \cdot \Br_1 = k r_1 \cos\theta\), so the integral takes the form
%
\begin{equation}\label{eqn:variationalHelium:160}
\begin{aligned}
\int
d^3 r_1
e^{i \Bk \cdot \Br_1 } e^{ -2 Z r_1 /a_0}
&=
-\int
r_1^2 d r_1 d\phi d(\cos\theta)
e^{i k r_1 \cos\theta } e^{ -2 Z r_1 /a_0} \\
&=
- 2 \pi \int_{r=0}^\infty \int_{u=1}^{-1}
r^2 dr du
e^{i k r u } e^{ -2 Z r /a_0} \\
&=
- 2 \pi \int_{r=0}^\infty
r^2 dr
\inv{i k r} \left( e^{-i k r } - e^{i k r} \right) e^{ -2 Z r /a_0} \\
&=
\frac{4 \pi}{k } \int_{r=0}^\infty
r dr
\inv{2i} \left( e^{i k r } - e^{-i k r} \right) e^{ -2 Z r /a_0} \\
&=
\frac{4 \pi}{k } \int_{r=0}^\infty r dr \sin(k r) e^{ -2 Z r /a_0}.
\end{aligned}
\end{equation}
%
For this last, Mathematica gives me (24.75) from the text
%
\begin{equation}\label{eqn:variationalHelium:10}
\int
d^3 r_1
e^{i \Bk \cdot \Br_1 } e^{ -2 Z r_1 /a_0}
=
\frac{ 16 \pi Z a_0^3 }{(k^2 a_0^2 + 4 Z^2)^2}.
\end{equation}
%
For the second integral, if we align the axis so that \(-\Bk \cdot \Br_2 = k r \cos\theta\) and repeat, then we have
%
\begin{equation}\label{eqn:variationalHelium:180}
\begin{aligned}
\expectation{\frac{e^2}{\Abs{\Br_1 - \Br_2}}}
&=
\left( \frac{Z^3}{\pi a_0^3}\right)^2 e^2
\inv{2 \pi^2}
16^2 \pi^2 Z^2 a_0^6
\int d^3 k \inv{k^2}
\frac{ 1 }{(k^2 a_0^2 + 4 Z^2)^4} \\
&=
\frac{128 Z^8}{\pi^2 } e^2
\int dk d\Omega
\frac{ 1 }{(k^2 a_0^2 + 4 Z^2)^4} \\
&=
\frac{512 Z^8}{\pi} e^2
\int dk
\frac{ 1 }{(k^2 a_0^2 + 4 Z^2)^4} \\
&=
\frac{512 Z^8}{\pi} e^2
\int dk
\frac{ 1 }{(k^2 a_0^2 + 4 Z^2)^4}.
\end{aligned}
\end{equation}
%
With \(k a_0 = 2 Z \kappa\) this is
%
\begin{equation}\label{eqn:variationalHelium:200}
\begin{aligned}
\expectation{\frac{e^2}{\Abs{\Br_1 - \Br_2}}}
&=
\frac{512 Z^8}{\pi} e^2
\int d\kappa
\frac{ 2 Z }{a_0}
\frac{ 1 }{(2 Z)^8 (\kappa^2 + 1)^4} \\
&=
\frac{4 Z}{\pi a_0} e^2
\int d\kappa
\frac{ 1 }{(\kappa^2 + 1)^4} \\
&=
\frac{4 Z}{\pi a_0} e^2 \frac{5 \pi}{32}.
\end{aligned}
\end{equation}
%
This gives us
%
\begin{equation}\label{eqn:variationalHelium:260}
\begin{aligned}
\expectation{\frac{e^2}{\Abs{\Br_1 - \Br_2}}}
&=
\frac{4 Z e^2}{ \pi a_0 }
\frac{5 \pi}{32}
&=
\frac{5 Z e^2}{ 8 a_0 }.
\end{aligned}
\end{equation}
} % answer
%

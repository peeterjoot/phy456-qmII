%
% Copyright � 2012 Peeter Joot.  All Rights Reserved.
% Licenced as described in the file LICENSE under the root directory of this GIT repository.
%

%
%
\mychapter{Reading and problem status.}

Reading and problems from our text \citep{desai2009quantum}.

\begin{itemize}
\item \S 30.1: entanglement.  Do not see reason for (30.5).  Otherwise: DONE.
\item \S 30.2: singlet state.  TODO.
\item \S 30.5: problem: TODO.
\item \S 24.2 Variational method: DONE.
\item \S 24.4: problems: 3-6 DONE.  problems 7,10,11 TODO.
\item \S 16.1 - \S 16.2 Time independent perturbation, theory and HO example.  DONE.
\item \S 16.4 stark effect.  Not covered in class.  I am having trouble confirming 16.72 using 16.66?  See \nbref{desai_attempt_to_verify_section_16.3.nb}.  Revisit after course.
\item \S 16.5 degeneracy.  Excellent description (helpful to also read \S 13.1.1 - \S 13.1.2 on two level problems.)  Revisit degeneracy problem set question having read this, and derive the perturbation expansion for two simple cases: 1 two fold degeneracy + 1 non-degenerate state.  1 two fold degeneracy + 2 independent non-degenerate states.  Then do the general derivation.  DONE.
\item \S 16.6: Problems: All look like good ones: TODO.
\item \S 3.3: Review.  Prof Sipe's coverage was much clearer than the text.  DONE.
\item \S 17.1 Time dependent perturbation theory.  DONE.
\item \S 18.3 Coulomb excitation.  Given as an example in lecture 6 when starting time dependent perturbation.  Rather than working with the Taylor expansion and identifying the first term as the electric field the text works with the operators directly, and goes a lot deeper than we did in class (class goes to about 18.53 and then skips the rest of the section).  This would be good to revisit after the course.  TODO.
\item \S 17.5.1 Adiabatic perturbation.  DONE.
\item \S 17.5.2 Berry phase. TODO.
\item \S 17.2 Fermi's golden rule. TODO.
\item \S 17.6 Problems.  Most look appropriate.  TODO.
\item \S 26.5 Generalized spin and angular momentum: DONE.
\item \S 26.6 Representations of angular momentum operators: DONE.
\item \S 26.8: Problems: TODO.
\item \S 24 WKB: portions on connection formulas not studied completely (started Airy and Bessel function review as prep).  TODO.
\item \S 5.1 - \S 5.9 and \S 26.  Ket and spin representation.  review: TODO.
\item \S 16.5 Linear Stark effect (Recitation 3).  TODO.
\item \S 27: up to 27.3.1: DONE.
\item \S 27.3.1+: some parts of time reversal symmetry left.  Much of this chapter not covered in class.  TODO.
\item \S 28: addition of angular momentum.  DONE.
\item \S 28.5: problems: TODO.
\item \S 29: irreducible tensors.  DONE up to (29.61)
\item \S 29.3: Wigner-Eckart theorem.
\item \S 29.6: problems: TODO.
\item \S 19: Scattering.  DONE: up to before (19.102).  Tunneling and \S 19.5 approximations: TODO.
\item \S 19.7: Problems. 1-5: TODO.
\item \S 20: 3D Scattering. TODO.
\item \S 20.14: Problems: TODO.
\item \S 17.3 - \S 17.4 scattering cross section, resonance and decay.  TODO.
\end{itemize}

NOTE: Section references for readings includes up to lecture 21.
%-------------------------------------------------------

%
% Copyright � 2012 Peeter Joot.  All Rights Reserved.
% Licenced as described in the file LICENSE under the root directory of this GIT repository.
%
%
%
%
%\input{../peeter_prologue_print.tex}
%\input{../peeter_prologue_widescreen.tex}
%
%\chapter{PHY456H1F: Quantum Mechanics II.  Lecture 21 (Taught by Prof J.E. Sipe).  Scattering theory}
\index{scattering}
%\chapter{Scattering theory}
\label{chap:qmTwoL21}
%
%\blogpage{http://sites.google.com/site/peeterjoot2/math2011/qmTwoL21.pdf}
%\date{Nov 23, 2011}
%
\section{Setup.}
%
READING: \S 19, \S 20 of the text \citep{desai2009quantum}.

\Cref{fig:qmTwoL21:qmTwoL21Fig1} shows a simple classical picture of a two particle scattering collision
%
\imageFigure{../figures/phy456-qmII/qmTwoL21Fig1}{classical collision of particles.}{fig:qmTwoL21:qmTwoL21Fig1}{0.2}
%
We will focus on point particle elastic collisions (no energy lost in the collision).  With particles of mass \(m_1\) and \(m_2\) we write for the total and reduced mass respectively
%
\begin{equation}\label{eqn:qmTwoL21:10}
M = m_1 + m_2
\end{equation}
%
\begin{equation}\label{eqn:qmTwoL21:30}
\inv{\mu} = \inv{m_1} + \inv{m_2},
\end{equation}
%
so that interaction due to a potential \(V(\Br_1 - \Br_2)\) that depends on the difference in position \(\Br = \Br_1 - \Br\) has, in the center of mass frame, the Hamiltonian
%
\begin{equation}\label{eqn:qmTwoL21:50}
H = \frac{\Bp^2}{2 \mu} + V(\Br)
\end{equation}
%
In the classical picture we would investigate the scattering radius \(r_0\) associated with the impact parameter \(\rho\) as depicted in \cref{fig:qmTwoL21:qmTwoL21Fig2}.
\imageFigure{../figures/phy456-qmII/qmTwoL21Fig2}{Classical scattering radius and impact parameter.}{fig:qmTwoL21:qmTwoL21Fig2}{0.2}
\section{1D QM scattering.  No potential wave packet time evolution.}
\index{time evolution}

Now lets move to the QM picture where we assume that we have a particle that can be represented as a wave packet as in \cref{fig:qmTwoL21:qmTwoL21Fig3}.
\imageFigure{../figures/phy456-qmII/qmTwoL21Fig3}{Wave packet for a particle wavefunction \(\Re(\psi(x,0))\).}{fig:qmTwoL21:qmTwoL21Fig3}{0.2}
First without any potential \(V(x) = 0\), lets consider the evolution.  Our position and momentum space representations are related by
%
\begin{equation}\label{eqn:qmTwoL21:70}
\int \Abs{\psi(x, t)}^2 dx = 1 = \int \Abs{\psi(p, t)}^2 dp,
\end{equation}
%
and by Fourier transform
%
\begin{equation}\label{eqn:qmTwoL21:90}
\psi(x, t) = \int \frac{dp}{\sqrt{2 \pi \Hbar}} \overline{\psi}(p, t) e^{i p x/\Hbar}.
\end{equation}
%
Schr\"{o}dinger's equation takes the form
%
\begin{equation}\label{eqn:qmTwoL21:110}
i \Hbar \PD{t}{\psi(x,t)} = - \frac{\Hbar^2}{2 \mu} \PDSq{x}{\psi(x, t)},
\end{equation}
%
or more simply in momentum space
%
\begin{equation}\label{eqn:qmTwoL21:130}
i \Hbar \PD{t}{\overline{\psi}(p,t)} = \frac{p^2}{2 \mu} \PDSq{x}{\overline{\psi}(p, t)}.
\end{equation}
%
Rearranging to integrate we have
%
\begin{equation}\label{eqn:qmTwoL21:150}
\PD{t}{\overline{\psi}} = -\frac{i p^2}{2 \mu \Hbar} \overline{\psi},
\end{equation}
%
and integrating
%
\begin{equation}\label{eqn:qmTwoL21:170}
\ln \overline{\psi} = -\frac{i p^2 t}{2 \mu \Hbar} + \ln C,
\end{equation}
%
or
\begin{equation}\label{eqn:qmTwoL21:190}
\overline{\psi} = C e^{-\frac{i p^2 t}{2 \mu \Hbar}} = \overline{\psi}(p, 0) e^{-\frac{i p^2 t}{2 \mu \Hbar}}.
\end{equation}
%
Time evolution in momentum space for the free particle changes only the phase of the wavefunction, the momentum probability density of that particle.

Fourier transforming, we find our position space wavefunction to be
%
\begin{equation}\label{eqn:qmTwoL21:210}
\psi(x, t) = \int \frac{dp}{\sqrt{2 \pi \Hbar}} \overline{\psi}(p, 0) e^{i p x/\Hbar} e^{-i p^2 t/2 \mu \Hbar}.
\end{equation}
%
To clean things up, write
%
\begin{equation}\label{eqn:qmTwoL21:230}
p = \Hbar k,
\end{equation}
%
for
%
\begin{equation}\label{eqn:qmTwoL21:250}
\psi(x, t) = \int \frac{dk}{\sqrt{2 \pi}} a(k, 0) ) e^{i k x} e^{-i \Hbar k^2 t/2 \mu},
\end{equation}
%
where
\begin{equation}\label{eqn:qmTwoL21:270}
a(k, 0) = \sqrt{\Hbar} \overline{\psi}(p, 0).
\end{equation}
%
Putting
%
\begin{equation}\label{eqn:qmTwoL21:290}
a(k, t) = a(k, 0) e^{ -i \Hbar k^2/2 \mu},
\end{equation}
%
we have
%
\begin{equation}\label{eqn:qmTwoL21:310}
\psi(x, t) = \int \frac{dk}{\sqrt{2 \pi}} a(k, t) ) e^{i k x}
\end{equation}
%
Observe that we have
%
\begin{equation}\label{eqn:qmTwoL21:330}
\int dk \Abs{ a(k, t)}^2 = \int dp \Abs{ \overline{\psi}(p, t)}^2 = 1.
\end{equation}
%
\section{A Gaussian wave packet.}
\index{Gaussian wave packet}
Suppose that we have, as depicted in \cref{fig:qmTwoL21:qmTwoL21Fig4},
\imageFigure{../figures/phy456-qmII/qmTwoL21Fig4}{Gaussian wave packet.}{fig:qmTwoL21:qmTwoL21Fig4}{0.2}
a Gaussian wave packet of the form
%
\begin{equation}\label{eqn:qmTwoL21:350}
\psi(x, 0) = \frac{ (\pi \Delta^2)^{1/4}} e^{i k_0 x} e^{- x^2/2 \Delta^2}.
\end{equation}
%
This is actually a minimum uncertainty packet with
%
\begin{equation}\label{eqn:qmTwoL21:370}
\begin{aligned}
\Delta x &= \frac{\Delta}{\sqrt{2}} \\
\Delta p &= \frac{\Hbar}{\Delta \sqrt{2}}.
\end{aligned}
\end{equation}
%
Taking Fourier transforms we have
%
\begin{equation}\label{eqn:qmTwoL21:390}
\begin{aligned}
a(k, 0) &= \left(\frac{\Delta^2}{\pi}\right)^{1/4} e^{-(k - k_0)^2 \Delta^2/2} \\
a(k, t) &= \left(\frac{\Delta^2}{\pi}\right)^{1/4} e^{-(k - k_0)^2 \Delta^2/2} e^{ -i \Hbar k^2 t/ 2\mu} \equiv \alpha(k, t)
\end{aligned}
\end{equation}
%
For \(t > 0\) our wave packet will start moving and spreading as in \cref{fig:qmTwoL21:qmTwoL21Fig5}.
\imageFigure{../figures/phy456-qmII/qmTwoL21Fig5}{moving spreading Gaussian packet.}{fig:qmTwoL21:qmTwoL21Fig5}{0.2}
\section{With a potential.}
%
Now ``switch on'' a potential, still assuming a wave packet representation for the particle.  With a positive (repulsive) potential as in \cref{fig:qmTwoL21:qmTwoL21Fig6}, at a time long before the interaction of the wave packet with the potential we can visualize the packet as heading towards the barrier.
%
\imageFigure{../figures/phy456-qmII/qmTwoL21Fig6}{QM wave packet prior to interaction with repulsive potential.}{fig:qmTwoL21:qmTwoL21Fig6}{0.2}
%
After some time long after the interaction, classically for this sort of potential where the particle kinetic energy is less than the barrier ``height'', we would have total reflection.  In the QM case, we have seen before that we will have a reflected and a transmitted portion of the wave packet as depicted in \cref{fig:qmTwoL21:qmTwoL21Fig7}.
\imageFigure{../figures/phy456-qmII/qmTwoL21Fig7}{QM wave packet long after interaction with repulsive potential.}{fig:qmTwoL21:qmTwoL21Fig7}{0.15}
Even if the particle kinetic energy is greater than the barrier height, 
or for a negative potential (both such potentials are depicted in \cref{fig:qmTwoL21:qmTwoL21Fig8}), then
we can still have a reflected component.
\imageFigure{../figures/phy456-qmII/kineticEnergyGreaterThanPotentialFig9}{Kinetic energy greater than two potential energy distributions.}{fig:qmTwoL21:qmTwoL21Fig8}{0.15}
%\imageFigure{../figures/phy456-qmII/qmTwoL21Fig8}{Kinetic energy greater than potential energy.}{fig:qmTwoL21:qmTwoL21Fig8}{0.15}
%\imageFigure{../figures/phy456-qmII/qmTwoL21Fig9}{qmTwoL21Fig9.}{fig:qmTwoL21:qmTwoL21Fig9}{0.2}
%
Consider the probability for the particle to be found anywhere long after the interaction, summing over the transmitted and reflected wave functions, we have
%
\begin{equation}\label{eqn:qmTwoL21:630}
\begin{aligned}
1
&= \int \Abs{\psi_r + \psi_t}^2 \\
&= \int \Abs{\psi_r}^2  + \int \Abs{\psi_t}^2 + 2 \Re \int \psi_r^\conj \psi_t
\end{aligned}
\end{equation}
%
Observe that long after the interaction the cross terms in the probabilities will vanish because they are non-overlapping, leaving just the probably densities for the transmitted and reflected probably densities independently.

We define
%
\begin{equation}\label{eqn:qmTwoL21:410}
\begin{aligned}
T &= \int \Abs{\psi_t(x, t)}^2 dx \\
R &= \int \Abs{\psi_r(x, t)}^2 dx.
\end{aligned}
\end{equation}
%
The objective of most of our scattering problems will be the calculation of these probabilities and the comparisons of their ratios.
%
\paragraph{Question.}  Can we have more than one wave packet reflect off.  Yes, we could have multiple wave packets for both the reflected and the transmitted portions.  For example, if the potential has some internal structure there could be internal reflections before anything emerges on either side and things could get quite messy.
%
\section{Considering the time independent case temporarily.}
%
We are going to work through something that is going to seem at first to be completely unrelated.  We will (eventually) see that this can be applied to this problem, so a bit of patience will be required.

We will be using the time independent Schr\"{o}dinger equation
%
\begin{equation}\label{eqn:qmTwoL21:430}
- \frac{\Hbar^2}{2 \mu} \psi_k''(x) = V(x) \psi_k(x) = E \psi_k(x),
\end{equation}
%
where we have added a subscript \(k\) to our wave function with the intention (later) of allowing this to vary.  For ``future use'' we define for \(k > 0\)
%
\begin{equation}\label{eqn:qmTwoL21:450}
E = \frac{\Hbar^2 k^2}{2 \mu}.
\end{equation}
%
Consider a potential as in \cref{fig:qmTwoL21:qmTwoL21Fig10}, where \(V(x) = 0\) for \(x > x_2\) and \(x < x_1\).
%
\imageFigure{../figures/phy456-qmII/qmTwoL21Fig10}{potential zero outside of a specific region.}{fig:qmTwoL21:qmTwoL21Fig10}{0.15}
%
We will not have bound states here (repulsive potential).  There will be many possible solutions, but we want to look for a solution that is of the form
%
\begin{equation}\label{eqn:qmTwoL21:470}
\psi_k(x) = C e^{i k x}, \qquad x > x_2
\end{equation}
%
Suppose \(x = x_3 > x_2\), we have
%
\begin{equation}\label{eqn:qmTwoL21:490}
\psi_k(x_3) = C e^{i k x_3}
\end{equation}
%
\begin{equation}\label{eqn:qmTwoL21:510}
\evalbar{\frac{d\psi_k}{dx}}{x = x_3} = i k C e^{i k x_3} \equiv \phi_k(x_3)
\end{equation}
%
\begin{equation}\label{eqn:qmTwoL21:530}
\evalbar{\frac{d^2\psi_k}{dx^2}}{x = x_3} = -k^2 C e^{i k x_3}
\end{equation}
%
Defining
\begin{equation}\label{eqn:qmTwoL21:550}
\phi_k(x) = \frac{d\psi_k}{dx},
\end{equation}
%
we write Schr\"{o}dinger's equation as a pair of coupled first order equations
%
\begin{equation}\label{eqn:qmTwoL21:570}
\begin{aligned}
\frac{d\psi_k}{dx} &= \phi_k(x) \\
-\frac{\Hbar^2}{2 \mu} \frac{d\phi_k(x)}{dx} = - V(x) \psi_k(x) + \frac{\Hbar^2 k^2}{2\mu} \psi_k(x).
\end{aligned}
\end{equation}
%
At this \(x = x_3\) specifically, we ``know'' both \(\phi_k(x_3)\) and \(\psi_k(x_3)\) and have
%
\begin{equation}\label{eqn:qmTwoL21:590}
\begin{aligned}
\evalbar{\frac{d\psi_k}{dx}}{x_3} &= \phi_k(x) \\
-\frac{\Hbar^2}{2 \mu} \evalbar{\frac{d\phi_k(x)}{dx}}{x_3} =
- V(x_3) \psi_k(x_3) + \frac{\Hbar^2 k^2}{2\mu} \psi_k(x_3),
\end{aligned}
\end{equation}
%
This allows us to find both
%
\begin{equation}\label{eqn:qmTwoL21:610}
\begin{aligned}
&\evalbar{\frac{d\psi_k}{dx}}{x_3} \\
&\evalbar{\frac{d\phi_k(x)}{dx}}{x_3}
\end{aligned}
\end{equation}
%
then proceed to numerically calculate \(\phi_k(x)\) and \(\psi_k(x)\) at neighbouring points \(x = x_3 + \epsilon\).  Essentially, this allows us to numerically integrate backwards from \(x_3\) to find the wave function at previous points for any sort of potential.



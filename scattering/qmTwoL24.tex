%
% Copyright � 2012 Peeter Joot.  All Rights Reserved.
% Licenced as described in the file LICENSE under the root directory of this GIT repository.
%
%
%There will be drop in hours this week.  I will hold the usual office hour
%Friday: 4-5.
%
%Exam: Dec 12.
%Extra office hours:
%Larkin Building: Rm. 212
%(Devonshire place)
%
%Sat 9-5
%Sun 9-5
%
%There will be a lecture on Wed.
%
%\chapter{PHY456H1F: Quantum Mechanics II.  Lecture L24 (Taught by Prof J.E. Sipe).  3D Scattering cross sections (cont.)}
%\chapter{3D Scattering cross sections (cont.)}
\index{3D scattering}
\index{scattering cross sections}
\label{chap:qmTwoL24}
%
%\blogpage{http://sites.google.com/site/peeterjoot2/math2011/qmTwoL24.pdf}
%\date{Dec 5, 2011}
%
\section{Scattering cross sections.}
%
READING: \S 20 \citep{desai2009quantum}

Recall that we are studing the case of a potential that is zero outside of a fixed bound, \(V(\Br) = 0\) for \(r > r_0\), as in \cref{fig:qmTwoL24:qmTwoL22fig5},
\imageFigure{../figures/phy456-qmII/qmTwoL22fig5}{Bounded potential.}{fig:qmTwoL24:qmTwoL22fig5}{0.2}
and were looking for solutions to Schr\"{o}dinger's equation
%
\begin{equation}\label{eqn:qmTwoL24:10}
-\frac{\Hbar^2}{2\mu} \spacegrad^2
\psi_\Bk(\Br)
+ V(\Br)
\psi_\Bk(\Br)
=
\frac{\Hbar^2 \Bk^2}{2 \mu}
\psi_\Bk(\Br),
\end{equation}
%
in regions of space, where \(r > r_0\) is very large.  We found
%
\begin{equation}\label{eqn:qmTwoL24:30}
\psi_\Bk(\Br) \sim e^{i \Bk \cdot \Br} + \frac{e^{i k r}}{r} f_\Bk(\theta, \phi).
\end{equation}
%
For \(r \le r_0\) this will be something much more complicated.

To study scattering we will use the concept of probability flux as in electromagnetism
%
\begin{equation}\label{eqn:qmTwoL24:50}
\spacegrad \cdot \Bj + \dot{\rho} = 0
\end{equation}
%
Using
%
\begin{equation}\label{eqn:qmTwoL24:70}
\psi(\Br, t) =
\psi_\Bk(\Br)^\conj
\psi_\Bk(\Br)
\end{equation}
%
we find
%
\begin{equation}\label{eqn:qmTwoL24:90}
\Bj(\Br, t) = \frac{\Hbar}{2 \mu i} \Bigl(
\psi_\Bk(\Br)^\conj \spacegrad \psi_\Bk(\Br)
- (\spacegrad \psi_\Bk^\conj(\Br)) \psi_\Bk(\Br)
\Bigr)
\end{equation}
%
when
%
\begin{equation}\label{eqn:qmTwoL24:110}
-\frac{\Hbar^2}{2\mu} \spacegrad^2
\psi_\Bk(\Br)
+ V(\Br)
\psi_\Bk(\Br)
=
i \Hbar \PD{t}{
\psi_\Bk(\Br)
}
\end{equation}
%
In a fashion similar to what we did in the 1D case, let us suppose that we can write our wave function
%
\begin{equation}\label{eqn:qmTwoL24:130}
\psi(\Br, t_{\text{initial}}) = \int d^3k \alpha(\Bk, t_{\text{initial}}) \psi_\Bk(\Br)
\end{equation}
%
and treat the scattering as the scattering of a plane wave front (idealizing a set of wave packets) off of the object of interest as depicted in \cref{fig:qmTwoL24:qmTwoL24fig3}.
\imageFigure{../figures/phy456-qmII/qmTwoL24fig3}{plane wave front incident on particle.}{fig:qmTwoL24:qmTwoL24fig3}{0.2}
We assume that our incoming particles are sufficiently localized in \(k\) space as depicted in the idealized representation of \cref{fig:qmTwoL24:qmTwoL24fig4}.
\imageFigure{../figures/phy456-qmII/qmTwoL24fig4}{k space localized wave packet.}{fig:qmTwoL24:qmTwoL24fig4}{0.2}
We assume that \(\alpha(\Bk, t_{\text{initial}})\) is localized.
%
\begin{equation}\label{eqn:qmTwoL24:150}
\psi(\Br, t_{\text{initial}}) =
\int d^3k
\left(
\alpha(\Bk, t_{\text{initial}})
e^{i k_z z}
+
\alpha(\Bk, t_{\text{initial}}) \frac{e^{i k r}}{r} f_\Bk(\theta, \phi)
\right)
\end{equation}
%
We suppose that
%
\begin{equation}\label{eqn:qmTwoL24:170}
\alpha(\Bk, t_{\text{initial}}) = \alpha(\Bk) e^{-i \Hbar k^2 t_{\text{initial}}/ 2\mu}
\end{equation}
%
where this is chosen (\(\alpha(\Bk, t_{\text{initial}})\) is built in this fashion) so that this is non-zero for \(z\) large in magnitude and negative.

This last integral can be approximated
\begin{equation}\label{eqn:qmTwoL24:190}
\begin{aligned}
\int d^3k
\alpha(\Bk, t_{\text{initial}}) \frac{e^{i k r}}{r} f_\Bk(\theta, \phi)
&\approx
\frac{f_{\Bk_0}(\theta, \phi)}{r}
\int d^3k
\alpha(\Bk, t_{\text{initial}}) e^{i k r} \\
&\rightarrow 0
\end{aligned}
\end{equation}
%
This is very much like the 1D case where we found no reflected component for our initial time.

We will normally look in a locality well away from the wave front as indicted in \cref{fig:qmTwoL24:qmTwoL24fig5}.
\imageFigure{../figures/phy456-qmII/qmTwoL24fig5}{point of measurement of scattering cross section.}{fig:qmTwoL24:qmTwoL24fig5}{0.2}
There are situations where we do look in the locality of the wave front that has been scattered.
%
\paragraph{Incoming wave}
Our income wave is of the form
%
\begin{equation}\label{eqn:qmTwoL24:210}
\psi_i = A e^{i k z} e^{-i \Hbar k^2 t/2 \mu}
\end{equation}
%
Here we have made the approximation that \(k = \Abs{\Bk} \sim k_z\).  We can calculate the probability current
%
\begin{equation}\label{eqn:qmTwoL24:230}
\Bj = \zcap \frac{\Hbar k}{\mu} A
\end{equation}
%
(notice the \(v = p/m\) like term above, with \(p = \Hbar k\)).

For the scattered wave (dropping \(A\) factor)
%
\begin{equation}\label{eqn:qmTwoL24:390}
\begin{aligned}
\Bj &=
\frac{\Hbar}{2 \mu i}
\left(
f_\Bk^\conj(\theta, \phi) \frac{e^{-i k r}}{r} \spacegrad \left(
f_\Bk(\theta, \phi) \frac{e^{i k r}}{r}
\right)
-
\spacegrad \left(
f_\Bk^\conj(\theta, \phi) \frac{e^{-i k r}}{r}
\right)
f_\Bk(\theta, \phi) \frac{e^{i k r}}{r}
\right)
\\
&\approx
\frac{\Hbar}{2 \mu i}
\left(
f_\Bk^\conj(\theta, \phi) \frac{e^{-i k r}}{r} i k \rcap f_\Bk(\theta, \phi)
\frac{e^{i k r}}{r}
-
f_\Bk^\conj(\theta, \phi) \frac{e^{-i k r}}{r} (-i k \rcap) f_\Bk(\theta, \phi)
\frac{e^{i k r}}{r}
\right)
\end{aligned}
\end{equation}
%
We find that the radial portion of the current density is
%
\begin{equation}\label{eqn:qmTwoL24:410}
\begin{aligned}
\rcap \cdot \Bj
&= \frac{\Hbar}{2 \mu i} \Abs{f}^2 \frac{ 2 i k }{r^2} \\
&= \frac{\Hbar k}{\mu} \inv{r^2} \Abs{f}^2,
\end{aligned}
\end{equation}
%
and the flux through our element of solid angle is
%
\begin{equation}\label{eqn:qmTwoL24:430}
\begin{aligned}
\rcap dA \cdot \Bj
&=
\frac{\text{probability}}{\text{unit area per time}} \times \text{area}  \\
&= \frac{\text{probability}}{\text{unit time}} \\
&=
\frac{\Hbar k}{\mu} \frac{\Abs{f_\Bk(\theta, \phi)}^2}{r^2} r^2 d\Omega \\
&=
\frac{\Hbar k }{\mu}
\Abs{f_\Bk(\theta, \phi)}^2 d\Omega \\
&=
j_{\text{incoming}}
\mathLabelBox{\Abs{f_\Bk(\theta, \phi)}^2}{\(d\sigma/d\Omega\)} d\Omega.
\end{aligned}
\end{equation}
%
We identify the scattering cross section above
%
\begin{equation}\label{eqn:qmTwoL24:250}
\frac{d\sigma}{d\Omega}
=
\Abs{f_\Bk(\theta, \phi)}^2
\end{equation}
%
\begin{equation}\label{eqn:qmTwoL24:270}
\sigma = \int \Abs{f_\Bk(\theta, \phi)}^2 d\Omega
\end{equation}
%
We have been somewhat unrealistic here since we have used a plane wave approximation, and can as in \cref{fig:qmTwoL24:qmTwoL24fig6},
\imageFigure{../figures/phy456-qmII/qmTwoL24fig6}{Plane wave vs packet wave front.}{fig:qmTwoL24:qmTwoL24fig6}{0.2}
will actually produce the same answer.  For details we are referred to \citep{messiah1999quantum} and \citep{taylor1972scattering}.
\paragraph{Working towards a solution}
We have done a bunch of stuff here but are not much closer to a real solution because we do not actually know what \(f_\Bk\) is.

Let us write Schr\"{o}dinger
%
\begin{equation}\label{eqn:qmTwoL24:290}
-\frac{\Hbar^2}{2\mu} \spacegrad^2
\psi_\Bk(\Br)
+ V(\Br)
\psi_\Bk(\Br)
=
\frac{\Hbar^2 \Bk^2}{2 \mu}
\psi_\Bk(\Br),
\end{equation}
%
instead as
%
\begin{equation}\label{eqn:qmTwoL24:310}
(\spacegrad^2 + \Bk^2)
\psi_\Bk(\Br)
= s(\Br)
\end{equation}
%
where
%
\begin{equation}\label{eqn:qmTwoL24:330}
s(\Br) = \frac{2\mu}{\Hbar^2} V(\Br) \psi_\Bk(\Br)
\end{equation}
%
where \(s(\Br)\) is really the particular solution to this differential problem.   We want
%
\begin{equation}\label{eqn:qmTwoL24:350}
\psi_\Bk(\Br) =
\psi_\Bk^{\text{homogeneous}}(\Br)
+ \psi_\Bk^{\text{particular}}(\Br)
\end{equation}
%
and
%
\begin{equation}\label{eqn:qmTwoL24:370}
\psi_\Bk^{\text{homogeneous}}(\Br) = e^{i \Bk \cdot \Br}
\end{equation}
%


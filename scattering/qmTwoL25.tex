%
% Copyright � 2012 Peeter Joot.  All Rights Reserved.
% Licenced as described in the file LICENSE under the root directory of this GIT repository.
%
%
%
%
%\input{../peeter_prologue_print.tex}
%\input{../peeter_prologue_widescreen.tex}
%
% exam:
% 6 questions, one from roblem set
% 1 essay derivation question with choice
% 4 regular
% 1/2 hr each question.
%
%\chapter{PHY456H1F: Quantum Mechanics II.  Lecture 25 (Taught by Prof J.E. Sipe).  Born approximation}
%\chapter{Born approximation}
\index{Born approximation}
\label{chap:qmTwoL25}
%\blogpage{http://sites.google.com/site/peeterjoot2/math2011/qmTwoL25.pdf}
%\date{Dec 7, 2011}
%
%\section{Born approximation}
%
READING: \S 20 \citep{desai2009quantum}

We have been arguing that we can write the stationary equation
%
\begin{equation}\label{eqn:qmTwoL25:10}
\left( \spacegrad^2 + \Bk^2\right) \psi_\Bk(\Br) = s(\Br)
\end{equation}
%
with
%
\begin{equation}\label{eqn:qmTwoL25:30}
s(\Br) = \frac{2\mu}{\Hbar^2} V(\Br) \psi_\Bk(\Br)
\end{equation}
%
\begin{equation}\label{eqn:qmTwoL25:50}
\psi_\Bk(\Br) = \psi_\Bk^{\text{homogeneous}}(\Br) + \psi_\Bk^{\text{particular}}(\Br)
\end{equation}
%
Introduce Green function
%
\begin{equation}\label{eqn:qmTwoL25:70}
\left( \spacegrad^2 + \Bk^2\right) G^0(\Br, \Br') = \delta(\Br- \Br')
\end{equation}
%
Suppose that I can find \(G^0(\Br, \Br')\), then
%
\begin{equation}\label{eqn:qmTwoL25:90}
\psi_\Bk^{\text{particular}}(\Br) = \int G^0(\Br, \Br') s(\Br') d^3 \Br'
\end{equation}
%
It turns out that finding the Green's function \(G^0(\Br, \Br')\) is not so hard.  Note the following, for \(k = 0\), we have
%
\begin{equation}\label{eqn:qmTwoL25:110}
\spacegrad^2 G^0_0(\Br, \Br') = \delta(\Br - \Br')
\end{equation}
%
(where a zero subscript is used to mark the \(k = 0\) case).  We know this Green's function from electrostatics, and conclude that
%
\begin{equation}\label{eqn:qmTwoL25:130}
G^0_0(\Br, \Br') = - \inv{4 \pi} \inv{\Abs{\Br - \Br'}}
\end{equation}
%
For \(\Br \ne \Br'\) we can easily show that
%
\begin{equation}\label{eqn:qmTwoL25:150}
G^0(\Br, \Br') = - \inv{4 \pi} \frac{e^{i k\Abs{\Br - \Br'}}}{\Abs{\Br - \Br'}}
\end{equation}
%
This is correct for all \(\Br\) because it also gives the right limit as \(\Br \rightarrow \Br'\).  This argument was first given by Lorentz.  An outline for a derivation, utilizing the usual Fourier transform and contour integration arguments for these Green's derivations, can be found in \S 7.4 of \citep{byron1992mca}.  A direct verification, not quite as easy as claimed can be found in \ref{chap:helmholtzGreens}.

We can now write our particular solution
%
\begin{equation}\label{eqn:qmTwoL25:170}
\psi_\Bk(\Br)
= e^{i \Bk \cdot \Br}
- \inv{4 \pi} \int \frac{e^{i k\Abs{\Br - \Br'}}}{\Abs{\Br - \Br'}} s(\Br') d^3 \Br'
\end{equation}
%
This is of no immediate help since we do not know \(\psi_\Bk(\Br)\) and that is embedded in \(s(\Br)\).
%
\begin{equation}\label{eqn:qmTwoL25:190}
\psi_\Bk(\Br)
= e^{i \Bk \cdot \Br}
- \frac{2 \mu}{4 \pi \Hbar^2} \int \frac{e^{i k\Abs{\Br - \Br'}}}{\Abs{\Br - \Br'}} V(\Br') \psi_\Bk(\Br') d^3 \Br'
\end{equation}
%
Now look at this for \(\Br \gg \Br'\)
%
\begin{equation}\label{eqn:qmTwoL25:310}
\begin{aligned}
\Abs{\Br - \Br'}
&=
\left(
\Br^2 + (\Br')^2 - 2 \Br \cdot \Br'
\right)^{1/2} \\
&=
r
\left(
1 + \frac{(\Br')^2}{\Br^2} - 2 \inv{\Br^2} \Br \cdot \Br'
\right)^{1/2} \\
&=
r
\left(
1 - \inv{2} \frac{2}{\Br^2} \Br \cdot \Br'
+ O\left(\frac{r'}{r}\right)^2
\right)^{1/2} \\
&=
r
- \rcap \cdot \Br'
+ O\left(\frac{{r'}^2}{r}\right)
\end{aligned}
\end{equation}
%
We get
%
\begin{equation}\label{eqn:qmTwoL25:210}
\begin{aligned}
\psi_\Bk(\Br)
&\rightarrow e^{i \Bk \cdot \Br} - \frac{2 \mu}{4 \pi \Hbar^2} \frac{ e^{i k r}}{r} \int e^{-i k \rcap \cdot \Br'} V(\Br') \psi_\Bk(\Br') d^3 \Br' \\
&=
e^{i \Bk \cdot \Br} + f_\Bk(\theta, \phi) \frac{ e^{i k r}}{r},
\end{aligned}
\end{equation}
%
where
%
\begin{equation}\label{eqn:qmTwoL25:230}
f_\Bk(\theta, \phi) =
- \frac{\mu}{2 \pi \Hbar^2} \int e^{-i k \rcap \cdot \Br'} V(\Br') \psi_\Bk(\Br') d^3 \Br'
\end{equation}
%
If the scattering is weak we have the \textunderline{Born approximation}
%
\begin{equation}\label{eqn:qmTwoL25:250}
f_\Bk(\theta, \phi) =
- \frac{\mu}{2 \pi \Hbar^2} \int e^{-i k \rcap \cdot \Br'} V(\Br') e^{i \Bk \cdot \Br'} d^3 \Br',
\end{equation}
%
or
%
\begin{equation}\label{eqn:qmTwoL25:270}
\psi_\Bk(\Br) =
e^{i \Bk \cdot \Br} - \frac{\mu}{2 \pi \Hbar^2} \frac{ e^{i k r}}{r} \int e^{-i k \rcap \cdot \Br'} V(\Br') e^{i \Bk \cdot \Br'} d^3 \Br'.
\end{equation}
%
Should we wish to make a further approximation, we can take the wave function resulting from application of the Born approximation, and use that a second time.  This gives us the ``Born again'' approximation of
%
\begin{equation}\label{eqn:qmTwoL25:290}
\begin{aligned}
\psi_\Bk(\Br)
&=
e^{i \Bk \cdot \Br} - \frac{\mu}{2 \pi \Hbar^2} \frac{ e^{i k r}}{r} \int e^{-i k \rcap \cdot \Br'} V(\Br')
\, \times \\ &\quad
\lr{
   e^{i \Bk \cdot \Br'} - \frac{\mu}{2 \pi \Hbar^2} \frac{ e^{i k r'}}{r'} \int e^{-i k \rcap' \cdot \Br''} V(\Br'') e^{i \Bk \cdot \Br''} d^3 \Br''
}
d^3 \Br' \\
&=
e^{i \Bk \cdot \Br} - \frac{\mu}{2 \pi \Hbar^2} \frac{ e^{i k r}}{r} \int e^{-i k \rcap \cdot \Br'} V(\Br') e^{i \Bk \cdot \Br'} d^3 \Br' \\
&\quad +\frac{\mu^2}{(2 \pi)^2 \Hbar^4}
\frac{ e^{i k r}}{r}
\int e^{-i k \rcap \cdot \Br'} V(\Br')
\frac{ e^{i k r'}}{r'} \int e^{-i k \rcap' \cdot \Br''} V(\Br'') e^{i \Bk \cdot \Br''} d^3 \Br'' d^3 \Br'.
\end{aligned}
\end{equation}
%


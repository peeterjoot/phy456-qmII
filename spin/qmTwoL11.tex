%
% Copyright � 2012 Peeter Joot.  All Rights Reserved.
% Licenced as described in the file LICENSE under the root directory of this GIT repository.
%
%
%
%
%\input{../peeter_prologue_print.tex}
%\input{../peeter_prologue_widescreen.tex}
%
%\chapter{PHY456H1F: Quantum Mechanics II.  Lecture 11 (Taught by Prof J.E. Sipe).  Spin and Spinors}
%\chapter{Spin and Spinors}
\index{spin}
\index{spinors}
\label{chap:qmTwoL11}
%
%\blogpage{http://sites.google.com/site/peeterjoot/math2011/qmTwoL11.pdf}
%\date{Oct 17, 2011}
%
%
%
%
%
\section{Generators}
\index{generator}
%
Covered in \S 26 of the text \citep{desai2009quantum}.

\makeexample{Time translation}{example:qmTwoL11:1}{
%
\begin{equation}\label{eqn:qmTwoL11:10}
\ket{\psi(t)} = e^{-i H t/\Hbar} \ket{\psi(0)} .
\end{equation}
%
The Hamiltonian ``generates'' evolution (or translation) in time.
}

\makeexample{Spatial translation}{example:qmTwoL11:2}{
%
\begin{equation}\label{eqn:qmTwoL11:30}
\ket{\Br + \Ba} =
e^{-i \Ba \cdot \BP/\Hbar}
\ket{\Br}.
\end{equation}
%
\imageFigure{../figures/phy456-qmII/qmTwoL11fig1}{Vector translation}{fig:qmTwoL11:qmTwoL11fig1}{0.2}
%\cref{fig:qmTwoL11:qmTwoL11fig1}

\(\BP\) is the operator that generates translations.  Written out, we have
%
\begin{equation}\label{eqn:qmTwoL11:50}
\begin{aligned}
e^{-i \Ba \cdot \BP/\Hbar}
&= e^{- i (a_x P_x + a_y P_y + a_z P_z)/\Hbar} \\
&= e^{- i a_x P_x/\Hbar}
e^{- i a_y P_y/\Hbar}
e^{- i a_z P_z/\Hbar},
\end{aligned}
\end{equation}
%
where the factorization was possible because \(P_x\), \(P_y\), and \(P_z\) commute
%
\begin{equation}\label{eqn:qmTwoL11:70}
\antisymmetric{P_i}{P_j} = 0,
\end{equation}
%
for any \(i, j\) (including \(i = i\) as I dumbly questioned in class ... this is a  commutator, so \(\antisymmetric{P_i}{P_j} = P_i P_i - P_i P_i = 0\)).

The fact that the \(P_i\) commute means that successive translations can be done in any order and have the same result.

In class we were rewarded with a graphic demo of translation component commutation as Professor Sipe pulled a giant wood carving of a cat (or tiger?) out from beside the desk and proceeded to translate it around on the desk in two different orders, with the cat ending up in the same place each time.
%
\paragraph{Exponential commutation}
%
Note that in general
%
\begin{equation}\label{eqn:qmTwoL11:90}
e^{A + B} \ne e^A e^B,
\end{equation}
%
unless \(\antisymmetric{A}{B} = 0\).  To show this one can compare
%
\begin{equation}\label{eqn:qmTwoL11:110}
\begin{aligned}
e^{A + B}
&= 1 + A + B + \inv{2}(A + B)^2 + \cdots \\
&= 1 + A + B + \inv{2}(A^2 + A B + BA + B^2) + \cdots \\
\end{aligned}
\end{equation}
%
and
\begin{equation}\label{eqn:qmTwoL11:130}
\begin{aligned}
e^A e^B
&=
\left(1 + A + \inv{2}A^2 + \cdots\right)
\left(1 + B + \inv{2}B^2 + \cdots\right) \\
&= 1 + A + B + \inv{2}( A^2 + 2 A B + B^2 ) + \cdots
\end{aligned}
\end{equation}
%
Comparing the second order (for example) we see that we must have for equality
%
\begin{equation}\label{eqn:qmTwoL11:150}
A B + B A = 2 A B,
\end{equation}
%
or
%
\begin{equation}\label{eqn:qmTwoL11:170}
B A = A B,
\end{equation}
%
or
\begin{equation}\label{eqn:qmTwoL11:190}
\antisymmetric{A}{B} = 0
\end{equation}
%
\paragraph{Translating a ket}
%
If we consider the quantity
%
\begin{equation}\label{eqn:qmTwoL11:210}
e^{-i \Ba \cdot \BP/\Hbar}
\ket{\psi} = \ket{\psi'} ,
\end{equation}
%
does this ket ``translated'' by \(\Ba\) make any sense?  The vector \(\Ba\) lives in a 3D space and our ket \(\ket{\psi}\) lives in Hilbert space.  A quantity like this deserves some careful thought and is the subject of some such thought in the Interpretations of Quantum mechanics course.  For now, we can think of the operator and ket as a ``gadget'' that prepares a state.

A student in class pointed out that \(\ket{\psi}\) can be dependent on many degrees of freedom, for example, the positions of eight different particles.  This translation gadget in such a case acts on the whole kit and caboodle.

Now consider the matrix element
%
\begin{equation}\label{eqn:qmTwoL11:230}
\braket{\Br}{\psi'}
= \bra{\Br} e^{-i \Ba \cdot \BP/\Hbar} \ket{\psi}.
\end{equation}
%
Note that
%
\begin{equation}\label{eqn:qmTwoL11:670}
\begin{aligned}
\bra{\Br} e^{-i \Ba \cdot \BP/\Hbar}
&=
\left( e^{i \Ba \cdot \BP/\Hbar}
\ket{\Br} \right)^\dagger \\
&=
\left( \ket{\Br - \Ba} \right)^\dagger,
\end{aligned}
\end{equation}
%
so
%
\begin{equation}\label{eqn:qmTwoL11:250}
\braket{\Br}{\psi'}
= \braket{\Br -\Ba}{\psi},
\end{equation}
%
or
%
\begin{equation}\label{eqn:qmTwoL11:270}
\psi'(\Br) = \psi(\Br - \Ba)
\end{equation}
%
This is what we expect of a translated function, as illustrated in \cref{fig:qmTwoL11:qmTwoL11fig2}.
\imageFigure{../figures/phy456-qmII/qmTwoL11fig2}{Active spatial translation}{fig:qmTwoL11:qmTwoL11fig2}{0.2}
}

\makeexample{Spatial rotation}{example:qmTwoL11:3}{
We have been introduced to the angular momentum operator
%
\begin{equation}\label{eqn:qmTwoL11:290}
\BL = \BR \cross \BP,
\end{equation}
%
where
\begin{equation}\label{eqn:qmTwoL11:310}
\begin{aligned}
L_x &= Y P_z - Z P_y \\
L_y &= Z P_x - X P_z \\
L_z &= X P_y - Y P_x.
\end{aligned}
\end{equation}
%
We also found that
%
\begin{equation}\label{eqn:qmTwoL11:330}
\antisymmetric{L_i}{L_j} = i \Hbar \sum_k \epsilon_{ijk} L_k.
\end{equation}
%
These non-zero commutators show that the components of angular momentum \textunderline{do not} commute.

Define
%
\begin{equation}\label{eqn:qmTwoL11:350}
\ket{\calR(\Br)} =
e^{-i \theta \ncap \cdot \BL/\Hbar}
\ket{\Br} .
\end{equation}
%
This is the vector that we get by actively rotating the vector \(\Br\) by an angle \(\theta\) counterclockwise about \(\ncap\), as in \cref{fig:qmTwoL11:qmTwoL11fig3}.
\imageFigure{../figures/phy456-qmII/qmTwoL11fig3}{Active vector rotations}{fig:qmTwoL11:qmTwoL11fig3}{0.2}
An active rotation rotates the vector, leaving the coordinate system fixed, whereas a passive rotation is one for which the coordinate system is rotated, and the vector is left fixed.

Note that rotations do not commute.  Suppose that we have a pair of rotations as in \cref{fig:qmTwoL11:qmTwoL11fig4}.
\imageFigure{../figures/phy456-qmII/qmTwoL11fig4}{A example pair of non-commuting rotations}{fig:qmTwoL11:qmTwoL11fig4}{0.2}
Again, we get the graphic demo, with Professor Sipe rotating the big wooden cat sculpture.  Did he bring that in to class just to make this point (too bad I missed the first couple minutes of the lecture).

Rather amusingly, he points out that most things in life do not commute.  We get much different results if we apply the operations of putting water into the teapot and turning on the stove in different orders.
\paragraph{Rotating a ket}
With a rotation gadget
%
\begin{equation}\label{eqn:qmTwoL11:370}
\ket{\psi'} =
e^{-i \theta \ncap \cdot \BL/\Hbar }
\ket{\psi},
\end{equation}
%
we can form the matrix element
\begin{equation}\label{eqn:qmTwoL11:390}
\braket{\Br}{\psi'} =
\bra{\Br} e^{-i \theta \ncap \cdot \BL/\Hbar }
\ket{\psi}.
\end{equation}
%
In this we have
\begin{equation}\label{eqn:qmTwoL11:690}
\begin{aligned}
\bra{\Br} e^{-i \theta \ncap \cdot \BL/\Hbar }
&=
\left( e^{i \theta \ncap \cdot \BL/\Hbar } \ket{\Br} \right)^\dagger \\
&=
\left( \ket{\calR^{-1}(\Br) } \right)^\dagger,
\end{aligned}
\end{equation}
%
so
\begin{equation}\label{eqn:qmTwoL11:410}
\braket{\Br}{\psi'} =
\braket{\calR^{-1}(\Br) }{\psi'},
\end{equation}
%
or
\begin{equation}\label{eqn:qmTwoL11:430}
\psi'(\Br) = \psi( \calR^{-1}(\Br) )
\end{equation}
}

\section{Generalizations}

Recall what you did last year, where \(H\), \(\BP\), and \(\BL\) were defined mechanically.  We found

\begin{itemize}
\item \(H\) generates time evolution (or translation in time).
\item \(\BP\) generates spatial translation.
\item \(\BL\) generates spatial rotation.
\end{itemize}

For our mechanical definitions we have
%
\begin{equation}\label{eqn:qmTwoL11:450}
\antisymmetric{P_i}{P_j} = 0,
\end{equation}
%
and
%
\begin{equation}\label{eqn:qmTwoL11:470}
\antisymmetric{L_i}{L_j} = i \Hbar \sum_k \epsilon_{ijk} L_k.
\end{equation}
%
These are the relations that show us the way translations and rotations combine.  We want to move up to a higher plane, a new level of abstraction.  To do so we \textunderline{define} \(H\) as the operator that generates time evolution.  If we have a theory that covers the behavior of how anything evolves in time, \(H\) encodes the rules for this time evolution.

\textunderline{Define} \(\BP\) as the operator that generates translations in space.

\textunderline{Define} \(\BJ\) as the operator that generates rotations in space.

In order that these match expectations, we require
%
\begin{equation}\label{eqn:qmTwoL11:490}
\antisymmetric{P_i}{P_j} = 0,
\end{equation}
%
and
%
\begin{equation}\label{eqn:qmTwoL11:510}
\antisymmetric{J_i}{J_j} = i \Hbar \sum_k \epsilon_{ijk} J_k.
\end{equation}
%
In the simple theory of a spin less particle we have
%
\begin{equation}\label{eqn:qmTwoL11:530}
\BJ \equiv \BL = \BR \cross \BP.
\end{equation}
%
We actually need a generalization of this since this is, in fact, not good enough, even for low energy physics.
%
\paragraph{Many component wave functions}
%
We are free to construct tuples of spatial vector functions like
%
\begin{equation}\label{eqn:qmTwoL11:550}
\begin{bmatrix}
\Psi_I(\Br, t) \\
\Psi_{II}(\Br, t)
\end{bmatrix},
\end{equation}
%
or
\begin{equation}\label{eqn:qmTwoL11:570}
\begin{bmatrix}
\Psi_I(\Br, t) \\
\Psi_{II}(\Br, t) \\
\Psi_{III}(\Br, t)
\end{bmatrix},
\end{equation}
%
etc.

We will see that these behave qualitatively different than one component wave functions.  We also do not have to be considering multiple particle wave functions, but just \textunderline{one} particle that requires three functions in \R{3} to describe it (ie: we are moving in on spin).
%
\paragraph{Question:} Do these live in the same vector space?
\paragraph{Answer:} We will get to this.
%
\paragraph{A classical analogy}
%
``There is only bad analogies, since if the are good they would be describing the same thing.  We can however, produce some useful bad analogies''

\begin{enumerate}
\item A temperature field
%
\begin{equation}\label{eqn:qmTwoL11:590}
T(\Br)
\end{equation}
%
\item Electric field
%
\begin{equation}\label{eqn:qmTwoL11:610}
\begin{bmatrix}
E_x(\Br) \\
E_y(\Br) \\
E_z(\Br)
\end{bmatrix}
\end{equation}
%
\end{enumerate}

These behave in a much different way.  If we rotate a scalar field like \(T(\Br)\) as in \cref{fig:qmTwoL11:qmTwoL11fig5}.
\imageFigure{../figures/phy456-qmII/qmTwoL11fig5}{Rotated temperature (scalar) field}{fig:qmTwoL11:qmTwoL11fig5}{0.2}
Suppose we have a temperature field generated by, say, a match.  Rotating the match above, we have
%
\begin{equation}\label{eqn:qmTwoL11:630}
T'(\Br) = T(\calR^{-1}(\Br)).
\end{equation}
%
Compare this to the rotation of an electric field, perhaps one produced by a capacitor, as in \cref{fig:qmTwoL11:qmTwoL11fig6}.
\imageFigure{../figures/phy456-qmII/qmTwoL11fig6}{Rotating a capacitance electric field}{fig:qmTwoL11:qmTwoL11fig6}{0.2}
Is it true that we have
\begin{equation}\label{eqn:qmTwoL11:650}
\begin{bmatrix}
E_x(\Br) \\
E_y(\Br) \\
E_z(\Br)
\end{bmatrix}
\questionEquals
\begin{bmatrix}
E_x(\calR^{-1}(\Br)) \\
E_y(\calR^{-1}(\Br)) \\
E_z(\calR^{-1}(\Br))
\end{bmatrix}
\end{equation}
%
\paragraph{No.}  Because the components get mixed as well as the positions at which those components are evaluated.
%
We will work with many component wave functions, some of which will behave like vectors, and will have to develop the methods and language to tackle this.



%
% Copyright � 2012 Peeter Joot.  All Rights Reserved.
% Licenced as described in the file LICENSE under the root directory of this GIT repository.
%

%
%
%\input{../peeter_prologue_print.tex}
%\input{../peeter_prologue_widescreen.tex}

%\chapter{PHY456H1F: Quantum Mechanics II.  Lecture 13 (Taught by Prof J.E. Sipe).  Spin and spinors (cont.)}
%\chapter{Spin and spinors (cont.)}
\index{spin}
\index{spinors}
\label{chap:qmTwoL13}

\blogpage{http://sites.google.com/site/peeterjoot/math2011/qmTwoL13.pdf}
%\date{Oct 24, 2011}





\section{Multiple wavefunction spaces}

Reading: See \S 26.5 in the text \citep{desai2009quantum}.

We identified
%
\begin{equation}\label{eqn:qmTwoL13:10}
\psi(\Br) = \braket{ \Br}{\psi}
\end{equation}
%
with improper basis kets
%
\begin{equation}\label{eqn:qmTwoL13:30}
\ket{\Br}
\end{equation}
%
Now introduce many function spaces
%
\begin{equation}\label{eqn:qmTwoL13:50}
\begin{bmatrix}
\psi_1(\Br) \\
\psi_2(\Br) \\
\vdots \\
\psi_\gamma(\Br)
\end{bmatrix}
\end{equation}
%
with improper (unnormalizable) basis kets
%
\begin{equation}\label{eqn:qmTwoL13:70}
\ket{\Br \alpha}, \qquad \alpha \in 1, 2, ... \gamma
\end{equation}
%
\begin{equation}\label{eqn:qmTwoL13:90}
\psi_\alpha(\Br) = \braket{ \Br\alpha}{\psi}
\end{equation}
%
for an abstract ket \(\ket{\psi}\)

We will try taking this Hilbert space
%
\begin{equation}\label{eqn:qmTwoL13:110}
H = H_o \otimes H_s
\end{equation}
%
Where \(H_o\) is the Hilbert space of "scalar" QM, ``o'' orbital and translational motion, associated with kets \(\ket{\Br}\) and \(H_s\) is the Hilbert space associated with the \(\gamma\) components \(\ket{\alpha}\).  This latter space we will label the ``spin'' or ``internal physics'' (class suggestion: or perhaps intrinsic).  This is ``unconnected'' with translational motion.

We build up the basis kets for \(H\) by direct products
%
\begin{equation}\label{eqn:qmTwoL13:130}
\ket{\Br \alpha} = \ket{\Br} \otimes \ket{\alpha}
\end{equation}
%
Now, for a rotated ket we seek a general angular momentum operator \(\BJ\) such that
%
\begin{equation}\label{eqn:qmTwoL13:150}
\ket{\psi'} = e^{-i \theta \ncap \cdot \BJ/\Hbar} \ket{\psi}
\end{equation}
%
where
%
\begin{equation}\label{eqn:qmTwoL13:170}
\BJ = \BL + \BS,
\end{equation}
%
where \(\BL\) acts over kets in \(H_o\), ``orbital angular momentum'', and \(\BS\) is the ``spin angular momentum'', acting on kets in \(H_s\).

Strictly speaking this would be written as direct products involving the respective identities
%
\begin{equation}\label{eqn:qmTwoL13:190}
\BJ = \BL \otimes I_s + I_o \otimes \BS.
\end{equation}
%
We require
%
\begin{equation}\label{eqn:qmTwoL13:210}
\antisymmetric{J_i}{J_j} = i \Hbar \sum \epsilon_{i j k} J_k
\end{equation}
%
Since \(\BL\) and \(\BS\) ``act over separate Hilbert spaces''.   Since these come from legacy operators
%
\begin{equation}\label{eqn:qmTwoL13:230}
\antisymmetric{L_i}{S_j} = 0
\end{equation}
%
We also know that
\begin{equation}\label{eqn:qmTwoL13:250}
\antisymmetric{L_i}{L_j} = i \Hbar \sum \epsilon_{i j k} L_k
\end{equation}
%
so
\begin{equation}\label{eqn:qmTwoL13:270}
\antisymmetric{S_i}{S_j} = i \Hbar \sum \epsilon_{i j k} S_k,
\end{equation}
%
as expected.  We could, in principle, have more complicated operators, where this would not be true.  This is a proposal of sorts.  Given such a definition of operators, let us see where we can go with it.

For matrix elements of \(\BL\) we have
%
\begin{equation}\label{eqn:qmTwoL13:290}
\bra{\Br} L_x \ket{\Br'} = -i \Hbar \left(
y \PD{z}{}
-z \PD{y}{} \right) \delta(\Br- \Br')
\end{equation}
%
What are the matrix elements of \(\bra{\alpha} S_i \ket{\alpha'}\)?  From the commutation relationships we know
%
\begin{equation}\label{eqn:qmTwoL13:310}
\sum_{\alpha'' = 1}^\gamma
\bra{\alpha} S_i \ket{\alpha''}
\bra{\alpha''} S_j \ket{\alpha'}
-
\sum_{\alpha'' = 1}^\gamma
\bra{\alpha} S_j \ket{\alpha''}
\bra{\alpha''} S_i \ket{\alpha'}
=
i \Hbar \sum_k \epsilon_{ijk}
\bra{\alpha} S_k \ket{\alpha''}
\end{equation}
%
We see that our matrix element is tightly constrained by our choice of commutator relationships.  We have \(\gamma^2\) such matrix elements, and it turns out that it is possible to choose (or find) matrix elements that satisfy these constraints?

The \(\bra{\alpha} S_i \ket{\alpha'}\) matrix elements that satisfy these constraints are found by imposing the commutation relations
%
\begin{equation}\label{eqn:qmTwoL13:330}
\antisymmetric{S_i}{S_j} = i \Hbar \sum \epsilon_{i j k} S_k,
\end{equation}
%
and with
%
\begin{equation}\label{eqn:qmTwoL13:350}
S^2 = \sum_j S_j^2,
\end{equation}
%
(this is just a definition).  We find
%
\begin{equation}\label{eqn:qmTwoL13:370}
\antisymmetric{S^2}{S_i} = 0
\end{equation}
%
and seeking eigenkets
%
\begin{equation}\label{eqn:qmTwoL13:390}
\begin{aligned}
S^2 \ket{s m_s} &= s(s+1) \Hbar^2 \ket{s m_s} \\
S_z \ket{s m_s} &= \Hbar m_s \ket{s m_s}
\end{aligned}
\end{equation}
%
Find solutions for \(s = 1/2, 1, 3/2, 2, \cdots\), where \(m_s \in \{-s, \cdots, s\}\).  ie.  \(2 s + 1\) possible vectors \(\ket{s m_s}\) for a given \(s\).
%
\begin{equation}\label{eqn:qmTwoL13:610}
\begin{aligned}
s = \inv{2} &\implies \gamma = 2 \\
s = 1 &\implies \gamma = 3 \\
s = \frac{3}{2} &\implies \gamma = 4
\end{aligned}
\end{equation}
%
We start with the algebra (mathematically the Lie algebra), and one can compute the Hilbert spaces that are consistent with these algebraic constraints.

We assume that for any type of given particle \(S\) is fixed, where this has to do with the nature of the particle.
%
\begin{equation}\label{eqn:qmTwoL13:630}
\begin{aligned}
s = \inv{2} &\qquad \text{A spin \(1/2\) particle} \\
s = 1 &\qquad \text{A spin \(1\) particle} \\
s = \frac{3}{2} &\qquad \text{A spin \(3/2\) particle}
\end{aligned}
\end{equation}
%
\(S\) is fixed once we decide that we are talking about a specific type of particle.

A non-relativistic particle in this framework has two nondynamical quantities.  One is the mass \(m\) and we now introduce a new invariant, the spin \(s\) of the particle.

This has been introduced as a kind of strategy.  It is something that we are going to try, and it turns out that it does.  This agrees well with experiment.

In 1939 Wigner asked, ``what constraints do I get if I constrain the constraints of quantum mechanics with special relativity.''  It turns out that in the non-relativistic limit, we get just this.

There is a subtlety here, because we get into some logical trouble with the photon with a rest mass of zero (\(m = 0\) is certainly allowed as a value of our invariant \(m\) above).  We can not stop or slow down a photon, so orbital angular momentum is only a conceptual idea.  Really, the orbital angular momentum and the spin angular momentum cannot be separated out for a photon, so talking of a spin \(1\) particle really means spin as in \(\BJ\), and not spin as in \(\BL\).
%
\paragraph{Spin one half particles}
%
Reading: See \S 26.6 in the text \citep{desai2009quantum}.

Let us start talking about the simplest case.  This includes electrons, all leptons (integer spin particles like photons and the weakly interacting W and Z bosons), and quarks.
%
\begin{equation}\label{eqn:qmTwoL13:410}
\begin{aligned}
s &= \inv{2} \\
m_s &= \pm \inv{2}
\end{aligned}
\end{equation}
%
states
%
\begin{equation}\label{eqn:qmTwoL13:430}
\ket{s m_s} = \ket{ \inv{2}, \inv{2} },\ket{ \inv{2}, -\inv{2} }
\end{equation}
%
Note there is a convention
%
\begin{equation}\label{eqn:qmTwoL13:450}
\begin{aligned}
\ket{ \inv{2} \overline{\inv{2}} } &= \ket{ \inv{2}, -\inv{2} } \\
\ket{ \inv{2} \inv{2} } &= \ket{ \inv{2} \inv{2} }
\end{aligned}
\end{equation}
%
\begin{equation}\label{eqn:qmTwoL13:470}
\begin{aligned}
S^2
\ket{\inv{2} m_s}
&=
\inv{2} \left( \inv{2} + 1 \right) \Hbar^2
\ket{\inv{2} m_s}  \\
&=
\frac{3}{4} \Hbar^2 \ket{\inv{2} m_s}  \\
\end{aligned}
\end{equation}
%
\begin{equation}\label{eqn:qmTwoL13:490}
S_z
\ket{\inv{2} m_s}
=
m_s \Hbar
\ket{\inv{2} m_s}
\end{equation}
%
For shorthand
%
\begin{equation}\label{eqn:qmTwoL13:510}
\begin{aligned}
\ket{ \inv{2} \inv{2} } &= \ket{ + } \\
\ket{ \inv{2} \overline{\inv{2}} } &= \ket{ - }
\end{aligned}
\end{equation}
%
\begin{equation}\label{eqn:qmTwoL13:530}
S^2 \rightarrow \frac{3}{4} \Hbar^2
\begin{bmatrix}
1 & 0 \\
0 & 1
\end{bmatrix}
\end{equation}
%
\begin{equation}\label{eqn:qmTwoL13:550}
S_z \rightarrow
\frac{\Hbar}{2}
\begin{bmatrix}
1 & 0 \\
0 & -1
\end{bmatrix}
\end{equation}
%
One can easily work out from the commutation relationships that
%
\begin{equation}\label{eqn:qmTwoL13:570}
S_x \rightarrow
\frac{\Hbar}{2}
\begin{bmatrix}
0 & 1 \\
1 & 0
\end{bmatrix}
\end{equation}
%
\begin{equation}\label{eqn:qmTwoL13:590}
S_y \rightarrow
\frac{\Hbar}{2}
\begin{bmatrix}
0 & -i \\
i & 0
\end{bmatrix}
\end{equation}
%
We will start with adding \(\BL\) into the mix on Wednesday.




%
% Copyright � 2012 Peeter Joot.  All Rights Reserved.
% Licenced as described in the file LICENSE under the root directory of this GIT repository.
%

%
%
%\input{../peeter_prologue_print.tex}
%\input{../peeter_prologue_widescreen.tex}

%\chapter{Representation of two state kets and Pauli spin matrices}
\index{spin matrix}
\index{Pauli matrix}
\index{two state ket}
%\chapter{PHY456H1F: Quantum Mechanics II.  Lecture 14 (Taught by Prof J.E. Sipe).  Representation of two state kets and Pauli spin matrices}
\label{chap:qmTwoL14}

\blogpage{http://sites.google.com/site/peeterjoot/math2011/qmTwoL14.pdf}
%\date{Oct 26, 2011}





\section{Representation of kets}
\index{representation}

Reading: \S 5.1 - \S 5.9 and \S 26 in \citep{desai2009quantum}.

We found the representations of the spin operators

\begin{align}\label{eqn:qmTwoL14:10}
S_x &\rightarrow \frac{\Hbar}{2} \PauliX \\
S_y &\rightarrow \frac{\Hbar}{2} \PauliY \\
S_z &\rightarrow \frac{\Hbar}{2} \PauliZ
\end{align}

How about kets?  For example for \(\ket{\chi} \in H_s\)
%
\begin{equation}\label{eqn:qmTwoL14:30}
\ket{\chi} \rightarrow
\begin{bmatrix}
\braket{+}{\chi} \\
\braket{-}{\chi}
\end{bmatrix},
\end{equation}
%
and

\begin{align}\label{eqn:qmTwoL14:640}
\ket{+} &\rightarrow
\begin{bmatrix}
1 \\
0
\end{bmatrix} \\
\ket{0} &\rightarrow
\begin{bmatrix}
0 \\
1
\end{bmatrix}
\end{align}

So, for example
%
\begin{equation}\label{eqn:qmTwoL14:170}
S_y\ket{+} \rightarrow \frac{\Hbar}{2} \PauliY
\begin{bmatrix}
1 \\
0
\end{bmatrix} =
\frac{i\Hbar}{2}
\begin{bmatrix}
0 \\
1
\end{bmatrix}
\end{equation}
%
Kets in \(H_o \otimes H_s\)
%
\begin{equation}\label{eqn:qmTwoL14:50}
\ket{\psi} \rightarrow
\begin{bmatrix}
\braket{\Br+}{\psi} \\
\braket{\Br-}{\psi}
\end{bmatrix}
=
\begin{bmatrix}
\psi_{+}(\Br) \\
\psi_{-}(\Br)
\end{bmatrix}.
\end{equation}
%
This is a ``spinor''

Put
%
\begin{equation}\label{eqn:qmTwoL14:660}
\begin{aligned}
\braket{\Br \pm}{\psi}
&= \psi_{\pm}(\Br) \\
&=
\psi_{+}
\begin{bmatrix}
1 \\
0
\end{bmatrix}
+
\psi_{-}
\begin{bmatrix}
0 \\
1
\end{bmatrix}
\end{aligned}
\end{equation}
%
with
%
\begin{equation}\label{eqn:qmTwoL14:680}
\braket{\psi}{\psi} = 1
\end{equation}
%
Use
%
\begin{equation}\label{eqn:qmTwoL14:700}
\begin{aligned}
I
&= I_o \otimes I_s \\
&=
\int d^3 \Br \ketbra{\Br}{\Br} \otimes \left(
\ketbra{+}{+}
+
\ketbra{-}{-}
\right) \\
&=
\int d^3 \Br \ketbra{\Br}{\Br} \otimes \sum_{\sigma=\pm}
\ketbra{\sigma}{\sigma} \\
&=
\sum_{\sigma = \pm}
\int d^3 \Br \ketbra{\Br \sigma}{\Br \sigma}
\end{aligned}
\end{equation}
%
So
%
\begin{equation}\label{eqn:qmTwoL14:720}
\begin{aligned}
\bra{\psi} I \ket{\psi}
&=
\sum_{\sigma = \pm}
\int d^3 \Br \braket{\psi}{\Br \sigma} \braket{\Br \sigma}{\psi}  \\
&=
\int d^3 \Br
\left(
\Abs{\psi_{+}(\Br)}^2
+
\Abs{\psi_{-}(\Br)}^2
\right)
\end{aligned}
\end{equation}
%
\paragraph{Alternatively}
%
\begin{equation}\label{eqn:qmTwoL14:740}
\begin{aligned}
\ket{\psi}
&= I \ket{\psi} \\
&=
\int d^3 \Br
\sum_{\sigma = \pm}
\ket{\Br \sigma}\braket{\Br \sigma}{\psi} \\
&=
\sum_{\sigma = \pm} \left(
\int d^3 \Br \psi_\sigma(\Br)
\right)
\ket{\Br \sigma} \\
&=
\sum_{\sigma = \pm}
\left(
\int d^3 \Br \psi_\sigma(\Br) \ket{\Br}
\right)
\otimes \ket{\sigma}
\end{aligned}
\end{equation}
%
In braces we have a ket in \(H_o\), let us call it
%
\begin{equation}\label{eqn:qmTwoL14:760}
\ket{\psi_\sigma} = \int d^3 \Br \psi_\sigma(\Br) \ket{\Br},
\end{equation}
%
then
%
\begin{equation}\label{eqn:qmTwoL14:780}
\ket{\psi} = \ket{\psi_{+}} \ket{+} + \ket{\psi_{-}} \ket{-}
\end{equation}
%
where the direct product \(\otimes\) is implied.

We can form a ket in \(H_s\) as
%
\begin{equation}\label{eqn:qmTwoL14:800}
\braket{\Br}{\psi} = \psi_{+}(\Br) \ket{+} + \psi_{-}(\Br) \ket{-}
\end{equation}
%
An operator \(O_o\) which acts on \(H_o\) alone can be promoted to \(O_o \otimes I_s\), which is now an operator that acts on \(H_o \otimes H_s\).  We are sometimes a little cavalier in notation and leave this off, but we should remember this.
%
\begin{equation}\label{eqn:qmTwoL14:190}
O_o \ket{\psi} = (O_o \ket{\psi+}) \ket{+}
+ (O_o \ket{\psi+}) \ket{+}
\end{equation}
%
and likewise
\begin{equation}\label{eqn:qmTwoL14:70}
O_s \ket{\psi} =
\ket{\psi+} (O_s \ket{+})
+
\ket{\psi-} (O_s \ket{-})
\end{equation}
%
and
%
\begin{equation}\label{eqn:qmTwoL14:90}
O_o O_s \ket{\psi} =
(O_o \ket{\psi+}) (O_s \ket{+})
+
(O_o \ket{\psi-}) (O_s \ket{-})
\end{equation}
%
Suppose we want to rotate a ket, we do this with a full angular momentum operator
%
\begin{equation}\label{eqn:qmTwoL14:110}
e^{-i \theta \ncap \cdot \BJ/\Hbar} \ket{\psi}
=
e^{-i \theta \ncap \cdot \BL/\Hbar}
e^{-i \theta \ncap \cdot \BS/\Hbar}
\ket{\psi}
\end{equation}
%
(recalling that \(\BL\) and \(\BS\) commute)

So
%
\begin{equation}\label{eqn:qmTwoL14:130}
e^{-i \theta \ncap \cdot \BJ/\Hbar} \ket{\psi}
=
(e^{-i \theta \ncap \cdot \BL/\Hbar} \ket{\psi+}) (e^{-i \theta \ncap \cdot \BS/\Hbar} \ket{+})
+
(e^{-i \theta \ncap \cdot \BL/\Hbar} \ket{\psi-}) (e^{-i \theta \ncap \cdot \BS/\Hbar} \ket{-})
\end{equation}
%
\paragraph{A simple example}
%
\begin{equation}\label{eqn:qmTwoL14:210}
\ket{\psi} =
\ket{\psi_+} \ket{+}
+
\ket{\psi_-} \ket{-}
\end{equation}
%
Suppose

\begin{align}\label{eqn:qmTwoL14:230}
\ket{\psi_+} &= \alpha \ket{\psi_0} \\
\ket{\psi_-} &= \beta \ket{\psi_0}
\end{align}

where
%
\begin{equation}\label{eqn:qmTwoL14:250}
\Abs{\alpha}^2 + \Abs{\beta}^2 = 1
\end{equation}
%
Then
%
\begin{equation}\label{eqn:qmTwoL14:150}
\ket{\psi} = \ket{\psi_0} \ket{\chi}
\end{equation}
%
where
%
\begin{equation}\label{eqn:qmTwoL14:270}
\ket{\chi} = \alpha \ket{+} + \beta \ket{-}
\end{equation}
%
for
%
\begin{equation}\label{eqn:qmTwoL14:290}
\braket{\psi}{\psi} = 1,
\end{equation}
%
\begin{equation}\label{eqn:qmTwoL14:310}
\braket{\psi_0}{\psi_0}
\braket{\chi}{\chi}  = 1
\end{equation}
%
so
%
\begin{equation}\label{eqn:qmTwoL14:330}
\braket{\psi_0}{\psi_0} = 1
\end{equation}
%
We are going to concentrate on the unentangled state of \eqnref{eqn:qmTwoL14:150}.

\begin{itemize}
\item
How about with
%
\begin{equation}\label{eqn:qmTwoL14:350}
\Abs{\alpha}^2 = 1, \beta = 0
\end{equation}
%
\(\ket{\chi}\) is an eigenket of \(S_z\) with eigenvalue \(\Hbar/2\).
\item
%
\begin{equation}\label{eqn:qmTwoL14:370}
\Abs{\beta}^2 = 1, \alpha = 0
\end{equation}
%
\(\ket{\chi}\) is an eigenket of \(S_z\) with eigenvalue \(-\Hbar/2\).
\item
What is \(\ket{\chi}\) if it is an eigenket of \(\ncap \cdot \BS\)?
\end{itemize}

FIXME: F1: standard spherical projection picture, with \(\ncap\) projected down onto the \(x,y\) plane at angle \(\phi\) and at an angle \(\theta\) from the \(z\) axis.

The eigenvalues will still be \(\pm \Hbar/2\) since there is nothing special about the \(z\) direction.
%
\begin{equation}\label{eqn:qmTwoL14:820}
\begin{aligned}
\ncap \cdot \BS &=
n_x S_x
+n_y S_y
+n_z S_z \\
&\rightarrow
\frac{\Hbar}{2}
\begin{bmatrix}
n_z & n_x - i n_y \\
n_x + i n_y & -n_z
\end{bmatrix} \\
&=
\frac{\Hbar}{2}
\begin{bmatrix}
\cos\theta & \sin\theta e^{-i\phi}
\sin\theta e^{i\phi} & -\cos\theta
\end{bmatrix}
\end{aligned}
\end{equation}
%
To find the eigenkets we diagonalize this, and we find representations of the eigenkets are

\begin{align}\label{eqn:qmTwoL14:390}
\ket{\ncap+} &\rightarrow
\begin{bmatrix}
\cos\left(\frac{\theta}{2}\right) e^{-i\phi/2} \\
\sin\left(\frac{\theta}{2}\right) e^{i\phi/2}
\end{bmatrix} \\
\ket{\ncap-} &\rightarrow
\begin{bmatrix}
-\sin\left(\frac{\theta}{2}\right) e^{-i\phi/2} \\
\cos\left(\frac{\theta}{2}\right) e^{i\phi/2}
\end{bmatrix},
\end{align}

with eigenvalues \(\Hbar/2\) and \(-\Hbar/2\) respectively.

So in the abstract notation, tossing the specific representation, we have

\begin{align}\label{eqn:qmTwoL14:410}
\ket{\ncap+} &\rightarrow
\cos\left(\frac{\theta}{2}\right) e^{-i\phi/2} \ket{+}
\sin\left(\frac{\theta}{2}\right) e^{i\phi/2}  \ket{-} \\
\ket{\ncap-} &\rightarrow
-\sin\left(\frac{\theta}{2}\right) e^{-i\phi/2} \ket{+}
\cos\left(\frac{\theta}{2}\right) e^{i\phi/2}  \ket{-}
\end{align}

\section{Representation of two state kets}

Every ket
\begin{equation}\label{eqn:qmTwoL14:840}
\ket{\chi} \rightarrow
\begin{bmatrix}
\alpha \\
\beta
\end{bmatrix}
\end{equation}
%
for which
%
\begin{equation}\label{eqn:qmTwoL14:430}
\Abs{\alpha}^2 + \Abs{\beta}^2 = 1
\end{equation}
%
can be written in the form \eqnref{eqn:qmTwoL14:390} for some \(\theta\) and \(\phi\), neglecting an overall phase factor.

For any ket in \(H_s\), that ket is ``spin up'' in some direction.

FIXME: show this.

\section{Pauli spin matrices}

It is useful to write

\begin{align}\label{eqn:qmTwoL14:450}
S_x
&= \frac{\Hbar}{2} \PauliX \equiv
\frac{\Hbar}{2} \sigma_x \\
S_y
&= \frac{\Hbar}{2} \PauliY \equiv
\frac{\Hbar}{2} \sigma_y \\
&= \frac{\Hbar}{2} \PauliZ \equiv
\frac{\Hbar}{2} \sigma_z
\end{align}

where
\begin{align}\label{eqn:qmTwoL14:470}
\sigma_x &= \PauliX \\
\sigma_y &= \PauliY \\
\sigma_z &= \PauliZ
\end{align}

These are the Pauli spin matrices.
%
\paragraph{Interesting properties}
%
\begin{itemize}
\item
%
\begin{equation}\label{eqn:qmTwoL14:490}
\antisymmetric{\sigma_i}{\sigma_j} = \sigma_i \sigma_j + \sigma_j \sigma_i = 0, \qquad \mbox{ if \(i < j\)}
\end{equation}
%
\item
%
\begin{equation}\label{eqn:qmTwoL14:860}
\sigma_x \sigma_y = i \sigma_z
\end{equation}
%
(and cyclic permutations)
\item
\begin{equation}\label{eqn:qmTwoL14:880}
\tr(\sigma_i) = 0
\end{equation}
%
\item
%
\begin{equation}\label{eqn:qmTwoL14:900}
(\ncap \cdot \Bsigma)^2 = \sigma_0
\end{equation}
%
where
%
\begin{equation}\label{eqn:qmTwoL14:510}
\ncap \cdot \Bsigma \equiv n_x \sigma_x + n_y \sigma_y + n_z \sigma_z,
\end{equation}
%
and
\begin{equation}\label{eqn:qmTwoL14:530}
\sigma_0 =
\begin{bmatrix}
1 & 0 \\
0 & 1
\end{bmatrix}
\end{equation}
%
(note \(\tr(\sigma_0) \ne 0\))
\item
\begin{align}\label{eqn:qmTwoL14:550}
\antisymmetric{\sigma_i}{\sigma_j} &= 2 \delta_{ij} \sigma_0 \\
\antisymmetric{\sigma_x}{\sigma_y} &= 2 i \sigma_z
\end{align}

(and cyclic permutations of the latter).

Can combine these to show that
%
\begin{equation}\label{eqn:qmTwoL14:570}
(\BA \cdot \Bsigma)
(\BB \cdot \Bsigma)
=
(\BA \cdot \BB) \sigma_0 + i (\BA \times \BB) \cdot \Bsigma
\end{equation}
%
where \(\BA\) and \(\BB\) are vectors (or more generally operators that commute with the \(\Bsigma\) matrices).
\item
%
\begin{equation}\label{eqn:qmTwoL14:590}
\tr(\sigma_i \sigma_j) = 2 \delta_{ij}
\end{equation}
%
\item
%
\begin{equation}\label{eqn:qmTwoL14:610}
\tr(\sigma_\alpha \sigma_\beta) = 2 \delta_{\alpha \beta},
\end{equation}
%
where \(\alpha, \beta = 0, x, y, z\)
\end{itemize}

Note that any complex matrix \(M\) can be written as
%
\begin{equation}\label{eqn:qmTwoL14:920}
\begin{aligned}
M &= \sum_\alpha m_a \sigma_\alpha \\
  &=
\begin{bmatrix}
m_0 + m_z & m_x - i m_y \\
m_x + i m_y & m_0 - m_z
\end{bmatrix}
\end{aligned}
\end{equation}
%
for any four complex numbers \(m_0, m_x, m_y, m_z\)

where
%
\begin{equation}\label{eqn:qmTwoL14:630}
m_\beta = \inv{2} \tr(M \sigma_\beta).
\end{equation}
%


%
% Copyright � 2012 Peeter Joot.  All Rights Reserved.
% Licenced as described in the file LICENSE under the root directory of this GIT repository.
%
%
%
%
%%\input{../peeter_prologue_print.tex}
%%\input{../peeter_prologue_widescreen.tex}
%
%%\title{An elaboration}
%%\chapter{mytitle}
%%\label{chap:template}
%
%%\blogpage{http://sites.google.com/site/peeterjoot/math2011/template.pdf}
%%\date{Oct XX, 2011}
%
%
%
%
%
%%\section{Motivation}
%%
%%\section{Guts}
%
%% Sorry for what ended up being an incoherent interjection in class today.  I have attached a note attempting to explain what I was referring to.
%
However, what I was referring to, was that a general rotation of a vector in a Pauli matrix basis
%
\begin{equation}\label{eqn:qmTwoL15:5000}
R(\sum a_k \sigma_k) = R( \Ba \cdot \Bsigma),
\end{equation}
%
can be expressed by sandwiching the Pauli vector representation by two half angle rotation operators like our spin 1/2 operators from class today
%
\begin{equation}\label{eqn:qmTwoL15:5020}
R( \Ba \cdot \Bsigma) = e^{-\theta \ucap \cdot \Bsigma \vcap \cdot \Bsigma/2} \Ba \cdot \Bsigma e^{\theta \ucap \cdot \Bsigma \vcap \cdot \Bsigma/2},
\end{equation}
%
where \(\ucap\) and \(\vcap\) are two non-colinear orthogonal unit vectors that define the oriented plane that we are rotating in.
%
For example, rotating in the \(x-y\) plane, with \(\ucap = \xcap\) and \(\vcap = \ycap\), we have
%
\begin{equation}\label{eqn:qmTwoL15:5040}
R( \Ba \cdot \Bsigma)
= e^{-\theta \sigma_1 \sigma_2/2} (a_1 \sigma_1 + a_2 \sigma_2 + a_3 \sigma_3) e^{\theta \sigma_1 \sigma_2/2}.
\end{equation}
%
Observe that these exponentials commute with \(\sigma_3\), leaving
%
\begin{equation}\label{eqn:qmTwoL15halfAngle:5120}
\begin{aligned}
R( \Ba \cdot \Bsigma)
&=
(a_1 \sigma_1 + a_2 \sigma_2) e^{\theta \sigma_1 \sigma_2} +
a_3 \sigma_3 \\
&=
(a_1 \sigma_1 + a_2 \sigma_2) (\cos\theta + \sigma_1 \sigma_2 \sin\theta)
+a_3 \sigma_3 \\
&=
\sigma_1 (a_1 \cos\theta - a_2 \sin\theta)
+ \sigma_2 (a_2 \cos\theta + a_1 \sin\theta)
+ \sigma_3 (a_3),
\end{aligned}
\end{equation}
%
yielding our usual coordinate rotation matrix.  Expressed in terms of a unit normal to that plane, we form the normal by multiplication with the unit spatial volume element \(I = \sigma_1 \sigma_2 \sigma_3\).  For example:
%
\begin{equation}\label{eqn:qmTwoL15:5060}
\sigma_1 \sigma_2 \sigma_3( \sigma_3 )
=
\sigma_1 \sigma_2,
\end{equation}
%
and can in general write a spatial rotation in a Pauli basis representation as a sandwich of half angle rotation matrix exponentials
%
\begin{equation}\label{eqn:qmTwoL15:5080}
R( \Ba \cdot \Bsigma)
=
e^{-I \theta (\ncap \cdot \Bsigma)/2}
(\Ba \cdot \Bsigma)
e^{I \theta (\ncap \cdot \Bsigma)/2},
\end{equation}
%
when \(\ncap \cdot \Ba = 0\) we get the complex-number like single sided exponential rotation exponentials (since \(\Ba \cdot \Bsigma\) commutes with \(\Bn \cdot \Bsigma\) in that case)
%
\begin{equation}\label{eqn:qmTwoL15:5100}
R( \Ba \cdot \Bsigma)
=
(\Ba \cdot \Bsigma )
e^{I \theta (\ncap \cdot \Bsigma)}.
\end{equation}
%
I believe it was pointed out in one of \citep{doran2003gap} or \citep{hestenes1999nfc} that rotations expressed in terms of half angle Pauli matrices has caused some confusion to students of quantum mechanics, because this \(2 \pi\) ``rotation'' only generates half of the full spatial rotation.  It was argued that this sort of confusion can be avoided if one observes that these half angle rotations exponentials are exactly what we require for general spatial rotations, and that a pair of half angle operators are required to produce a full spatial rotation.

The book \citep{doran2003gap} takes this a lot further, and produces a formulation of spin operators that is devoid of the normal scalar imaginary \(i\) (using the Clifford algebra spatial unit volume element instead), and also does not assume a specific matrix representation of the spin operators.  They argue that this leads to some subtleties associated with interpretation, but at the time I was attempting to read that text I did know enough QM to appreciate what they were doing, and have not had time to attempt a new study of that content.

Asked about this offline, our Professor says, ``Yes.... but I think this kind of result is essentially what I was saying about the 'rotation of operators' in lecture.  As to 'interpreting' the \(-1\), there are a number of different strategies and ways of thinking about things.  But I think the fact remains that a \(2 \pi\) rotation of a spinor replaces the spinor by \(-1\) times itself, no matter how you formulate things.''

That this double sided half angle construction to rotate a vector falls out of the Heisenberg picture is interesting.  Even in a purely geometric Clifford algebra context, I suppose that a vector can be viewed as an operator (acting on another vector it produces a scalar and a bivector, acting on higher grade algebraic elements one gets \(+1\), \(-1\) grade elements as a result).  Yet that is something that is true, independent of any quantum mechanics.  In the books I mentioned, this was not derived, but instead stated, and then proved.  That is something that I think deserves a bit of exploration.  Perhaps there is a more natural derivation possible using infinitesimal arguments ... I had guess that scalar or grade selection would take the place of an expectation value in such a geometric argument.

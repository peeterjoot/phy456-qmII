%
% Copyright � 2012 Peeter Joot.  All Rights Reserved.
% Licenced as described in the file LICENSE under the root directory of this GIT repository.
%
%
%
%
%\input{../peeter_prologue_print.tex}
%\input{../peeter_prologue_widescreen.tex}
%
%\chapter{Hydrogen atom with spin, and two spin systems}
\index{hydrogen atom}
\index{two spin system}
\label{chap:qmTwoL16}
%
%\blogpage{http://sites.google.com/site/peeterjoot/math2011/qmTwoL16.pdf}
%\date{Nov 2, 2011}
%
%
%
%
%
%\text{rel} = \text{rel}
%\text{electron} = \text{electron}
%\text{proton} = \text{proton}
%\text{CM} = \text{CM}
%
\section{The hydrogen atom with spin}
READING: what chapter of \citep{desai2009quantum} ?

For a spinless hydrogen atom, the Hamiltonian was
%
\begin{equation}\label{eqn:qmTwoL16:10}
H = H_{\text{CM}} \otimes H_{\text{rel}}
\end{equation}
%
where we have independent Hamiltonian's for the motion of the center of mass and the relative motion of the electron to the proton.

The basis kets for these could be designated \(\ket{\Bp_{\text{CM}}}\) and \(\ket{\Bp_{\text{rel}}}\) respectively.

Now we want to augment this, treating
%
\begin{equation}\label{eqn:qmTwoL16:30}
H = H_{\text{CM}} \otimes H_{\text{rel}} \otimes H_{\text{s}}
\end{equation}
%
where \(H_{\text{s}}\) is the Hamiltonian for the spin of the electron.  We are neglecting the spin of the proton, but that could also be included (this turns out to be a lesser effect).

We will introduce a Hamiltonian including the dynamics of the relative motion and the electron spin
%
\begin{equation}\label{eqn:qmTwoL16:50}
H_{\text{rel}} \otimes H_{\text{s}}
\end{equation}
%
Covering the Hilbert space for this system we will use basis kets
%
\begin{equation}\label{eqn:qmTwoL16:70}
\ket{nlm\pm}
\end{equation}
%
\begin{equation}\label{eqn:qmTwoL16:90}
\begin{aligned}
\ket{nlm+}
&\rightarrow
\begin{bmatrix}
\braket{\Br+}{nlm+} \\
\braket{\Br-}{nlm+} \\
\end{bmatrix}
=
\begin{bmatrix}
\Phi_{nlm}(\Br) \\
0
\end{bmatrix} \\
\ket{nlm-}
&\rightarrow
\begin{bmatrix}
\braket{\Br+}{nlm-} \\
\braket{\Br-}{nlm-} \\
\end{bmatrix}
=
\begin{bmatrix}
0 \\
\Phi_{nlm}(\Br)
\end{bmatrix}.
\end{aligned}
\end{equation}
%
Here \(\Br\) should be understood to really mean \(\Br_{\text{rel}}\).  Our full Hamiltonian, after introducing a magnetic pertubation is
%
\begin{equation}\label{eqn:qmTwoL16:110}
H =
\frac{P_{\text{CM}}^2}{2M}
+
\left(
\frac{P_{\text{rel}}^2}{2\mu}
-
\frac{e^2}{R_{\text{rel}}}
\right)
- \Bmu_0 \cdot \BB
- \Bmu_s \cdot \BB
\end{equation}
%
where
%
\begin{equation}\label{eqn:qmTwoL16:130}
M = m_{\text{proton}} + m_{\text{electron}},
\end{equation}
%
and
\begin{equation}\label{eqn:qmTwoL16:150}
\inv{\mu} = \inv{m_{\text{proton}}} + \inv{m_{\text{electron}}}.
\end{equation}
%
For a uniform magnetic field
%
\begin{align}\label{eqn:qmTwoL16:170}
\Bmu_0 &= \left( -\frac{e}{2 m c} \right) \BL \\
\Bmu_s &= g \left( -\frac{e}{2 m c} \right) \BS
\end{align}
%
We also have higher order terms (higher order multipoles) and relativistic corrections (like spin orbit coupling \citep{wiki:spinorbit}).

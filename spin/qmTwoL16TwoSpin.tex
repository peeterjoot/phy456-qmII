%
% Copyright � 2013 Peeter Joot.  All Rights Reserved.
% Licenced as described in the file LICENSE under the root directory of this GIT repository.
%
\section{Two spins}
\index{two spin}
\label{chap:qmTwoL16TwoSpin}

READING: \S 28 of \citep{desai2009quantum}.

\paragraph{Example}: Consider two electrons, 1 in each of 2 quantum dots.

\begin{equation}\label{eqn:qmTwoL16:190}
H = H_{1} \otimes H_{2}
\end{equation}
%
where \(H_1\) and \(H_2\) are both spin Hamiltonian's for respective 2D Hilbert spaces.  Our complete Hilbert space is thus a 4D space.

We will write

\begin{equation}\label{eqn:qmTwoL16:210}
\begin{aligned}
\ket{+}_1 \otimes \ket{+}_2 &= \ket{++} \\
\ket{+}_1 \otimes \ket{-}_2 &= \ket{+-} \\
\ket{-}_1 \otimes \ket{+}_2 &= \ket{-+} \\
\ket{-}_1 \otimes \ket{-}_2 &= \ket{--}
\end{aligned}
\end{equation}
%
Can introduce

\begin{equation}\label{eqn:qmTwoL16:230}
\begin{aligned}
\BS_1 &= \BS_1^{(1)} \otimes I^{(2)} \\
\BS_2 &= I^{(1)} \otimes \BS_2^{(2)}
\end{aligned}
\end{equation}
%
Here we ``promote'' each of the individual spin operators to spin operators in the complete Hilbert space.

We write

\begin{equation}\label{eqn:qmTwoL16:250}
\begin{aligned}
S_{1z}\ket{++} &= \frac{\Hbar}{2} \ket{++} \\
S_{1z}\ket{+-} &= \frac{\Hbar}{2} \ket{+-}
\end{aligned}
\end{equation}
%
Write
\begin{equation}\label{eqn:qmTwoL16:270}
\BS = \BS_1 + \BS_2,
\end{equation}
%
for the full spin angular momentum operator.  The \(z\) component of this operator is

\begin{equation}\label{eqn:qmTwoL16:290}
S_z = S_{1z} + S_{2z}
\end{equation}
%
\begin{equation}\label{eqn:qmTwoL16:310}
\begin{aligned}
S_z\ket{++} &= (S_{1z} + S_{2z}) \ket{++} = \left( \frac{\Hbar}{2} +\frac{\Hbar}{2} \right) \ket{++} = \Hbar \ket{++} \\
S_z\ket{+-} &= (S_{1z} + S_{2z}) \ket{+-} = \left( \frac{\Hbar}{2} -\frac{\Hbar}{2} \right) \ket{+-} = 0 \\
S_z\ket{-+} &= (S_{1z} + S_{2z}) \ket{-+} = \left( -\frac{\Hbar}{2} +\frac{\Hbar}{2} \right) \ket{-+} = 0 \\
S_z\ket{--} &= (S_{1z} + S_{2z}) \ket{--} = \left( -\frac{\Hbar}{2} -\frac{\Hbar}{2} \right) \ket{--} = -\Hbar \ket{--}
\end{aligned}
\end{equation}
%
So, we find that \(\ket{x x}\) are all eigenkets of \(S_z\).  These will also all be eigenkets of \(\BS_1^2 = S_{1x}^2 +S_{1y}^2 +S_{1z}^2\) since we have

\begin{equation}\label{eqn:qmTwoL16:330}
\begin{aligned}
S_1^2 \ket{x x} &= \Hbar^2 \left(\inv{2}\right) \left(1 + \inv{2}\right) \ket{x x} = \frac{3}{4} \Hbar^2 \ket{x x} \\
S_2^2 \ket{x x} &= \Hbar^2 \left(\inv{2}\right) \left(1 + \inv{2}\right) \ket{x x} = \frac{3}{4} \Hbar^2 \ket{x x}
\end{aligned}
\end{equation}
%
\begin{equation}\label{eqn:qmTwoL16:350}
\begin{aligned}
\BS^2 &=
(\BS_1
+\BS_2)
\cdot
(\BS_1
+\BS_2)  \\
&=
S_1^2 + S_2^2 + 2 \BS_1 \cdot \BS_2
\end{aligned}
\end{equation}
%
Note that we have a commutation assumption here \(\antisymmetric{S_{1i}}{S_{2i}} = 0\), since we have written \(2 \BS_1 \cdot \BS_2\) instead of \(\sum_i S_{1i}S_{2i} + S_{2i}S_{1i}\).  The justification for this appears to be the promotion of the spin operators in \eqnref{eqn:qmTwoL16:230} to operators in the complete Hilbert space, since each of these spin operators acts only on the kets associated with their index.

Are all the product kets also eigenkets of \(\BS^2\)?  Calculate

\begin{equation}\label{eqn:qmTwoL16TwoSpin:630}
\begin{aligned}
\BS^2 \ket{+-}
&=
(S_1^2 + S_2^2 + 2 \BS_1 \cdot \BS_2) \ket{+-} \\
&=
\left(\frac{3}{4}\Hbar^2
+\frac{3}{4}\Hbar^2\right)
+ 2 S_{1x} S_{2x} \ket{+-}
+ 2 S_{1y} S_{2y} \ket{+-}
+ 2 S_{1z} S_{2z} \ket{+-}
\end{aligned}
\end{equation}
%
For the \(z\) mixed terms, we have

\begin{equation}\label{eqn:qmTwoL16:370}
2 S_{1z} S_{2z} \ket{+-}  =
2
\left(\frac{\Hbar}{2}\right)
\left(-\frac{\Hbar}{2}\right)
\ket{+-}
\end{equation}
%
So

\begin{equation}\label{eqn:qmTwoL16:390}
\BS^2\ket{+-} =
\Hbar^2 \ket{+-}
+ 2 S_{1x} S_{2x} \ket{+-}
+ 2 S_{1y} S_{2y} \ket{+-}
\end{equation}
%
Since we have set our spin direction in the z direction with

\begin{equation}\label{eqn:qmTwoL16:410}
\begin{aligned}
\ket{+} &\rightarrow
\begin{bmatrix}
1 \\
0
\end{bmatrix} \\
\ket{-} &\rightarrow
\begin{bmatrix}
0 \\
1
\end{bmatrix}
\end{aligned}
\end{equation}
%
We have
\begin{equation}\label{eqn:qmTwoL16TwoSpin:650}
\begin{aligned}
S_x\ket{+}
&\rightarrow
\frac{\Hbar}{2} \PauliX
\begin{bmatrix}
1 \\
0
\end{bmatrix}
=
\frac{\Hbar}{2}
\begin{bmatrix}
0 \\
1
\end{bmatrix}
=
\frac{\Hbar}{2} \ket{-} \\
S_x\ket{-} &\rightarrow
\frac{\Hbar}{2} \PauliX
\begin{bmatrix}
0 \\
1
\end{bmatrix}
=
\frac{\Hbar}{2}
\begin{bmatrix}
1  \\
0
\end{bmatrix}
=
\frac{\Hbar}{2} \ket{+} \\
S_y\ket{+} &\rightarrow
\frac{\Hbar}{2} \PauliY
\begin{bmatrix}
1  \\
0
\end{bmatrix}
=
\frac{i\Hbar}{2}
\begin{bmatrix}
0  \\
1
\end{bmatrix}
=
\frac{i\Hbar}{2} \ket{-} \\
S_y\ket{-} &\rightarrow
\frac{\Hbar}{2} \PauliY
\begin{bmatrix}
0  \\
1
\end{bmatrix}
=
\frac{-i\Hbar}{2}
\begin{bmatrix}
1  \\
0
\end{bmatrix}
=
-\frac{i\Hbar}{2} \ket{+} \\
\end{aligned}
\end{equation}
%
And are able to arrive at the action of \(\BS^2\) on our mixed composite state

\begin{equation}\label{eqn:qmTwoL16:430}
\BS^2\ket{+-} = \Hbar^2 (\ket{+-} + \ket{-+} ).
\end{equation}
%
For the action on the \(\ket{++}\) state we have

\begin{equation}\label{eqn:qmTwoL16TwoSpin:670}
\begin{aligned}
\BS^2 \ket{++}
&=
\left(\frac{3}{4}\Hbar^2 +\frac{3}{4}\Hbar^2\right)
\ket{++}
+ 2 \frac{\Hbar^2}{4}
\ket{--}
+ 2 i^2 \frac{\Hbar^2}{4} \ket{--}
+2
\left(\frac{\Hbar}{2}\right)
\left(\frac{\Hbar}{2}\right)
\ket{++} \\
&=
2 \Hbar^2 \ket{++} \\
\end{aligned}
\end{equation}
%
and on the \(\ket{--}\) state we have

\begin{equation}\label{eqn:qmTwoL16TwoSpin:690}
\begin{aligned}
\BS^2 \ket{--}
&=
\left(\frac{3}{4}\Hbar^2 +\frac{3}{4}\Hbar^2\right)
\ket{--}
+ 2 \frac{(-\Hbar)^2}{4}
\ket{++}
+ 2 i^2 \frac{\Hbar^2}{4} \ket{++}
+2
\left(-\frac{\Hbar}{2}\right)
\left(-\frac{\Hbar}{2}\right)
\ket{--} \\
&=
2 \Hbar^2 \ket{--}
\end{aligned}
\end{equation}
%
All of this can be assembled into a tidier matrix form

\begin{equation}\label{eqn:qmTwoL16:450}
\BS^2
\rightarrow
\Hbar^2
\begin{bmatrix}
2 & 0 & 0 & 0 \\
0 & 1 & 1 & 0 \\
0 & 1 & 1 & 0 \\
0 & 0 & 0 & 2 \\
\end{bmatrix},
\end{equation}
%
where the matrix is taken with respect to the (ordered) basis

\begin{equation}\label{eqn:qmTwoL16:470}
\{
\ket{++},
\ket{+-},
\ket{-+},
\ket{--}
\}.
\end{equation}
%
However,

\begin{equation}\label{eqn:qmTwoL16:490}
\begin{aligned}
\antisymmetric{\BS^2}{S_z} &= 0 \\
\antisymmetric{S_i}{S_j} &= i \Hbar \sum_k \epsilon_{ijk} S_k
\end{aligned}
\end{equation}
%
(Also, \(\antisymmetric{\BS^2}{S_i} = 0\).)

It should be possible to find eigenkets of \(\BS^2\) and \(S_z\)

\begin{equation}\label{eqn:qmTwoL16:510}
\begin{aligned}
\BS^2 \ket{s m_s} &= s(s+1)\Hbar^2 \ket{s m_s} \\
S_z \ket{s m_s} &= \Hbar m_s \ket{s m_s}
\end{aligned}
\end{equation}
%
An orthonormal set of eigenkets of \(\BS^2\) and \(S_z\) is found to be

\begin{equation}\label{eqn:qmTwoL16:530}
\begin{array}{l l}
\ket{++}
& \mbox{\(s = 1\) and \(m_s = 1\)} \\
\inv{\sqrt{2}} \left( \ket{+-} + \ket{-+} \right)
& \mbox{\(s = 1\) and \(m_s = 0\)} \\
\ket{--}
& \mbox{\(s = 1\) and \(m_s = -1\)} \\
\inv{\sqrt{2}} \left( \ket{+-} - \ket{-+} \right)
& \mbox{\(s = 0\) and \(m_s = 0\)}
\end{array}
\end{equation}
%
The first three kets here can be grouped into a triplet in a 3D Hilbert space, whereas the last treated as a singlet in a 1D Hilbert space.

Form a grouping

\begin{equation}\label{eqn:qmTwoL16:550}
H = H_1 \otimes H_2
\end{equation}
%
Can write

\begin{equation}\label{eqn:qmTwoL16:570}
\inv{2} \otimes \inv{2} = 1 \oplus 0
\end{equation}
%
where the \(1\) and \(0\) here refer to the spin index \(s\).

\paragraph{Other examples}

Consider, perhaps, the \(l=5\) state of the hydrogen atom

\begin{equation}\label{eqn:qmTwoL16:590}
\begin{aligned}
J_1^2 \ket{j_1 m_1} &= j_1(j_1+1)\Hbar^2 \ket{j_1 m_1} \\
J_{1z} \ket{j_1 m_1} &= \Hbar m_1 \ket{j_1 m_1}
\end{aligned}
\end{equation}
%
\begin{equation}\label{eqn:qmTwoL16:610}
\begin{aligned}
J_2^2 \ket{j_2 m_2} &= j_2(j_2+1)\Hbar^2 \ket{j_2 m_2} \\
J_{2z} \ket{j_2 m_2} &= \Hbar m_2 \ket{j_2 m_2}
\end{aligned}
\end{equation}
%
Consider the Hilbert space spanned by \(\ket{j_1 m_1} \otimes \ket{j_2 m_2}\), a \((2 j_1 + 1)(2 j_2 + 1)\) dimensional space.  How to find the eigenkets of \(J^2\) and \(J_z\)?



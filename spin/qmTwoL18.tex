%
% Copyright � 2012 Peeter Joot.  All Rights Reserved.
% Licenced as described in the file LICENSE under the root directory of this GIT repository.
%

%
%
%\input{../peeter_prologue_print.tex}
%\input{../peeter_prologue_widescreen.tex}
%
%\chapter{PHY456H1F: Quantum Mechanics II.  Lecture 18 (Taught by Prof J.E. Sipe).  The Clebsch-Gordon convention for the basis elements of summed generalized angular momentum}
%\chapter{The Clebsch-Gordon convention for the basis elements of summed generalized angular momentum}
\index{Clebsch-Gordon}
\index{generalized angular momentum}
\label{chap:qmTwoL18}
%
%\blogpage{http://sites.google.com/site/peeterjoot/math2011/qmTwoL18.pdf}
%\date{Nov 12, 2011}
%
%
%
%
%
\section{Recap: table of two spin angular momenta.}
\index{angular momenta}
Recall our table
%
\begin{equation}\label{eqn:qmTwoL18:10}
\begin{array}{| l | l | l | l | l |}
\hline
j = & j_1 + j_2				& j_1 + j_2 -1 				& \cdots 	& j_1 - j_2 \\
\hline
\hline
  &  \ket{j_1 + j_2, j_1 + j_2}	 	&					& 		& \\
\hline
  &  \ket{j_1 + j_2, j_1 + j_2 - 1}	&  \ket{j_1 + j_2 - 1, j_1 + j_2 - 1}	& 		& \\
\hline
  &                                     & \ket{j_1 + j_2 - 1, j_1 + j_2 - 2}	& 		& \\
\hline
  & \vdots 	 			&					& 		& \ket{j_1 - j_2, j_1 - j_2} \\
\hline
  & \vdots 	 			&					& 		& \vdots \\
\hline
  & \vdots 	 			&					& 		& \ket{j_1 - j_2, -(j_1 - j_2)} \\
\hline
  & \vdots 	 			&					& 		& \\
\hline
  &  \ket{j_1 + j_2, -(j_1 + j_2 - 1)}	& \ket{j_1 + j_2 -1, -(j_1 + j_2 - 1)}	& 		& \\
\hline
  &  \ket{j_1 + j_2, -(j_1 + j_2)}	&					& 		&  \\
\hline
\end{array}
\end{equation}
%
\paragraph{First column}
%
Let us start with computation of the kets in the lowest position of the first column, which we will obtain by successive application of the lowering operator to the state
%
\begin{equation}\label{eqn:qmTwoL18:30}
\ket{j_1 + j_2, j_1 + j_2} = \ket{j_1 j_1} \otimes \ket{j_2 j_2}.
\end{equation}
%
Recall that our lowering operator was found to be (or defined as)
%
\begin{equation}\label{eqn:qmTwoL18:570}
J_{-} \ket{j, m} = \sqrt{(j+m)(j-m+1)} \Hbar \ket{j, m-1},
\end{equation}
%
so that application of the lowering operator gives us
%
\begin{equation}\label{eqn:qmTwoL18:610}
\begin{aligned}
\ket{j_1 + j_2, j_1 + j_2 -1}
&=
\frac{J_{-} \ket{j_1 j_1} \otimes \ket{j_2 j_2}}{
\left(2 (j_1+ j_2)\right)^{1/2} \Hbar
} \\
&=
\frac{(J_{1-} + J_{2-}) \ket{j_1 j_1} \otimes \ket{j_2 j_2}}{
\left(2 (j_1+ j_2)\right)^{1/2} \Hbar
} \\
&=
\frac{
\left(
\sqrt{(j_1 + j_1)(j_1 - j_1 + 1)} \Hbar
\ket{j_1(j_1 - 1)} \right) \otimes \ket{j_2 j_2}
}
{
\left(2 (j_1+ j_2)\right)^{1/2} \Hbar
} \\
&\quad+
\frac{
\ket{j_1 j_1} \otimes
\left(
\sqrt{(j_2 + j_2)(j_2 - j_2 + 1)} \Hbar
\ket{j_2(j_2 -1)}
\right)
}
{
\left(2 (j_1+ j_2)\right)^{1/2} \Hbar
} \\
&=
\left(\frac{j_1}{j_1 + j_2}\right)^{1/2}
\ket{j_1 (j_1-1)} \otimes \ket{j_2 j_2}
+
\left(\frac{j_2}{j_1 + j_2}\right)^{1/2}
\ket{j_1 j_1} \otimes \ket{j_2 (j_2-1)}
 \\
\end{aligned}
\end{equation}
%
Proceeding iteratively would allow us to finish off this column.
%
\paragraph{Second column}
%
Moving on to the second column, the top most element in the table
%
\begin{equation}\label{eqn:qmTwoL18:50}
\ket{j_1 + j_2 - 1, j_1 + j_2 -1} ,
\end{equation}
%
can only be made up of \(\ket{j_1 m_1} \otimes \ket{j_2 m_2}\) with \(m_1 + m_2 = j_1 + j_2 -1\).  There are two possibilities
%
\begin{equation}\label{eqn:qmTwoL18:70}
\begin{array}{l l l l}
m_1 &= j_1 	& m_2 &= j_2 - 1 \\
m_1 &= j_1 - 1  & m_2 &= j_2
\end{array}
\end{equation}
%
So for some \(A\) and \(B\) to be determined we must have
%
\begin{equation}\label{eqn:qmTwoL18:90}
\ket{j_1 + j_2 - 1, j_1 + j_2 -1}
=
A
\ket{j_1 j_1} \otimes \ket{j_2 (j_2-1)}
+
B
\ket{j_1 (j_1-1)} \otimes \ket{j_2 j_2}
\end{equation}
%
Observe that these are the same kets that we ended up with by application of the lowering operator on the topmost element of the first column in our table.  Since \(\ket{j_1 + j_2, j_1 + j_2 -1}\) and \(\ket{j_1 + j_2 - 1, j_1 + j_2 -1}\) are orthogonal, we can construct our ket for the top of the second column by just seeking such an orthonormal superposition.  Consider for example
%
\begin{equation}\label{eqn:qmTwoL18:630}
\begin{aligned}
0
&=
(a \bra{b} + c \bra{d})( A \ket{b} + C \ket{d}) \\
&=
a A + c C
\end{aligned}
\end{equation}
%
With \(A = 1\) we find that \(C = -a/c\), so we have
%
\begin{equation}\label{eqn:qmTwoL18:650}
\begin{aligned}
A \ket{b} + C \ket{d}
&=
\ket{b} - \frac{a}{c} \ket{d}  \\
&\sim
c \ket{b} - a \ket{d}  \\
\end{aligned}
\end{equation}
%
So we find, for real \(a\) and \(c\) that
%
\begin{equation}\label{eqn:qmTwoL18:590}
0 = (a \bra{b} + c \bra{d})( c \ket{b} - a \ket{d}),
\end{equation}
%
for any orthonormal pair of kets \(\ket{a}\) and \(\ket{d}\).  Using this we find
%
\begin{equation}\label{eqn:qmTwoL18:110}
\ket{j_1 + j_2 - 1, j_1 + j_2 -1}
=
\left(\frac{j_2}{j_1 + j_2}\right)^{1/2}
\ket{j_1 j_1} \otimes \ket{j_2 (j_2-1)}
-
\left(\frac{j_1}{j_1 + j_2}\right)^{1/2}
\ket{j_1 (j_1-1)} \otimes \ket{j_2 j_2}
\end{equation}
%
This will work, although we could also multiply by any phase factor if desired.  Such a choice of phase factors is essentially just a convention.
%
\paragraph{The Clebsch-Gordon convention}
%
This is the convention we will use, where we

\begin{itemize}
\item choose the coefficients to be real.
\item require the coefficient of the \(m_1 = j_1\) term to be \(\ge 0\)
\end{itemize}

This gives us the first state in the second column, and we can proceed to iterate using the lowering operators to get all those values.

Moving on to the third column
%
\begin{equation}\label{eqn:qmTwoL18:130}
\ket{j_1 + j_2 - 2, j_1 + j_2 -2}
\end{equation}
%
can only be made up of \(\ket{j_1 m_1} \otimes \ket{j_2 m_2}\) with \(m_1 + m_2 = j_1 + j_2 -2\).  There are now three possibilities
%
\begin{equation}\label{eqn:qmTwoL18:150}
\begin{array}{l l l l}
m_1 &= j_1	 &  m_2 &= j_2 - 2 \\
m_1 &= j_1 - 2  &  m_2 &= j_2 \\
m_1 &= j_1 - 1  &  m_2 &= j_2 - 1
\end{array}
\end{equation}
%
and 2 orthogonality conditions, plus conventions.  This is enough to determine the ket in the third column.

We can formally write
%
\begin{equation}\label{eqn:qmTwoL18:170}
\ket{jm ; j_1 j_2} =
\sum_{m_1, m_2}
\ket{ j_1 m_1, j_2 m_2}
\braket{ j_1 m_1, j_2 m_2}{jm ; j_1 j_2}
\end{equation}
%
where
\begin{equation}\label{eqn:qmTwoL18:190}
\ket{ j_1 m_1, j_2 m_2} = \ket{j_1 m_1} \otimes \ket{j_2 m_2},
\end{equation}
and
%
\begin{equation}\label{eqn:qmTwoL18:210}
\braket{ j_1 m_1, j_2 m_2}{jm ; j_1 j_2}
\end{equation}
are the Clebsch-Gordon coefficients, sometimes written as
%
\begin{equation}\label{eqn:qmTwoL18:230}
\braket{ j_1 m_1, j_2 m_2 }{ jm }
\end{equation}
%
Properties
\begin{enumerate}
\item \(\braket{ j_1 m_1, j_2 m_2 }{ jm } \ne 0\) only if \(j_1 - j_2 \le j \le j_1 + j+2\).
This is sometimes called the triangle inequality, depicted in \cref{fig:qmTwoL18:qmTwoL18fig1}.
\imageFigure{../figures/phy456-qmII/qmTwoL18fig1}{Angular momentum triangle inequality.}{fig:qmTwoL18:qmTwoL18fig1}{0.2}
\item \(\braket{ j_1 m_1, j_2 m_2 }{ jm } \ne 0\) only if \(m = m_1 + m_2\).
\item Real (convention).
\item \(\braket{ j_1 j_1, j_2 (j - j_1) }{ j j }\) positive (convention again).
\item Proved in the text.  If follows that
%
\begin{equation}\label{eqn:qmTwoL18:250}
\braket{ j_1 m_1, j_2 m_2 }{ j m }
=
(-1)^{j_1 + j_2 - j}
\braket{ j_1 (-m_1), j_2 (-m_2) }{ j (-m) }
\end{equation}
\end{enumerate}

Note that the \(\braket{ j_1 m_1, j_2 m_2 }{ j m }\) are all real.  So, they can be assembled into an orthogonal matrix.  Example
%
\begin{equation}\label{eqn:qmTwoL18:270}
\begin{bmatrix}
\ket{11} \\
\ket{10} \\
\ket{\overline{11}} \\
\ket{00}
\end{bmatrix}
=
\begin{bmatrix}
1 & 0 & 0 & 0 \\
0 & \inv{\sqrt{2}} & \inv{\sqrt{2}} & 0 \\
0 & 0 & 0 & 1 \\
0 & \inv{\sqrt{2}} & \frac{-1}{\sqrt{2}} & 0 \\
\end{bmatrix}
\begin{bmatrix}
\ket{++} \\
\ket{++} \\
\ket{-+} \\
\ket{--}
\end{bmatrix}
\end{equation}
%
\paragraph{Example.  Electrons}
%
Consider the special case of an electron, a spin \(1/2\) particle with \(s = 1/2\) and \(m_s = \pm 1/2\) where we have
%
\begin{equation}\label{eqn:qmTwoL18:290}
\BJ = \BL + \BS
\end{equation}
%
\begin{equation}\label{eqn:qmTwoL18:310}
\ket{lm} \otimes \ket{\inv{2} m_s}
\end{equation}
%
possible values of \(j\) are \(l \pm 1/2\)
%
\begin{equation}\label{eqn:qmTwoL18:330}
l \otimes \inv{2} =
\left(
l + \inv{2}
\right)
\oplus
\left(
l - \inv{2}
\right)
\end{equation}
%
Our table representation is then
%
\begin{equation}\label{eqn:qmTwoL18:350}
\begin{array}{| l | l | l |}
\hline
j = & l + \inv{2} 			& l - \inv{2} \\
\hline
\hline
  &  \ket{l + \inv{2}, l + \inv{2}}	 	&					 \\
\hline
  &  \ket{l + \inv{2}, l + \inv{2} - 1}	&  \ket{l - \inv{2}, l - \inv{2}}	 \\
\hline
  &                                     & \ket{l - \inv{2}, -(l - \inv{2}}	 \\
\hline
  &  \ket{l + \inv{2}, -(l + \inv{2})}	&					 \\
\hline
\end{array}
\end{equation}
%
Here \(\ket{l + \inv{2}, m}\)

can \textunderline{only} have contributions from
%
\begin{equation}\label{eqn:qmTwoL18:370}
\begin{aligned}
\ket{l, m-\inv{2}} &\otimes \ket{\inv{2}\inv{2}} \\
\ket{l, m+\inv{2}} &\otimes \ket{\inv{2}\overline{\inv{2}}}
\end{aligned}
\end{equation}
%
\(\ket{l - \inv{2}, m}\) from the same two.  So using this and conventions we can work out (in \S 28 page 524, of our text \citep{desai2009quantum}).
%
\begin{equation}\label{eqn:qmTwoL18:390}
\begin{aligned}
\ket{l\pm \inv{2}, m} &=
\pm
\inv{\sqrt{2 l + 1}}
(l + \inv{2} \pm m)^{1/2}
\ket{l, m - \inv{2}} \times \ket{\inv{2}\inv{2}} \\
&\pm
\inv{\sqrt{2 l + 1}}
(l + \inv{2} \mp m)^{1/2}
\ket{l, m + \inv{2}} \times \ket{\inv{2} \overline{\inv{2}}}
\end{aligned}
\end{equation}
%
\section{Tensor operators.}
\index{tensor operator}

\S 29 of the text.

Recall how we characterized a rotation
%
\begin{equation}\label{eqn:qmTwoL18:410}
\Br \rightarrow \calR(\Br).
\end{equation}
%
Here we are using an active rotation as depicted in \cref{fig:qmTwoL18:qmTwoL18fig2}.
\imageFigure{../figures/phy456-qmII/qmTwoL18fig2}{active rotation.}{fig:qmTwoL18:qmTwoL18fig2}{0.2}
Suppose that
%
\begin{equation}\label{eqn:qmTwoL18:430}
{\begin{bmatrix}
\calR(\Br)
\end{bmatrix}
}_i
=
\sum_j M_{ij} r_j
\end{equation}
%
so that
\begin{equation}\label{eqn:qmTwoL18:450}
U = e^{-i \theta \ncap \cdot \BJ/\Hbar}
\end{equation}
%
rotates in the same way.  Rotating a ket as in \cref{fig:qmTwoL18:qmTwoL18fig3}.
\imageFigure{../figures/phy456-qmII/qmTwoL18fig3}{Rotating a wavefunction.}{fig:qmTwoL18:qmTwoL18fig3}{0.2}
Rotating a ket
%
\begin{equation}\label{eqn:qmTwoL18:470}
\ket{\psi}
\end{equation}
%
using the prescription
%
\begin{equation}\label{eqn:qmTwoL18:490}
\ket{\psi'} = e^{-i \theta \ncap \cdot \BJ/\Hbar} \ket{\psi}
\end{equation}
%
and write
%
\begin{equation}\label{eqn:qmTwoL18:510}
\ket{\psi'} = U[M] \ket{\psi}
\end{equation}
%
Now look at
%
\begin{equation}\label{eqn:qmTwoL18:530}
\bra{\psi} \calO \ket{\psi}
\end{equation}
%
and compare with
%
\begin{equation}\label{eqn:qmTwoL18:550}
\bra{\psi'} \calO \ket{\psi'}
=
\bra{\psi} \mathLabelBox{U^\dagger[M] \calO U[M]}{\((\conj)\)} \ket{\psi}
\end{equation}
%
We will be looking in more detail at \((\conj)\).



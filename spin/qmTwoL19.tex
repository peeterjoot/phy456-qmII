%
% Copyright � 2012 Peeter Joot.  All Rights Reserved.
% Licenced as described in the file LICENSE under the root directory of this GIT repository.
%

%
%
%\input{../peeter_prologue_print.tex}
%\input{../peeter_prologue_widescreen.tex}

%\chapter{PHY456H1F: Quantum Mechanics II.  Lecture 19 (Taught by Prof J.E. Sipe).  Rotations of operators}
%\chapter{Rotations of operators}
\index{operator!rotation}
\label{chap:qmTwoL19}

\blogpage{http://sites.google.com/site/peeterjoot/math2011/qmTwoL19.pdf}
%\date{Nov 16, 2011}





\section{Setup}

READING: \S 28 \citep{desai2009quantum}.

Rotating with \(U[M]\) as in \cref{fig:qmTwoL19:qmTwoL19fig1}
\imageFigure{../figures/phy456-qmII/qmTwoL19fig1}{Rotating a state centered at \(F\)}{fig:qmTwoL19:qmTwoL19fig1}{0.2}
%
\begin{equation}\label{eqn:qmTwoL19:10}
\tilde{r}_i = \sum_j M_{ij} \overbar{r}_j
\end{equation}
%
\begin{equation}\label{eqn:qmTwoL19:30}
\bra{\psi} R_i \ket{\psi} = \overbar{r}_i
\end{equation}
%
\begin{equation}\label{eqn:qmTwoL19:430}
\begin{aligned}
\bra{\psi} U^\dagger[M] R_i U[M] \ket{\psi}
&= \tilde{r}_i = \sum_j M_{ij} \overbar{r}_j \\
&=
\bra{\psi} \Bigl( U^\dagger[M] R_i U[M] \Bigr) \ket{\psi}
\end{aligned}
\end{equation}
%
So
%
\begin{equation}\label{eqn:qmTwoL19:50}
U^\dagger[M] R_i U[M] = \sum_j M_{ij} R_j
\end{equation}
%
Any three operators \(V_x, V_y, V_z\) that transform according to
%
\begin{equation}\label{eqn:qmTwoL19:70}
U^\dagger[M] V_i U[M] = \sum_j M_{ij} V_j
\end{equation}
%
form the components of a \textunderline{vector operator}.

\section{Infinitesimal rotations}
\index{infinitesimal rotation}

Consider infinitesimal rotations, where we can show (problem set 11, problem 1) that
%
\begin{equation}\label{eqn:qmTwoL19:90}
\antisymmetric{V_i}{J_j} = i \Hbar \sum_k \epsilon_{ijk} V_k
\end{equation}
%
Note that for \(V_i = J_i\) we recover the familiar commutator rules for angular momentum, but this also holds for operators \(\BR\), \(\BP\), \(\BJ\), ...

Note that
%
\begin{equation}\label{eqn:qmTwoL19:110}
U^\dagger[M] = U[M^{-1}] = U[M^\T],
\end{equation}
%
so
%
\begin{equation}\label{eqn:qmTwoL19:130}
U^\dagger[M] V_i U^\dagger[M] = U^\dagger[M^\dagger] V_i U[M^\dagger] = \sum_j M_{ji} V_j
\end{equation}
%
so
%
\begin{equation}\label{eqn:qmTwoL19:150}
\bra{\psi} V_i \ket{\psi}
=
\bra{\psi}
U^\dagger[M] \Bigl( U[M] V_i U^\dagger[M] \Bigr) U[M]
\ket{\psi}
\end{equation}
%
In the same way, suppose we have nine operators
%
\begin{equation}\label{eqn:qmTwoL19:170}
\tau_{ij}, \qquad i, j = x, y, z
\end{equation}
%
that transform according to
%
\begin{equation}\label{eqn:qmTwoL19:190}
U[M] \tau_{ij} U^\dagger[M] = \sum_{lm} M_{li} M_{mj} \tau_{lm}
\end{equation}
%
then we will call these the components of (Cartesian) a second rank tensor operator.  Suppose that we have an operator \(S\) that transforms
%
\begin{equation}\label{eqn:qmTwoL19:210}
U[M] S U^\dagger[M] = S
\end{equation}
%
Then we will call \(S\) a scalar operator.

\section{A problem}

This all looks good, but it is really not satisfactory.  There is a problem.

Suppose that we have a Cartesian tensor operator like this, lets look at the quantity
%
\begin{equation}\label{eqn:qmTwoL19:450}
\begin{aligned}
\sum_i \tau_{ii}
&=
\sum_i
U[M] \tau_{ii} U^\dagger[M]  \\
&=
\sum_i
\sum_{lm} M_{li} M_{mi} \tau_{lm}
\\
&=
\sum_i
\sum_{lm} M_{li} M_{im}^\T \tau_{lm}
\\
&=
\sum_{lm} \delta_{lm} \tau_{lm}
\\
&=
\sum_{l} \tau_{ll}
\end{aligned}
\end{equation}
%
We see buried inside these Cartesian tensors of higher rank there is some simplicity embedded (in this case trace invariance).  Who knows what other relationships are also there?  We want to work with and extract the buried simplicities, and we will find that the Cartesian way of expressing these tensors is horribly inefficient.  What is a representation that does not have any excess information, and is in some sense minimal?

\section{How do we extract these buried simplicities?}

Recall
%
\begin{equation}\label{eqn:qmTwoL19:230}
U[M] \ket{j m''}
\end{equation}
%
gives a linear combination of the \(\ket{j m'}\).
%
\begin{equation}\label{eqn:qmTwoL19:470}
\begin{aligned}
U[M] \ket{j m''}
&=
\sum_{m'} \ket{j m'} \bra{j m'} U[M] \ket{j m''}
\\
&=
\sum_{m'} \ket{j m'}
D^{(j)}_{m' m''}[M]
\\
\end{aligned}
\end{equation}
%
We have talked about before how these \(D^{(j)}_{m' m''}[M]\) form a representation of the rotation group.  These are in fact (not proved here) an irreducible representation.

Look at each element of \(D^{(j)}_{m' m''}[M]\).  These are matrices and will be different according to which rotation \(M\) is chosen.  There is some \(M\) for which this element is nonzero.  There is no element in this matrix element that is zero for all possible \(M\).  There are more formal ways to think about this in a group theory context, but this is a physical way to think about this.

Think of these as the basis vectors for some eigenket of \(J^2\).
%
\begin{equation}\label{eqn:qmTwoL19:490}
\begin{aligned}
\ket{\psi}
&= \sum_{m''} \ket{j m''} \braket{j m''}{\psi} \\
&= \sum_{m''} \overbar{a}_{m''} \ket{j m''}
\end{aligned}
\end{equation}
%
where
%
\begin{equation}\label{eqn:qmTwoL19:250}
\overbar{a}_{m''} = \braket{j m''}{\psi}
\end{equation}
%
So
%
\begin{equation}\label{eqn:qmTwoL19:510}
\begin{aligned}
U[M] \ket{\psi} =
&= \sum_{m'} U[M] \ket{j m'} \braket{j m'}{\psi} \\
&= \sum_{m'} U[M] \ket{j m'} \overbar{a}_{m'} \\
&= \sum_{m', m''}
\ket{j m''} \bra{j m''}
U[M] \ket{j m'} \overbar{a}_{m'} \\
&= \sum_{m', m''}
\ket{j m''}
D^{(j)}_{m'', m'}
\overbar{a}_{m'} \\
&= \sum_{m''}
\tilde{a}_{m''}
\ket{j m''}
\end{aligned}
\end{equation}
%
where
%
\begin{equation}\label{eqn:qmTwoL19:270}
\tilde{a}_{m''} = \sum_{m'} D^{(j)}_{m'', m'} \overbar{a}_{m'} \\
\end{equation}
%
Recall that
%
\begin{equation}\label{eqn:qmTwoL19:290}
\tilde{r}_j = \sum_j M_{ij} \overbar{r}_j
\end{equation}
%
Define \((2k + 1)\) operators \({T_k}^q\), \(q = k, k-1, \cdots -k\) as the elements of a spherical tensor of rank \(k\) if
%
\begin{equation}\label{eqn:qmTwoL19:310}
U[M] {T_k}^q U^\dagger[M] = \sum_{q'} D^{(j)}_{q' q} {T_k}^{q'}
\end{equation}
%
Here we are looking for a better way to organize things, and it will turn out (not to be proved) that this will be an irreducible way to represent things.

\section{Motivating spherical tensors}
\index{spherical tensor}

We want to work though some examples of spherical tensors, and how they relate to Cartesian tensors.  To do this, a motivating story needs to be told.

Let us suppose that \(\ket{\psi}\) is a ket for a single particle.  Perhaps we are talking about an electron without spin, and write
%
\begin{equation}\label{eqn:qmTwoL19:530}
\begin{aligned}
\braket{\Br}{\psi}
&= Y_{lm}(\theta, \phi) f(r) \\
&= \sum_{m''} \overbar{a}_{m''} Y_{l m''}(\theta, \phi)
\end{aligned}
\end{equation}
%
for \(\overbar{a}_{m''} = \delta_{m'' m}\) and after dropping \(f(r)\).  So
%
\begin{equation}\label{eqn:qmTwoL19:330}
\bra{\Br} U[M] \ket{\psi}
=
\sum_{m''}
\sum_{m'}
D^{(j)}_{m'' m} \overbar{a}_{m'} Y_{l m''}(\theta, \phi)
\end{equation}
%
We are writing this in this particular way to make a point.  Now also assume that
%
\begin{equation}\label{eqn:qmTwoL19:350}
\braket{\Br}{\psi} = Y_{lm}(\theta, \phi)
\end{equation}
%
so we find
%
\begin{equation}\label{eqn:qmTwoL19:550}
\begin{aligned}
\bra{\Br} U[M] \ket{\psi}
&=
\sum_{m''}
Y_{l m''}(\theta, \phi)
D^{(j)}_{m'' m} \\
&=
Y_{l m}(\theta, \phi)
\end{aligned}
\end{equation}
%
\begin{equation}\label{eqn:qmTwoL19:370}
Y_{l m}(\theta, \phi)  = Y_{lm}(x, y, z)
\end{equation}
%
so
%
\begin{equation}\label{eqn:qmTwoL19:390}
Y'_{l m}(x, y, z)
=
\sum_{m''}
Y_{l m''}(x, y, z)
D^{(j)}_{m'' m}
\end{equation}
%
Now consider the spherical harmonic as an operator \(Y_{l m}(X, Y, Z)\)
%
\begin{equation}\label{eqn:qmTwoL19:410}
U[M] Y_{lm}(X, Y, Z) U^\dagger[M] =
\sum_{m''}
Y_{l m''}(X, Y, Z)
D^{(j)}_{m'' m}
\end{equation}
%
So this is a way to generate spherical tensor operators of rank \(0, 1, 2, \cdots\).


%
% Copyright � 2012 Peeter Joot.  All Rights Reserved.
% Licenced as described in the file LICENSE under the root directory of this GIT repository.
%
%
%
%
%\input{../peeter_prologue_print.tex}
%\input{../peeter_prologue_widescreen.tex}
%
%\chapter{PHY456H1F: Quantum Mechanics II.  Lecture 20 (Taught by Prof J.E. Sipe).  Spherical tensors}
%\chapter{Spherical tensors}
\index{spherical tensor}
\label{chap:qmTwoL20}
%
%\blogpage{http://sites.google.com/site/peeterjoot2/math2011/qmTwoL20.pdf}
%\date{Nov 21, 2011}
%
%
%
%
%
\section{Spherical tensors (cont.)}
READING: \S 29 of \citep{desai2009quantum}.
%
\paragraph{Definition.}  Any \((2k + 1)\) operator \(T(k, q)\), \(q = -k, \cdots, k\) are the elements of a spherical tensor of rank \(k\) if
%
\begin{equation}\label{eqn:qmTwoL20:10}
U[M] T(k, q) U^{-1}[M]
= \sum_{q'} T(k, q') D^{(k)}_{q q'}
\end{equation}
%
where \(D^{(k)}_{q q'}\) was the matrix element of the rotation operator
%
\begin{equation}\label{eqn:qmTwoL20:20}
D^{(k)}_{q q'} = \bra{k q'} U[M] \ket{k q''}.
\end{equation}
%
So, if we have a Cartesian vector operator with components \(V_x, V_y, V_z\) then we can construct a corresponding spherical vector operator
%
\begin{equation}\label{eqn:qmTwoL20:30}
\begin{array}{l l l}
T(1, 1) &= - \frac{V_x + i V_y}{\sqrt{2}} &\equiv V_{+1} \\
T(1, 0) &= V_z &\equiv V_0 \\
T(1, -1) &= - \frac{V_x - i V_y}{\sqrt{2}} &\equiv V_{-1}
\end{array}.
\end{equation}
%
By considering infinitesimal rotations we can come up with the commutation relations between the angular momentum operators
%
\begin{equation}\label{eqn:qmTwoL20:50}
\begin{aligned}
\antisymmetric{J_{\pm}}{T(k, q)} &= \Hbar \sqrt{(k \mp q)(k \pm q + 1)} T(k, q \pm 1) \\
\antisymmetric{J_{z}}{T(k, q)} &= \Hbar q T(k, q)
\end{aligned}
\end{equation}
%
Note that the text in (29.15) defines these, whereas in class these were considered consequences of \eqnref{eqn:qmTwoL20:10}, once infinitesimal rotations were used.

Recall that these match our angular momentum raising and lowering identities
%
\begin{equation}\label{eqn:qmTwoL20:50b}
\begin{aligned}
J_{\pm} \ket{k q} &= \Hbar \sqrt{(k \mp q)(k \pm q + 1)} \ket{k, q \pm 1} \\
J_{z} \ket{k q} &= \Hbar q \ket{k, q}.
\end{aligned}
\end{equation}
%
Consider two problems
%
\begin{equation}\label{eqn:qmTwoL20:70}
\begin{array}{l l l}
T(k, q)						& & \ket{k q} \\
\antisymmetric{J_{\pm}}{T(k, q)} 		&\leftrightarrow &J_{\pm} \ket{k q} \\
\antisymmetric{J_{z}}{T(k, q)} 			&\leftrightarrow &J_{z} \ket{k q}
\end{array}
\end{equation}
%
We have a correspondence between the spherical tensors and angular momentum kets
%
\begin{equation}\label{eqn:qmTwoL20:330}
\begin{array}{l l l l}
T_1(k_1, q_1)&\qquad q_1 = -k_1, \cdots, k_1 		& \qquad \ket{k_1 q_1} 		& \ket{k_2 q_2} \\
T_2(k_2, q_2)&\qquad q_2 = -k_2, \cdots, k_2		& \qquad q_1 = -k_1, \cdots k_1 	& q_2 = -k_2, \cdots k_2 \\
\end{array}
\end{equation}
%
So, as we can write for angular momentum
\begin{equation}\label{eqn:qmTwoL20:410}
\begin{aligned}
\ket{kq} &= \sum_{q_1, q_2}
\ket{k_1, q_1}
\ket{k_2, q_2}
\mathLabelBox{\braket{ k_1 q_1 k_2 q_2 }{ k q}}{These are the C.G coefficients}  \\
\ket{k_1 q_1 ; k_2 q_2}
&=
\sum_{k, q'}
\ket{k q'} \braket{ k q'}{ k_1 q_1 k_2 q_2 }
\end{aligned}
\end{equation}
%
We also have for spherical tensors
%
\begin{equation}\label{eqn:qmTwoL20:430}
\begin{aligned}
T(k, q) &= \sum_{q_1, q_2}
T_1(k_1, q_1)
T_2(k_2, q_2)
\braket{ k_1 q_1 k_2 q_2 }{ k q}
	\\
T_1(k_1, q_1)
T_2(k_2, q_2)
&=
\sum_{k, q'}
T(k, q') \braket{ k q'}{ k_1 q_1 k_2 q_2 } &
\end{aligned}
\end{equation}
%
Can form eigenstates \(\ket{kq}\) of \((\text{total angular momentum})^2\) and (z-comp of the total angular momentum).
FIXME: this will not be proven, but we are strongly suggested to try this ourselves.
%
\begin{equation}\label{eqn:qmTwoL20:350}
\begin{array}{l l l}
\text{spherical tensor (3)} 				&\leftrightarrow &\text{Cartesian vector (3)} \\
(\text{spherical vector})(\text{spherical vector})	&		 &\text{Cartesian tensor}
\end{array}
\end{equation}
%
We can check the dimensions for a spherical tensor decomposition into rank 0, rank 1 and rank 2 tensors.
%
%ATABXX
\begin{equation}\label{eqn:qmTwoL20:370}
\begin{array}{l l l}
\text{spherical tensor rank \(0\)} 	&	(1)	&	(\text{Cartesian vector})(\text{Cartesian vector}) \\
\text{spherical tensor rank \(1\)} 	&	(3)	&	(3)(3) \\
\text{spherical tensor rank \(2\)} 	&	(5)	&       9 \\
\hline
\text{dimension check sum}       	&	 9	&         \\
\end{array}
\end{equation}
%
Or in the direct product and sum shorthand
%
\begin{equation}\label{eqn:qmTwoL20:90}
1 \otimes 1 = 0 \oplus 1 \oplus 2
\end{equation}
%
Note that this is just like problem 4 in problem set 10 where we calculated the CG kets for the \(1 \otimes 1 = 0 \oplus 1 \oplus 2\) decomposition starting from kets \(\ket{1 m}\ket{1 m'}\).
%
%ATABXX
\begin{equation}\label{eqn:qmTwoL20:390}
\begin{array}{l l l}
\ket{22}		&				& 		\\
\ket{21}		& \ket{11} 			& 		\\
\ket{20}		& \ket{10} 			& \ket{00} 	\\
\ket{2\overline{1}}	& \ket{1\overline{1}} 		& 		\\
\ket{2\overline{2}}	&				&
\end{array}
\end{equation}
%
\paragraph{Example.}
%
How about a Cartesian tensor of rank 3?
%
\begin{equation}\label{eqn:qmTwoL20:110}
A_{ijk}
\end{equation}
%
%ATABXX
\begin{equation}\label{eqn:qmTwoL20:450}
\begin{aligned}
1 \otimes 1 \otimes 1
&=
1 \otimes ( 0 \oplus 1 \oplus 2) \\
&=
(1 \otimes 0) \oplus (1 \otimes 1) \oplus (1 \otimes 2) \\
&=
\begin{array}{l l l l l l l l l l l l l l}
1 &\oplus   &(0 &\oplus &1 &\oplus &2) &\oplus &(3  &\oplus & 2 &\oplus &1) \\
3 &+        &1 &+      &3 &+      &5  &+       &7  &+      & 5 &+      &3 = 27
\end{array}
\end{aligned}
\end{equation}
%
\paragraph{Why bother?}
%
Consider a tensor operator \(T(k, q)\) and an eigenket of angular momentum \(\ket{\alpha j m}\), where \(\alpha\) is a degeneracy index.

Look at
%
\begin{equation}\label{eqn:qmTwoL20:470}
\begin{aligned}
T(k, q) \ket{\alpha j m}
U[M] T(k, q) \ket{\alpha j m}
&=
U[M] T(k, q) U^\dagger[M] U[M] \ket{\alpha j m} \\
&=
\sum_{q' m'}
D^{(k)}_{q q'}
D^{(j)}_{m m'}
T(k, q') \ket{\alpha j m'}
\end{aligned}
\end{equation}
%
This transforms like \(\ket{k q} \otimes \ket{j m}\).  We can say immediately
%
\begin{equation}\label{eqn:qmTwoL20:150}
\bra{\alpha' j' m'} T(k, q) \ket{\alpha j m} = 0
\end{equation}
%
unless
\begin{equation}\label{eqn:qmTwoL20:170}
\begin{aligned}
\Abs{k - j} &\le j' \le k + j \\
m' &= m + q
\end{aligned}
\end{equation}
%
This is the ``selection rule''.

Examples.

\begin{itemize}
\item Scalar \(T(0, 0)\)
%
\begin{equation}\label{eqn:qmTwoL20:190}
\bra{\alpha' j' m'} T(0, 0) \ket{\alpha j m} = 0 ,
\end{equation}
%
unless \(j = j'\) and \(m = m'\).
\item \(V_x, V_y, V_z\).  What are the non-vanishing matrix elements?
%
\begin{equation}\label{eqn:qmTwoL20:210}
V_x = \frac{ V_{-1} - V_{+1}}{\sqrt{2}}, \cdots
\end{equation}
%
\begin{equation}\label{eqn:qmTwoL20:230}
\bra{\alpha' j' m'} V_{x, y} \ket{\alpha j m} = 0 ,
\end{equation}
%
unless
\begin{equation}\label{eqn:qmTwoL20:250}
\begin{aligned}
\Abs{j - 1} &\le j' \le j + 1  \\
m' &= m \pm 1
\end{aligned}
\end{equation}
%
\begin{equation}\label{eqn:qmTwoL20:270}
\bra{\alpha' j' m'} V_{z} \ket{\alpha j m} = 0 ,
\end{equation}
%
unless
\begin{equation}\label{eqn:qmTwoL20:290}
\begin{aligned}
\Abs{j - 1} &\le j' \le j + 1  \\
m' &= m
\end{aligned}
\end{equation}
\end{itemize}

Very generally one can prove (the Wigner-Eckart theory in the text \S 29.3)
%
\begin{equation}\label{eqn:qmTwoL20:310}
\bra{\alpha_2 j_2 m_2} T(k, q) \ket{\alpha_1 j_1 m_1}
=
\bra{\alpha_2 j_2 } T(k) \ket{\alpha_1 j_1} \cdot
\braket{j_2 m_2}{k q_1 ; j_1 m_1}
\end{equation}
%
where we split into a ``reduced matrix element'' describing the ``physics'', and the CG coefficient for ``geometry'' respectively.


